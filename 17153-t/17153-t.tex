%%%%%%%%%%%%%%%%%%%%%%%%%%%%%%%%%%%%%%%%%%%%%%%%%%%%%%%%%%%%%%%%%%%%%%%%%%%
%% Project Gutenberg's Synthetische Geometrie der Kugeln und linearen    %%
%% Kugelsysteme, by Theodor Reye                                         %%
%%                                                                       %%
%% This eBook is for the use of anyone anywhere at no cost and with      %%
%% almost no restrictions whatsoever.  You may copy it, give it away or  %%
%% re-use it under the terms of the Project Gutenberg License included   %%
%% with this eBook or online at www.gutenberg.net                        %%
%%                                                                       %%
%%                                                                       %%
%% Packages and substitutions:                                           %%
%%                                                                       %%
%% book:    Basic book class. Required.                                  %%
%% amsmath: Basic AMS math package. Required.                            %%
%% amssymb: Basic AMS symbols                                            %%
%%          May be substituted by the commands                           %%
%%          \newcommand{\varkappa}{\kappa}                               %%
%%          \newcommand{\centerdot}{\mathrel{.}}                         %%
%%          placed after \begin{document}                                %%
%% babel:   Basic multiple-language support.                             %%
%%          May be replaced by                                           %%
%%          \usepackage{german}                                          %%
%% soul:    Hyphenation for letterspacing, underlining, etc.             %%
%%          May be substituted by the command                            %%
%%          \newcommand{\so}[1]{\textbf{#1}}                             %%
%%          placed after \begin{document}                                %%
%%                                                                       %%
%%                                                                       %%
%% Producer's Comments:                                                  %%
%%                                                                       %%
%% The page numbers in the static table of contents are gathered         %%
%% with LaTeX page references, hence the file should be compiled         %%
%% three times to get them right. Care has been taken to retain          %%
%% maximal flexibility such as to different LaTeX distributions, or      %%
%% individual preferences for paper or font size.                        %%
%%                                                                       %%
%% Both pdflatex (to generate pdf) and latex (to generate dvi, and       %%
%% from this with appropriate tools other formats) should work. The      %%
%% book contains no illustrations.                                       %%
%%                                                                       %%
%%                                                                       %%
%% Things to Check:                                                      %%
%%                                                                       %%
%% Spellcheck: OK                                                        %%
%% LaCheck: OK                                                           %%
%% Lprep/gutcheck: OK                                                    %%
%% PDF pages, excl. Gutenberg boilerplate:  88                           %%
%% PDF pages, incl. Gutenberg boilerplate:  98                           %%
%% ToC page numbers: OK                                                  %%
%% Images: None                                                          %%
%% Fonts: OK                                                             %%
%%                                                                       %%
%%                                                                       %%
%% Compile History:                                                      %%
%%                                                                       %%
%% May 05: kfg. Compiled with pdflatex.                                  %%
%%         pdflatex kugel.tex                                            %%
%%         pdflatex kugel.tex                                            %%
%%         pdflatex kugel.tex                                            %%
%%                                                                       %%
%% Nov 05: jt. Compiled with                                             %%
%%         pdfeTeX, Version 3.141592-1.21a-2.2 (MiKTeX 2.4)              %%
%%         pdflatex kugel.tex                                            %%
%%         pdflatex kugel.tex                                            %%
%%         pdflatex kugel.tex                                            %%
%%         with 2 minor Overfull \hbox warnings                          %%
%%                                                                       %%
%%         I also made what looks to be a good PDF with                  %%
%%         latex kugel.tex                                               %%
%%         latex kugel.tex                                               %%
%%         latex kugel.tex                                               %%
%%         dvipdfm kugel                                                 %%
%%         dvipdfm, version 0.13.2c for MiKTeX 2.4                       %%
%%         complained that it wanted the bakoma font                     %%
%%         but couldn't get it (because it was removed                   %%
%%         from MiKTeX distro for licensing reasons).                    %%
%%         However, it compiled successfully, as far as                  %%
%%         I could see.                                                  %%
%%                                                                       %%
%%                                                                       %%
%%%%%%%%%%%%%%%%%%%%%%%%%%%%%%%%%%%%%%%%%%%%%%%%%%%%%%%%%%%%%%%%%%%%%%%%%%%


\documentclass[oneside,leqno,11pt]{book}
\listfiles
\usepackage{amsmath}% required
\usepackage{amssymb}% recommended, see substituting commands

\usepackage[german]{babel}% recommended, see substitution

\usepackage{soul}% recommended, see substituting commands

\begin{document}
\newcommand{\abschnitt}[1]{\subsection*{\begin{center}#1\end{center}}}

\thispagestyle{empty}
\small
\begin{verbatim}
Project Gutenberg's Synthetische Geometrie der Kugeln und linearen
Kugelsysteme, by Theodor Reye

This eBook is for the use of anyone anywhere at no cost and with
almost no restrictions whatsoever.  You may copy it, give it away or
re-use it under the terms of the Project Gutenberg License included
with this eBook or online at www.gutenberg.net


Title: Synthetische Geometrie der Kugeln und linearen
       Kugelsysteme

Author: Theodor Reye

Release Date: November 25, 2005 [EBook #17153]

Language: German

Character set encoding: TeX

*** START OF THIS PROJECT GUTENBERG EBOOK SYNTHETISCHE GEOMETRIE ***




Produced by K.F. Greiner, Joshua Hutchinson and the Online
Distributed Proofreading Team at http://www.pgdp.net from
images generously made available by Cornell University
Digital Collections.



\end{verbatim}
\normalsize
\newpage



%-----File: Titlepage.png-------------------------------

\frontmatter
\thispagestyle{empty}
\begin{center}
\vspace{1cm}

{\LARGE SYNTHETISCHE}
\bigskip\bigskip

{\Huge GEOMETRIE DER KUGELN}
\bigskip\bigskip

{\large UND}
\bigskip\bigskip

{\LARGE LINEAREN KUGELSYSTEME}
\bigskip\bigskip
\bigskip\bigskip

{\large MIT EINER EINLEITUNG}
\bigskip\bigskip

{\large IN DIE ANALYTISCHE GEOMETRIE DER KUGELSYSTEME}
\bigskip\bigskip\bigskip\bigskip

VON
\bigskip\bigskip\bigskip\bigskip

\textsc{\LARGE Dr. TH. REYE}
\bigskip

O. PROFESSOR AN DER UNIVERSIT\"AT STRASSBURG

\vfill

{\large
LEIPZIG \medskip

DRUCK UND VERLAG VON B.~G.~TEUBNER \medskip

1879
}
\end{center}

\newpage
\thispagestyle{empty}
\mainmatter

%-----File: 006.png-------------------------------
%[Blank Page]
%-----File: 007.png---------------------------------

\abschnitt{Vorwort.}


\hspace{-0.8pt}%
Die synthetische Geometrie der Kreise und Kugeln verdankt
den Auf\-schwung, welchen sie im Anfange unseres
Jahrhunderts genommen hat, haupt\-s\"achlich den bekannten
Ber\"uhrungsproblemen des Apollonius von Perga. Die Aufgabe,
zu drei gegebenen Kreisen einen vierten sie ber\"uhrenden
Kreis zu construiren, war freilich nebst ihren zahlreichen
Specialf\"allen schon von Vieta (1600) mit den H\"ulfsmitteln
der Alten, und von Newton, Euler und N.~Fuss analytisch
gel\"ost worden, auch hatte bereits Fermat\footnote{)
  Fermat, de contactibus sphaericis. (Varia opera mathematica,
  Tolosae 1679, fol.)})
von dem analogen
Problem f\"ur Kugeln eine synthetische Auf\/l\"osung gegeben.
Gleichwohl dienten diese Apollonischen Aufgaben noch lange
den Mathematikern zur fruchtbaren Anregung.

Zu neuen Auf\/l\"osungen dieser Ber\"uhrungsprobleme gelangten
zuerst einige Sch\"uler von Monge, indem sie die
Bewegung einer ver\"anderlichen Kugel untersuchten, welche
drei gegebene Kugeln fortw\"ahrend ber\"uhrt. Dupuis entdeckte
und Hachette\footnote{)
  Correspondance sur l'Ecole polytechnique, T.~I, S.~19;
  vgl.\ T.~II, S.~421.})
bewies (1804), dass der Mittelpunkt der
Kugel auf einem Kegelschnitte sich bewegt und dass ihre
Ber\"uhrungspunkte drei Kreise beschreiben. Bald darauf (1813)
ver\"offentlichte Dupin\footnote{)
  Ebenda T.~II, S.~420, und sp\"ater in seinen Applications de
  G\'{e}om\'{e}trie
  et de M\'{e}canique, Paris 1822.})
seine sch\"onen Untersuchungen \"uber
die merkw\"urdige, von jener ver\"anderlichen Kugel eingeh\"ullte
Fl\"ache, welcher er sp\"ater den Namen Cyclide beilegte; er
zeigte u.~A., dass diese Fl\"ache zwei Schaaren von kreisf\"ormigen
%-----File: 008.png--------------------------------
Kr\"ummungslinien besitzt, deren Ebenen durch zwei
zu einander rechtwinklige Gerade gehen. Fast gleichzeitig
(1812) f\"uhrte Gaultier\footnote{)
  Journal de l'Ecole polytechnique, $16^{\text{me}}$ cahier, 1813.})
die Potenzpunkte von Kreisen und
Kugeln sowie die Kreisb\"uschel und Kugelb\"uschel, wenn auch
unter anderen Namen, ein in die neuere Geometrie, und benutzte
dieselben zur L\"osung der Apollonischen Ber\"uhrungsprobleme.
Die Lehre von den Kreisb\"uscheln und von den
Aehnlichkeitspunkten mehrerer Kreise wurde sodann von
Pon\-ce\-let\footnote{)
  Poncelet, Trait\'e des propri\'et\'es projectives des figures, Paris
  1822;
  2.~Aufl.~1865.})
(1822) vervollkommnet und mit der Polarentheorie
des Kreises, deren Anf\"ange sich schon bei Monge\footnote{)
  Monge, G\'eom\'etrie descriptive, Paris 1795; $5^{\text{e}}$
  \'ed.~1827, S.~51.})
finden,
in Verbindung gebracht.

Vier Jahre sp\"ater (1826) erschienen die {\glqq}geometrischen
Betrachtungen{\grqq} von Jacob Steiner\footnote{)
  Crelle's Journal f\"ur die r.~u.~a.\ Mathematik, Bd.~1.}),
in welchen zum ersten
Male der Ausdruck {\glqq}Potenz{\grqq} bei Kreisen und Kugeln angewendet
wird. Indem er die Ber\"uhrung als speciellen Fall des
Schneidens auf\/fasst, erweitert Steiner in dieser Abhandlung
die Apollonischen Ber\"uhrungs-Aufgaben zu den folgenden:
\begin{quote}
  {\glqq}Einen Kreis zu construiren, welcher drei gegebene
  Kreise, oder eine Kugelfl\"ache, welche vier gegebene
  Kugeln unter bestimmten Winkeln schneidet.{\grqq}
\end{quote}
Zugleich giebt er die Absicht kund, ein Werk von 25 bis
30 Druckbogen herauszugeben \"uber {\glqq}das Schneiden (mit Einschluss
der Ber\"uhrung) der Kreise in der Ebene, das Schneiden
der Kugeln im Raume und das Schneiden der Kreise auf
der Kugelfl\"ache{\grqq}, in welchen jene und andere neue Probleme
ihre L\"osung finden sollten. Leider hat Steiner seinen Plan
nicht ausgef\"uhrt; unter seinen zahlreichen Schriften findet
sich nur noch ein kleineres aber gehaltvolles Werk \"uber
den Kreis\footnote{)
  Steiner, Die geometrischen Constructionen, ausgef\"uhrt mittelst
  der geraden Linie und eines festen Kreises, Berlin 1833.}),
in welchem unter anderen auch die harmonischen
und polaren Eigenschaften des Kreises elementar
abgeleitet werden.

Von Poncelet's invers liegenden und Steiner's potenzhaltenden
Punkten zu dem Princip der reciproken Radien
%-----File: 009.png---------------------------------
ist nur ein kleiner Schritt; trotzdem verdanken wir dieses
wichtige Abbildungsprincip nicht der synthetischen, sondern
der analytischen Geometrie, und in zweiter Linie der mathematischen
Physik. Pl\"ucker\footnote{)
  Pl\"ucker in Crelle's Journal f\"ur d.~r.~u.~a.\ Math., Bd.~XI.\ S.~219--225.
Die kleine Abhandlung ist von 1831 datirt.})
stellte es zuerst (1834) als {\glqq}ein
neues Uebertragungsprincip{\grqq} auf; er geht aus von Punkten,
die bez\"uglich eines Kreises einander zugeordnet sind, beweist
u.~A., dass jedem Kreise der Ebene ein Kreis oder eine Gerade
zugeordnet ist und dass zwei Gerade sich unter denselben
Winkeln schneiden wie die ihnen zugeordneten Kreise,
und giebt verschiedene Anwendungen des Princips, auch auf
das Apollonische Ber\"uhrungsproblem. Auf's Neue wurde das
Princip (1845) entdeckt von William Thomson\footnote{)
  W.~Thomson in Liouville, Journal de Math\'ematiques, T.~X.\ p.~364.}),
welcher es das Princip der elektrischen Bilder nannte; seinen heutigen
Namen erhielt es (1847) durch Liouville\footnote{)
  Liouville, Journal de Math\'ematiques, T.~XII, p.~276. }).
F\"ur Thomson sind die Anwendungen des Princips auf elektrostatische
Probleme und seine Wichtigkeit f\"ur die ganze Potentialtheorie
und f\"ur die Lehre von der W\"armeleitung nat\"urlich die
Hauptsache; nur beil\"aufig erw\"ahnt er, dass Kugeln durch
reciproke Radien allemal in Kugeln oder Ebenen \"ubergehen,
und dass die von ihnen gebildeten Winkel sich bei dieser
Transformation nicht \"andern. Liouville seinerseits hebt hervor,
dass zwei durch reciproke Radien einander zugeordnete
Fl\"achen oder Raumtheile conform auf einander abgebildet
sind, und dass die Kr\"ummungslinien der einen Fl\"ache in
diejenigen der anderen sich verwandeln; auch wendet er das
Princip u.~A.\ auf die Dupin'sche Cyclide an. Unabh\"angig
von Thomson und Liouville gelangte wenige Jahre sp\"ater
(1853) M\"obius\footnote{)
  Berichte der Kgl.\ S\"achsischen Gesellschaft der Wissenschaften,
1853, S.~14--24; Abhandlungen derselben Gesellschaft, Bd.~II,
Lpz.~1855, S.~531--595.})
zu demselben Abbildungsprincip, welchem
er den Namen {\glqq}Kreisverwandtschaft{\grqq} gab.

Die mannigfaltigen H\"ulfsmittel und fruchtbaren Methoden,
durch welche so die synthetische Geometrie der Kreise
und Kugeln allm\"alig bereichert worden ist, verdienen nun
wohl, einmal in einem neuen Zusammenhange dargestellt zu
%-----File: 010.png---------------------------------
werden. Wir gelangen zu einem solchen, innigen Zusammenhange
und zugleich zu gewissen Erweiterungen der Kugelgeometrie,
indem wir von dem bisher wenig beachteten
Kugelgeb\"usche ausgehen. Das Princip der reciproken Radien,
durch welches die meisten nachfolgenden Untersuchungen
wesentlich vereinfacht werden, tritt bei diesem Entwickelungsgange
geb\"uhrend in den Vordergrund; die Lehre von den
harmonischen Kreis-Vierecken, die Theorie der Kugelb\"undel
und Kugelb\"uschel und die Polarentheorie der Kugel und des
Kreises schliessen sich ungezwungen an, nur wird ihre Begr\"undung
eine andere; die Lehre von den linearen Kugelsystemen
aber erweitert sich von selbst zu der Geometrie des Kugelsystemes
von vier Dimensionen. Indem wir sodann den
Ber\"uhrungsproblemen uns zuwenden, treten uns alsbald
einerseits die Aehnlichkeitspunkte von Kugeln und Kreisen,
anderseits gewisse quadratische Kugel- und Kreissysteme
entgegen. Letztere, zu welchen auch die Dupin'schen Kugelschaaren
geh\"oren, werden in den sp\"ateren Abschnitten eingehend
untersucht und auf die vorhin erw\"ahnten und andere
bisher ungel\"oste Probleme Jacob Steiner's angewendet. Durch
Einf\"uhrung von Kugelcoordinaten wird schliesslich zu der
projectiven Beziehung von Kugelsystemen und zu den Kugelcomplexen,
insbesondere den quadratischen, ein leichter Zugang
gewonnen.

Den r\"aumlichen Mannigfaltigkeiten von vier und mehr
Dimensionen wird bekanntlich seit 1868 auf Anregung von
Riemann, Helmholtz und Pl\"ucker viel Beachtung geschenkt.
Deshalb m\"oge hier noch hervorgehoben werden, dass auch
dieses B\"uchlein es mit einer vierfach unendlichen Mannigfaltigkeit
zu thun hat, und zwar mit der einfachsten und
der Anschauung zug\"ang\-lich\-sten, die es giebt. Alle Kugeln
des Raumes n\"amlich bilden eine \so{lineare} Mannigfaltigkeit
von vier Dimensionen, w\"ahrend z.~B.\ die Gesammtheit aller
geraden Linien, womit die Pl\"ucker'sche Strahlengeometrie
sich besch\"aftigt, eine \so{quadratische} Mannigfaltigkeit von vier
Dimensionen bildet. Ein Kugelgeb\"usch ist demgem\"ass sehr
leicht, ein linearer Strahlencomplex dagegen nicht ohne viele
M\"uhe einem Anf\"anger verst\"andlich zu machen, und Aehnliches
gilt von dem Kugelb\"uschel und der Regelschaar. Die Kugelgeometrie
besitzt an dem Princip der reciproken Radien eine
%-----File: 011.png---------------------------------
wichtige Methode, die in der Strahlengeometrie ihres Gleichen
nicht hat; der analytischen Behandlung ist sie sehr leicht zug\"anglich,
und zudem umfasst sie die Geometrie der Punkte
und der Ebenen, weil diese als Grenzf\"alle der Kugel aufzufassen
sind. M\"oge deshalb die Kugelgeometrie ebenso wie
die Strahlengeometrie sich mehr und mehr Freunde und
F\"orderer gewinnen.
\bigskip

\hspace{5em}\so{Strassburg i.~E.}, den 20. December 1878.

\hfill\textbf{Der Verfasser.}
%-----File: 012.png---------------------------------
\newpage
\abschnitt{Inhalts-Verzeichniss.}


\begin{tabular}{l@{ }rl@{}r}
&&& \parbox{.05\textwidth}{Seite} \\
\S &1. &Potenz von Punktenpaaren, Kreisen und Kugeln~\dotfill &\pageref{p1} \\
\S &2. &Das Kugelgeb\"usch~\dotfill &\pageref{p2} \\
\S &3. &Das Princip der reciproken Radien~\dotfill &\pageref{p3}\\
\S &4. &Harmonische Kreisvierecke; harmonische Punkte, Strahlen\\
&& und Ebenen~\dotfill & \pageref{p4} \\
\S &5.& Kugelb\"undel und Kugelb\"uschel. Orthogonale Kreise~\dotfill & \pageref{p5} \\
\S &6.& Kreisb\"undel und Kreisb\"uschel~\dotfill & \pageref{p6} \\
\S &7.& Das sph\"arische und das cyklische Polarsystem~\dotfill & \pageref{p7} \\
\S &8.& Kugeln und Kreise mit reellem Centrum und rein\\
&& imagin\"arem Halbmesser~\dotfill & \pageref{p8} \\
\S &9.& Lineare Kugelsysteme~\dotfill & \pageref{p9} \\
\S &10.& Reciproke und collineare Gebilde~\dotfill & \pageref{p10} \\
\S &11.& Collineare und reciproke Gebilde in Bezug auf ein\\
&& Kugelgeb\"usch~\dotfill & \pageref{p11} \\
\S &12.& Harmonische Kugeln und Kreise~\dotfill & \pageref{p12} \\
\S &13.& Kugeln, die sich ber\"uhren. Aehnlichkeitspunkte von Kugeln~\dotfill & \pageref{p13} \\
\S &14.& Ber\"uhrung und Schnitt von Kreisen auf einer Kugelfl\"ache~\dotfill & \pageref{p14} \\
\S &15.& Die Dupin'sche Cyclide~\dotfill & \pageref{p15} \\
\S &16.& Lineare Kugelsysteme, die zu einander normal sind~\dotfill & \pageref{p16} \\
\S &17.& Kugeln, die sich unter gegebenen Winkeln schneiden~\dotfill & \pageref{p17} \\
\S &18.& Kreise auf einer Kugel, die sich unter gegebenen Winkeln\\
&& schneiden~\dotfill & \pageref{p18} \\
\\
&&\parbox{0.82\textwidth}{\centering \so{Einleitung in die analytische Geometrie\\ der Kugelsysteme.}}\\
\\
\S &19.& Kugelcoordinaten. Complexe, Congruenzen und Schaaren\\
&& von Kugeln~\dotfill & \pageref{p19} \\
\S &20.& Projective Verwandtschaft linearer Kugelsysteme~\dotfill & \pageref{p20} \\
\S & 21. &Quadratische Complexe, Congruenzen und Schaaren von\\
&& Kugeln~\dotfill & \pageref{p21}
\end{tabular}

%-----File: 013.png---------------------------------
\addtolength{\parskip}{1ex}
\newpage
\abschnitt{\S.~1.\\[\parskip]
Potenz von Punktenpaaren, Kreisen und Kugeln.}\label{p1}


\hspace{\parindent}%
1. Unter der {\glqq}Potenz{\grqq} eines Punktenpaares $P$, $P'$ in
einem Punkte $A$, welcher auf der Geraden $P$, $P'$ liegt, verstehen
wir das Produkt der beiden Strecken $AP$ und $AP'$,
welche $A$ mit den Punkten $P$ und $P'$ begrenzt; und zwar
fassen wir diese Potenz auf als eine positive oder negative
Gr\"osse, je nachdem $P$ und $P'$ auf derselben Seite von $A$
liegen oder nicht. Ist $d$ der Abstand des Punktes $A$ von
dem Mittelpunkte der Strecke $PP'$ und $r$ die halbe L\"ange
dieser Strecke, so erhalten wir f\"ur die Potenz die Gleichung:
\[
AP \centerdot AP' = (d-r) \centerdot (d + r) \quad \text{oder} \quad AP \centerdot AP' = d^2 - r^2.
\]
Das Punktenpaar hat demnach gleiche Potenz in je zwei
Punkten der Geraden, die von seinem Mittelpunkte gleich
weit abstehen. Die Potenz im Punkte $A$ ist Null, wenn $A$
mit $P$ oder $P'$ zusammenf\"allt; sie wird gleich dem Quadrate
des Abstandes $d$, wenn $P$ und $P'$ zusammenfallen.

2. Unter der {\glqq}Potenz einer Kugel oder eines Kreises
im Punkte $A${\grqq} verstehen wir die Potenz eines mit $A$ in einer
Geraden liegenden Punktenpaares der Kugelfl\"ache resp.\ der
Kreislinie. Zwei verschiedene solche Punktenpaare haben
gleiche Potenz im Punkte $A$, wie aus der Lehre von den
Kreissecanten bekannt ist. Nimmt man das Punktenpaar
$P$, $P'$ auf dem durch $A$ gehenden Durchmesser an, und bezeichnet
mit $d$ den Abstand des Punktes $A$ vom Centrum
und mit $r$ den Radius der Kugel oder des Kreises, so wird
die Potenz in $A$ dargestellt durch:
\[
AP \centerdot AP' = d^2 -r^2.
\]
Eine Kugel hat demnach gleiche Potenz in allen Punkten,
welche von ihrem Centrum gleich weit abstehen.

%-----File: 014.png---------------------------------

3. Alle Kreise, in welchen eine Kugel von den durch
$A$ gehenden Ebenen geschnitten wird, haben im Punkte $A$
gleiche Potenz, n\"amlich dieselbe wie die Kugel. Diese Potenz
ist gleich dem Quadrate einer von $A$ bis an die Kugelfl\"ache
gezogenen Tangente, wenn $A$ ausserhalb der Kugel
liegt; sie ist Null, wenn $A$ auf, und negativ, wenn $A$ innerhalb
der Kugel liegt (1.). Im ersten dieser drei F\"alle wird
die Kugelfl\"ache rechtwinklig geschnitten von derjenigen Kugelfl\"ache,
welche den Punkt $A$ zum Mittelpunkt und die
Quadratwurzel aus der Potenz zum Radius hat.

4. Wenn zwei Kugelfl\"achen sich rechtwinklig schneiden,
so ist die Potenz der einen im Centrum der anderen gleich
dem Quadrate des Radius dieser anderen Kugelfl\"ache; denn
die beiden Radien, welche nach irgend einem ihrer Schnittpunkte
gehen, stehen auf einander senkrecht, und jeder von
ihnen ber\"uhrt deshalb die zu dem anderen geh\"orige Kugel.
Dieser Satz und seine Umkehrung (3.) gilt auch von zwei
Kreisen, die in einer Ebene liegen und sich rechtwinklig
schneiden.

5. Jeder Punkt, in welchem zwei oder mehrere Kugeln
gleiche Potenz haben, wird ein {\glqq}Potenzpunkt{\grqq} der Kugeln
genannt; derselbe ist auch f\"ur die Kreise und Punktenpaare,
in welchen die Kugeln etwa sich schneiden, ein Punkt gleicher
Potenz oder {\glqq}Potenzpunkt{\grqq}. Die Mittelpunkte aller
Kugeln, welche zwei oder mehrere gegebene Kugeln rechtwinklig
schneiden, sind Potenzpunkte der letzteren (4.). Wenn
zwei Kugeln sich schneiden oder ber\"uhren, so haben sie
jeden Punkt der Ebene, in welcher ihr Schnittkreis liegt
oder welche sie in ihrem gemeinschaftlichen Punkte ber\"uhrt,
zum Potenzpunkt; in jedem ausserhalb dieser Ebene liegenden
Punkte dagegen haben sie ungleiche Potenz, wie sofort
einleuchtet, wenn man den Punkt mit einem gemeinschaftlichen
Punkte der Kugeln durch eine Secante verbindet.

6. Der Ort aller Potenzpunkte von drei Kugeln, von
denen zwei die dritte schneiden, ist (5.) die Gerade, welche
die Ebenen der beiden Schnittkreise mit einander gemein
haben. In jedem Punkte dieser Ebenen, welcher ausser\-halb
ihrer Schnittlinie liegt, haben die ersten beiden Kugeln
ungleiche Potenz; denn nur die eine von ihnen hat in einem
solchen Punkte mit der dritten Kugel gleiche Potenz. Zwei
%-----File: 015.png-------------------------------
Kugeln haben demnach unendlich viele Potenzpunkte; mit
dem Orte dieser Punkte hat jede Schnittebene der einen oder
der anderen Kugel eine Gerade gemein; jeder Punkt, welcher
mit zwei Potenzpunkten der Kugeln in einer Geraden liegt,
ist folglich selbst ein Potenzpunkt derselben. Somit ist der
Ort aller Potenzpunkte von zwei Kugeln eine Ebene, welche
die {\glqq}Potenz-Ebene{\grqq} der beiden Kugeln genannt wird.

7. Die Potenzebene, d.~h.\ der Ort aller Potenzpunkte
von zwei Kugeln, ist zu der Centrallinie dieser Kugeln normal.
Dieses folgt aus Gr\"unden der Symmetrie; auch liegt
in der Potenzebene die Schnittlinie von je zwei Kugeln, die
mit den gegebenen concentrisch sind und durch irgend einen
Potenzpunkt $P$ derselben gehen, weil (2.) die gegebenen
Kugeln in allen Punkten jener Schnittlinie die gleiche Potenz
haben wie in $P$. Die Potenzebene geht durch jeden
gemeinschaftlichen Punkt der beiden Kugeln, weil in demselben
die Potenz der Kugeln gleich, n\"amlich Null ist; sie
enth\"alt die Mittelpunkte aller Kugeln, welche die beiden gegebenen
rechtwinklig schneiden (5.), und insbesondere auch
die Halbirungspunkte aller gemeinschaftlichen Tangenten der
gegebenen Kugeln. Bringt man die beiden Kugeln zum
Durchschnitt mit einer beliebigen dritten, und sodann die
Ebenen der beiden Schnittkreise mit einander, so erh\"alt man
eine Gerade der Potenzebene (6.). Die Potenzebene von zwei
concentrischen Kugeln r\"uckt in's Unendliche.

8. Der Ort aller Potenzpunkte von drei beliebigen Kugeln
ist eine Gerade, welche wir die {\glqq}Potenz-Axe{\grqq} der drei
Kugeln nennen. In dieser Geraden schneiden sich die beiden
Potenzebenen, welche die eine der drei Kugeln mit den
beiden \"ubrigen bestimmt; sie liegt aber auch in der Potenzebene
der beiden letzteren, weil sie Potenzpunkte derselben
enth\"alt. Auf den Ausnahmefall, in welchem die drei Kugeln
paarweise dieselbe Potenzebene haben, kommen wir sp\"ater
zur\"uck. Die Potenzaxe der drei Kugeln steht auf der Centralebene
derselben normal (7.); sie r\"uckt in's Unendliche,
wenn die Mittelpunkte der Kugeln in einer Geraden liegen.
Sie enth\"alt die Mittelpunkte aller Kugeln, welche die drei
gegebenen rechtwinklig schneiden, sowie jeden gemeinschaftlichen
Punkt der drei Kugeln (7.). Bringt man die drei
Kugeln zum Durchschnitt mit einer beliebigen vierten und
%-----File: 016.png---------------------------------
sodann die Ebenen der drei Schnittkreise mit einander, so
erh\"alt man einen Punkt der Potenzaxe.

9. Vier beliebige Kugeln haben einen Potenzpunkt. In
demselben schneiden sich die Potenzebenen, welche jede der
Kugeln mit den drei \"ubrigen bestimmt, und folglich auch
die vier Potenzaxen, welche die vier Kugeln zu dreien bestimmen.
Den Ausnahmefall, in welchem die Kugeln zu
dreien eine und dieselbe Potenzaxe haben, schliessen wir
vorl\"aufig aus. Haben die vier Kugeln in ihrem Potenzpunkte
positive Potenz, so werden sie von einer Kugel, die den Potenzpunkt
zum Mittelpunkt und die Quadratwurzel aus der
Potenz zum Radius hat, rechtwinklig geschnitten. Der Potenzpunkt
r\"uckt in's Unendliche, wenn die Mittelpunkte der
vier Kugeln in einer Ebene liegen.

10. Als Grenzf\"alle der Kugel sind die Punktkugel und
die Ebene, und als Grenzf\"alle des Kreises sind der Punktkreis
und die Gerade aufzufassen. Wenn der Radius einer
durch den Punkt $P$ gehenden Kugel unbegrenzt abnimmt,
so reducirt sich die Kugel auf den Punkt $P$ und wird eine
Punktkugel; nimmt dagegen der Radius unbegrenzt zu, indem
der Mittelpunkt sich nach irgend einer Richtung entfernt,
so geht die Kugelfl\"ache \"uber in die durch $P$ gehende
und zu jener Richtung normale Ebene. Die Potenz einer
Punktkugel im Punkte $A$ ist gleich dem Quadrat ihres Abstandes
von $A$ (1.). Die Potenz einer Ebene in einem nicht
auf ihr liegenden Punkte $A$ ist unendlich; in einem auf ihr
liegenden Punkte $P$ ist sie unbestimmt, n\"amlich $0 \centerdot \infty$. Die
Potenzebene einer Punktkugel und einer gew\"ohnlichen Kugel
enth\"alt die Mittelpunkte aller Kugelfl\"achen, welche durch
die Punktkugel gehen und die andere Kugel rechtwinklig
schneiden; sie halbirt alle Tangenten, welche von der Punktkugel
an die andere Kugel gezogen werden k\"onnen. Zwei
Punktkugeln liegen zu ihrer Potenzebene symmetrisch; die
sechs Potenzebenen von vier Punktkugeln schneiden sich
in dem Centrum der Kugel, auf welcher die vier Punktkugeln
liegen, und welche hiernach leicht zu construiren ist.
Die Potenzebene einer gew\"ohnlichen Kugel und einer Ebene
f\"allt mit der letzteren zusammen.

\begin{center}
\makebox[15em]{\hrulefill}
\end{center}
%-----File: 017.png---------------------------------

\abschnitt{\S.~2.\\[\parskip]
Das Kugelgeb\"usch.}\label{p2}


\hspace{\parindent}%
11. Mit dem Namen {\glqq}Kugelgeb\"usch{\grqq} bezeichnen wir
die Gesammtheit aller Kugeln, die in einem gegebenen Punkte
$C$ eine bestimmte Potenz $p$ haben; $C$ heisst der Potenzpunkt
oder das {\glqq}Centrum{\grqq} und $p$ die {\glqq}Potenz des Geb\"usches{\grqq}. Die
Punktenpaare, in welchen je drei, und die Kreise, in welchen
je zwei Kugeln des Geb\"usches sich schneiden, rechnen wir
ebenfalls zu dem Geb\"usche; sie alle haben im Centrum $C$ die
Potenz $p$ und liegen auf den durch $C$ gehenden Geraden und
Ebenen. Das Geb\"usch enth\"alt alle Kugeln, die durch irgend
einen seiner Kreise oder durch ein beliebiges von seinen
Punktenpaaren gehen, insbesondere auch die durch $C$ gehenden
Ebenen dieser Kreise und Punktenpaare; es enth\"alt ferner alle
Kreise und Punktenpaare, in welchen seine Kugeln von den
durch $C$ gehenden Ebenen und Geraden geschnitten werden;
durch eine Drehung um das Centrum $C$ wird es nicht ver\"andert.

12. Um ein Kugelgeb\"usch zu bestimmen, kann man
sein Centrum $C$ und entweder seine Potenz $p$, oder eine
seiner Kugeln oder Kreislinien, oder eines seiner Punktenpaare
willk\"urlich annehmen; bei jeder der letzteren Annahmen
ergiebt sich die Potenz in $C$ sofort. Vier beliebig gegebene
Kugeln bestimmen ein durch sie gehendes Kugelgeb\"usch,
wenn sie nicht in mehreren Punkten gleiche Potenz
haben; n\"amlich ihr Potenzpunkt (9.) ist das Centrum des
Geb\"usches, und ihre Potenz in diesem Punkte ist zugleich
diejenige des Geb\"usches. Ebenso bestimmen zwei beliebige
Kreise, die nicht auf einer und derselben Kugel liegen, ein
Kugelgeb\"usch; dasselbe geht durch zwei Paar Kugeln, die
sich in den beiden Kreisen schneiden, und ist durch sie bestimmt.
Alle Ebenen, welche zwei nicht auf einer Kugel
liegende Kreise in vier Kreispunkten schneiden, gehen durch
einen Punkt, n\"amlich durch das Centrum des durch die
beiden Kreise bestimmten Kugelgeb\"usches; auch die Ebenen
der beiden Kreise gehen durch diesen Punkt.

13. Ist die Potenz $p$ eines Kugelgeb\"usches negativ, so
liegt sein Centrum $C$ innerhalb aller seiner Kugeln und
Kreise und zwischen allen seinen Punktenpaaren, und jede
Kugel des Geb\"usches schneidet alle \"ubrigen. Ist dagegen $p$
positiv, so liegt das Centrum $C$ ausserhalb aller Kugeln und
%-----File: 018.png---------------------------------
Kreise des Geb\"usches, und alle diese Kreise und Kugeln
werden rechtwinklig von derjenigen Kugel geschnitten, welche
mit dem Radius $\sqrt{p}$ um den Mittelpunkt $C$ beschrieben
werden kann (3.). Diese Kugel heisst deshalb die {\glqq}Orthogonalkugel{\grqq}
des Geb\"usches; sie ist der Ort aller Punktkugeln
desselben. Alle Kugeln und Kreise, welche die Orthogonalkugel
rechtwinklig schneiden, geh\"oren zu dem Geb\"usch (4.),
und dieses ist durch seine Orthogonalkugel v\"ollig bestimmt.
Wenn die Orthogonalkugel in eine Ebene \"ubergeht, so enth\"alt
das Geb\"usch alle Kugeln, deren Mittelpunkte in dieser
Ebene liegen; das Centrum $C$ dieses besonderen Geb\"usches
liegt unendlich fern, seine Potenz ist unendlich gross, und
jeder Kreis und jedes Punktenpaar desselben liegt symmetrisch
bez\"uglich der Orthogonalebene. Wir nennen dieses
besondere Geb\"usch ein {\glqq}symmetrisches{\grqq}. --- Einen Uebergangsfall
des Kugelgeb\"usches erhalten wir, wenn die Potenz $p$ Null ist;
dieses specielle Geb\"usch besteht aus allen Kugeln und Kreisen,
welche durch sein Centrum $C$ gehen, seine Orthogonalkugel
reducirt sich auf den Punkt $C$, und $C$ bildet mit jedem Punkte
des Raumes ein Punktenpaar des Geb\"usches. Wir schliessen
diesen Uebergangsfall vorl\"aufig von unserer Untersuchung aus.

14. Im Kugelgeb\"usch nennen wir zwei Punkte $P$, $P'$
{\glqq}einander zugeordnet{\grqq}, wenn sie ein Punktenpaar des Geb\"usches
bilden. Durch einen Punkt $P$ ist im Geb\"usche der
ihm zugeordnete Punkt $P'$ eindeutig bestimmt; denn die
Punkte $P$ und $P'$ liegen mit dem Centrum $C$ in einer Geraden
und das Produkt ihrer Abst\"ande $CP$ und $CP'$ vom
Centrum ist gleich der Potenz $p$ des Geb\"usches. Wenn $P$
nach irgend einer Richtung in's Unendliche r\"uckt, so f\"allt $P'$
mit $C$ zusammen. Alle durch einen Punkt $P$ gehenden Kugeln
und Kreise des Geb\"usches haben auch den zugeordneten
Punkt $P'$ mit einander gemein, weil sie im Centrum $C$ die
Potenz $p = CP \centerdot CP'$ haben. Aus demselben Grunde geh\"ort
jede Kugel oder Kreislinie, welche durch zwei einander zugeordnete
Punkte geht, zu dem Geb\"usch.

15. Zwei Punktenpaare des Geb\"usches k\"onnen deshalb
allemal durch einen Kreis, und drei Punktenpaare k\"onnen
durch eine Kugel verbunden werden. Durch drei beliebige
Punkte oder durch einen beliebigen Kreis geht im Allgemeinen
eine einzige Kugel des Geb\"usches; dieselbe verbindet
%-----File: 019.png---------------------------------
die drei Punkte mit den drei zugeordneten Punkten. Wenn
durch einen Kreis mehrere Kugeln des Geb\"usches gehen, so
geh\"ort er zu dem Geb\"usche und kann mit jedem Punktenpaare
desselben durch eine Kugel verbunden werden (11.).

16. Von den Punktenpaaren eines Kugelgeb\"usches,
welche auf einem Kreise desselben oder auf einer durch sein
Centrum gehenden Geraden liegen, pflegt man zu sagen, sie
bilden eine {\glqq}involutorische Punktreihe{\grqq} oder ihre Punkte
seien {\grqq}involutorisch gepaart{\grqq}; den Kreis oder die Gerade
nennt man den {\glqq}Tr\"ager{\grqq} dieser Punktreihe. Die Geraden,
auf welchen die Punktenpaare einer solchen involutorischen
Punktreihe liegen, gehen alle durch einen Punkt, n\"amlich
durch das Centrum $C$ des Geb\"usches. Jede Kugel des Geb\"usches,
welche durch einen Punkt $P$ der Punktreihe geht,
hat mit ihr auch den zugeordneten Punkt $P'$ gemein (11.,
14.). Verbindet man irgend zwei Punktenpaare der Reihe
mit zwei beliebig angenommenen Punkten durch zwei Kugeln,
so schneiden sich diese in einem Kreise $k$ des Geb\"usches,
und auf den durch $k$ gehenden anderen Kugeln liegen auch
die \"ubrigen Punktenpaare der involutorischen Reihe. Um die
Punkte einer Kreislinie oder Geraden involutorisch zu paaren,
kann man demnach zwei Punktenpaare auf derselben willk\"urlich
annehmen; die \"ubrigen Punktenpaare und das Kugelgeb\"usch,
in welchem die involutorische Punktreihe liegt, sind
dadurch v\"ollig bestimmt und leicht construirbar.

17. Wenn zwei Kreise $k$ und $k_1$ weder einen Punkt
mit einander gemein haben, noch durch eine Kugel oder
Ebene verbunden werden k\"onnen, so schneidet jeder von
ihnen die durch den anderen gehenden Kugeln in den Punktenpaaren
einer involutorischen Punktreihe. Dieselbe liegt
in dem durch $k$ und $k_1$ bestimmten Kugelgeb\"usch (12.), und
der Satz gilt auch dann, wenn einer, aber nicht jeder der
beiden Kreise in eine Gerade ausartet; in dem Centrum der
Punktreihe schneiden sich auch die durch $k$ und $k_1$ gehenden
Ebenen. Alle Punktenpaare einer involutorischen Punktreihe
haben in deren Centrum, d.~h.\ in dem Centrum $C$ des sie
enthaltenden Kugelgeb\"usches, gleiche Potenz, auch wenn die
Punktreihe auf einer Geraden liegt; r\"uckt $C$ in's Unendliche,
so liegen die Punktenpaare symmetrisch bez\"uglich der Orthogonal-Ebene
des Geb\"usches (13.).
%-----File: 020.png---------------------------------

18. Eine involutorische Punktreihe bestimmt ein sie
enthaltendes Kugelgeb\"usch (16.); sie hat zwei {\glqq}Ordnungspunkte{\grqq},
d.~h.\ sich selbst zugeordnete Punkte, wenn die Potenz
dieses Geb\"usches positiv ist. Von der Orthogonalkugel
des Geb\"usches wird der Tr\"ager der involutorischen Punktreihe
in den beiden Ordnungspunkten rechtwinklig geschnitten
(13.); diese Ordnungspunkte sind zwei Punktkugeln des
Geb\"usches, und je zwei einander zugeordnete Punkte $P$, $P'$
der Punktreihe sind durch sie getrennt. Der Tr\"ager der involutorischen
Punktreihe ber\"uhrt alle durch einen ihrer Ordnungspunkte
$O$, $Q$ gehenden Kugeln und Ebenen des Geb\"usches
in diesem Punkte (vgl.\ 16.). Die Potenz des Geb\"usches
in seinem Centrum $C$ wird dargestellt durch:
\[
CP \centerdot CP' = CO^2 = CQ^2.
\]


\begin{center}
\makebox[15em]{\hrulefill}
\end{center}

\abschnitt{\S.~3.\\[\parskip]
Das Princip der reciproken Radien.}\label{p3}


\hspace{\parindent}%
19. Es sei $C$ das Centrum, $p$ die positive oder negative
Potenz und $A$, $A'$ ein beliebiges Punktenpaar eines Kugelgeb\"usches.
Wir bezeichnen die Strecken $CA = r$ und $CA' = r'$
mit dem Namen {\glqq}Radien der beiden einander zugeordneten
Punkte $A$ und $A'${\grqq}; sie liegen auf einer und derselben
Geraden und ihr Produkt $r \centerdot r'$ ist gleich der Potenz $p$. Der
Radius $r$ eines beliebigen Punktes $A$ ist demnach dem reciproken
Werthe des Radius $r'$ seines zugeordneten Punktes
$A'$ proportional, er ist das $p$fache dieses reciproken Werthes,
n\"amlich $r = p \centerdot \frac{1}{r'}$. Man nennt deshalb $r$ und $r'$ {\glqq}reciproke
Radien{\grqq}, $C$ ihr Centrum und $p$ ihre Potenz, und sagt von
zwei einander zugeordneten Figuren, Linien oder Fl\"achen,
von welchen die eine durch $A$ und zugleich die andere durch
den zugeordneten Punkt $A'$ beschrieben ist, sie seien {\glqq}invers{\grqq}
und {\glqq}jede von ihnen sei durch reciproke Radien in die andere
transformirt oder verwandelt{\grqq}.

20. Alle Kugeln, Kreise und Punktenpaare des Geb\"usches
werden durch die reciproken Radien in sich selbst transformirt.
Zwei beliebige dieser Punktenpaare, $A$, $A'$ und $B$, $B'$
haben im Centrum $C$ die Potenz $p$, sodass:
\[
CA \centerdot CA' = CB \centerdot CB' \quad \text{und folglich} \quad CA : CB = CB' : CA'
\]
%-----File: 021.png---------------------------------
ist. Daraus aber folgt, wenn $CA$ und $CB$ nicht auf derselben
Geraden liegen, dass die Dreiecke $CAB$ und $CB'A'$
\"ahnlich und ihre Winkel bei $A$ und $B'$ gleich sind. Ist insbesondere
$\angle CAB$ ein rechter Winkel, so gilt dasselbe vom
Winkel $CB'A'$.

21. Eine beliebige Ebene $\varepsilon$ wird durch die reciproken
Radien in eine Kugelfl\"ache verwandelt, welche im Centrum
$C$ von einer zu $\varepsilon$ parallelen Ebene ber\"uhrt wird. Denn
seien $A$ und $B$ zwei Punkte von $\varepsilon$, von welchen $A$ in der
von $C$ auf $\varepsilon$ gef\"allten Normale liege, und seien $A'$ und $B'$
die ihnen zugeordneten Punkte. Dann sind die Dreiecke
$CAB$ und $CB'A'$ \"ahnlich und ihre Winkel bei $A$ und $B'$
Rechte (20.), und der Punkt $B'$, welcher einem ganz beliebigen
Punkte $B$ der Ebene $\varepsilon$ entspricht, liegt folglich auf
der Kugelfl\"ache, von welcher die zu $\varepsilon$ normale Strecke $CA'$
ein Durchmesser ist. Diese Kugelfl\"ache, in welche $\varepsilon$ transformirt
wird, hat in $C$ eine zum Durchmesser $CA'$ normale
und folglich zu $\varepsilon$ parallele Ber\"uhrungsebene. --- Jede durch
$C$ gehende Kugel wird durch die reciproken Radien in eine
Ebene transformirt; dieselbe ist der Ber\"uhrungsebene des
Punktes $C$ parallel und geht durch einen beliebigen Punkt,
dessen zugeordneter auf der Kugel liegt.

22. Zwei beliebige Ebenen schneiden sich unter denselben
Winkeln, wie die ihnen zugeordneten Kugelfl\"achen,
weil sie den Ber\"uhrungsebenen der letzteren im Punkte $C$
parallel sind (21.). Zwei beliebige Fl\"achen oder Linien
schneiden sich folglich in jedem ihrer gemeinschaftlichen
Punkte unter denselben Winkeln, wie die ihnen zugeordneten
Fl\"achen oder Linien in dem zugeordneten Punkte. Zwei unendlich
kleine Tetra\"eder, deren Eckpunkte einander zugeordnet
sind, haben demnach gleiche Fl\"achenwinkel und schon
deshalb auch gleiche Kantenwinkel; sie sind, wie einige
Ueberlegung lehrt, \"ahnlich, wenn die Potenz der reciproken
Radien negativ, und symmetrisch \"ahnlich, wenn sie positiv
ist; ihre homologen Fl\"achen sind allemal \"ahnlich. Zwei
einander zugeordnete Fl\"achen oder Raumtheile werden also
durch die reciproken Radien {\glqq}conform{\grqq}, d.~h.\ in den kleinsten
Theilen \"ahnlich, auf einander abgebildet.

23. Um hiernach eine Kugelfl\"ache $\varkappa$ auf eine beliebige
Ebene $\varepsilon$ conform abzubilden, w\"ahle man zum Centrum $C$
%-----File: 022.png---------------------------------
der reciproken Radien einen der beiden Punkte von $\varkappa$, deren
Ber\"uhrungsebenen zu $\varepsilon$ parallel sind, und setze die Potenz
gleich dem Produkte der beiden Abschnitte $CA$ und $CA'$,
welche $\varkappa$ und $\varepsilon$ auf irgend einer durch $C$ gehenden Geraden
bilden. Dann wird $\varkappa$ in $\varepsilon$ transformirt (21.). Projicirt man
also eine Kugelfl\"ache $\varkappa$ (stereographisch) aus einem ihrer
Punkte $C$ auf eine Ebene $\varepsilon$, die zu der Ber\"uhrungsebene von
$C$ parallel ist, so wird die Fl\"ache $\varkappa$ conform auf die Ebene
$\varepsilon$ abgebildet. Von dieser {\glqq}stereographischen{\grqq} Projection
der Kugel wird bei der Herstellung von Landkarten Gebrauch
gemacht. Man erreicht dadurch, dass wenigstens
die Winkel auf der Karte dieselbe Gr\"osse haben, wie die
ihnen entsprechenden auf der Erdkugel. Die L\"angen der
verschiedenen Linien unserer Erdoberfl\"ache m\"ussen auf den
Landkarten allemal in ver\"anderlichem Massstabe dargestellt
werden, weil eine Kugelfl\"ache sich nicht ohne Verzerrungen
auf einer Ebene abwickeln l\"asst.

24. Durch verschiedene reciproke Radien von gegebenem
Centrum $C$ wird eine gegebene Figur in \"ahnliche und \"ahnlich
liegende Figuren verwandelt, von welchen $C$ der Aehnlichkeitspunkt
ist. Zwei beliebigen Punkten $A'$, $B'$ der gegebenen
Figur m\"ogen n\"amlich die resp.\ Punkte $A$, $B$ oder
$A_1$, $B_1$ zugeordnet sein, jenachdem die Potenz der reciproken
Radien gleich $p$ oder $p_1$ ist. Dann ist:
\[
CA' \centerdot CA = CB' \centerdot CB = p \quad\text{und}\quad
CA' \centerdot CA_1 = CB' \centerdot CB_1 = p_1,
\]
und folglich:
\[
CA : CA_1 = CB : CB_1 = p : p_1 \quad\text{und}\quad
\triangle CAB \sim \triangle CA_1B_1.
\]
Die Geraden $\overline{AB}$ und $\overline{A_1B_1}$ sind also parallel, und $A$ und
$A_1$, sowie $B$ und $B_1$ sind homologe Punkte von zwei \"ahnlichen
und \"ahnlich liegenden r\"aumlichen Systemen; und zwar
ist $C$ ein \"ausserer oder innerer Aehnlichkeitspunkt, jenachdem
$p : p_1$ positiv oder negativ ist. Die r\"aumlichen Systeme
sind symmetrisch und $C$ ist ihr Symmetrie-Centrum, wenn
$p = -p_1$ ist.

25. Durch reciproke Radien wird eine nicht durch das
Centrum $C$ gehende Kugel $\varkappa$ in eine Kugel $\varkappa_1$ transformirt;
$C$ ist ein Aehnlichkeitspunkt von $\varkappa$ und $\varkappa_1$. Ist n\"amlich $p$
die Potenz der reciproken Radien und $p_1$ die Potenz der
%-----File: 023.png---------------------------------
Kugel $\varkappa$ %sic, not $\varkappa_1$
im Punkte $C$, so wird $\varkappa$ durch die verschiedenen
reciproken Radien vom Centrum $C$ und den Potenzen $p$ und
$p_1$ in zwei \"ahnliche und in Bezug auf $C$ \"ahnlich liegende
Fl\"achen verwandelt (24.). Die eine dieser Fl\"achen ist aber
die Kugel $\varkappa$ selbst, und folglich ist auch die andere eine
Kugel $\varkappa_1$. --- Der fr\"uhere Satz (21.), dass jeder Ebene eine
durch $C$ gehende Kugel zugeordnet ist, kann als ein specieller
Fall des eben bewiesenen betrachtet werden.

26. Einem Kreise ist durch die reciproken Radien allemal
ein Kreis zugeordnet; in dem letzteren schneiden sich
je zwei Kugeln, deren zugeordnete durch den ersteren gehen.
Die beiden Kreise liegen auf derjenigen Kugelfl\"ache des zu
den Radien geh\"origen Geb\"usches, welche durch den einen
von ihnen gelegt werden kann (15.). Geht der eine Kreis
durch das Centrum $C$, so artet der andere in eine Gerade
aus (21.). --- Durch die stereographische Projection (23.)
gehen alle Kreise der Erdkugel, insbesondere alle Meridiane
und Parallelkreise, \"uber in Kreise der Bildebene, und zwar
die Meridiane in Kreise, welche sich in den Projectionen des
Nord- und des S\"udpoles schneiden, und die Parallelkreise in
solche, welche die ersteren rechtwinklig, nicht aber einander
schneiden. Nur die durch das Centrum $C$ gehenden Kugelkreise
werden in der Bildebene durch gerade Linien dargestellt.
Wird $C$ in den Nord- oder S\"udpol gelegt, so werden
die Parallelkreise und die Meridiane dargestellt durch concentrische
Kreise und deren Durchmesser.

27. Wenn eine Kugel und ein Kegel sich in einem
Kreise schneiden, so haben sie noch einen zweiten Kreis mit
einander gemein. In diesen zweiten Kreis n\"amlich verwandelt
sich der erstere durch reciproke Radien, deren Centrum
der Mittelpunkt $C$ des Kegels und deren Potenz gleich derjenigen
der Kugel im Punkte $C$ ist (26.). Die beiden Kreise
ber\"uhren alle Kugelkreise, welche in den Ber\"uhrungsebenen
des Kegels liegen. --- Zwei beliebige Kreise $k$, $k'$ einer Kugel
k\"onnen allemal durch eine und im Allgemeinen noch durch
eine zweite Kegelfl\"ache verbunden werden. Sind n\"amlich $A$
und $A'$ zwei Punkte von $k$ resp.\ $k'$, deren Tangenten sich
schneiden, und $B$ und $B'$ zwei mit ihnen in einer Ebene
liegende Punkte von $k$ resp.\ $k'$; dann ist der Schnittpunkt $C$
der Geraden $\overline{AA'}$ und $\overline{BB'}$ Mittelpunkt eines durch $k$ und $k'$
%-----File: 024.png---------------------------------
gehenden Kegels. Denn der von $C$ aus durch $k$ gelegte
Kegel schneidet die Kugel noch in einem von $k$ verschiedenen
Kreise, welcher mit $k'$ die Punkte $A'$ und $B'$ sowie die Tangente
in $A'$ gemein hat und folglich mit $k'$ zusammenf\"allt.
Da eine beliebige Tangente von $k$ zwei Tangenten von $k'$
schneidet, so erh\"alt man zwei verschiedene durch $k$ und $k'$
gehende Kegel, ausgenommen, wenn die beiden Kreise sich
ber\"uhren oder einer derselben ein Punktkreis ist. --- Aus dem
Vorhergehenden folgt: Wenn eine Ebene sich so bewegt,
dass sie zwei auf einer Kugel liegende Kreise fortw\"ahrend
ber\"uhrt, so umh\"ullt sie eine die beiden Kreise verbindende
Kegelfl\"ache.

28. Ein beliebiges Kugelgeb\"usch $\varGamma$ verwandelt sich
durch reciproke Radien allemal in ein Kugelgeb\"usch; die
Centra $M$ und $M'$ der beiden Geb\"usche liegen mit dem Centrum
$C$ der reciproken Radien in einer Geraden. N\"amlich
die Kugeln, Kreise und Punktenpaare von $\varGamma$ werden durch
die reciproken Radien transformirt in andere Kugeln, Kreise
und Punktenpaare, deren Gesammtheit wir mit $\varGamma'$ bezeichnen
wollen. Die Ebenen aller Kreise und die Verbindungslinien
aller Punktenpaare von $\varGamma'$ gehen durch einen Punkt $M'$;
denn sie sind den durch $C$ gehenden Kugeln und Kreisen
des Geb\"usches $\varGamma$ zugeordnet, und diese haben ausser $C$ noch
denjenigen Punkt $C_1$ mit einander gemein, welcher in $\varGamma$ dem
Punkte $C$ zugeordnet ist (14.); die Punkte $C_1$ und $M'$ aber
sind durch die reciproken Radien einander zugeordnet und
liegen mit $C$ und $M$ in einer Geraden. Endlich aber haben
die Punktenpaare, Kreise und Kugeln von $\varGamma'$ alle im Punkte
$M'$ gleiche Potenz und bilden folglich ein Kugelgeb\"usch;
denn zwei beliebige von diesen Punktenpaaren liegen allemal
auf einem Kreise und drei von ihnen liegen auf einer Kugel
von $\varGamma'$, weil die ihnen zugeordneten Punktenpaare des Geb\"usches
$\varGamma$ durch einen Kreis resp.\ eine Kugel von $\varGamma$ verbunden
werden k\"onnen (15.). Damit ist bewiesen, dass $\varGamma'$
ebenso wie $\varGamma$ ein Kugelgeb\"usch ist.

29. Wenn das Kugelgeb\"usch $\varGamma$ eine Orthogonalkugel
hat, so wird diese durch die reciproken Radien in die Orthogonalkugel
des zugeordneten Geb\"usches $\varGamma'$ verwandelt;
denn wenn zwei Kugeln sich rechtwinklig schneiden, so gilt
dasselbe von den beiden ihnen zugeordneten Kugeln (22.).
%-----File: 025.png---------------------------------
Liegt das Centrum $C$ der reciproken Radien auf der Orthogonalkugel
von $\varGamma$, so ist $\varGamma'$ ein symmetrisches Geb\"usch,
dessen Kugeln, Kreise und Punktenpaare eine gemeinschaftliche
Symmetrie-Ebene haben, n\"amlich die Orthogonalebene
von $\varGamma'$ (13.). Das specielle Geb\"usch, dessen Kugeln und
Kreise alle durch einen gegebenen Punkt $M$ gehen, verwandelt
sich durch reciproke Radien in ein \"ahnliches specielles
Geb\"usch; nur wenn das Centrum der reciproken Radien mit
$M$ zusammenf\"allt, transformirt es sich in die Gesammtheit
aller Ebenen und Geraden des Raumes, welche also auch
als ein sehr specielles Kugelgeb\"usch zu betrachten ist.

30. Eine involutorische Punktreihe $k$ verwandelt sich
durch reciproke Radien in eine involutorische Punktreihe $k'$,
und zwar werden die Ordnungspunkte von $k$ in diejenigen
von $k'$ transformirt; denn $k$ und $k'$ sind einander zugeordnete
Gebilde von zwei durch sie bestimmten Kugelgeb\"uschen,
welche durch die reciproken Radien in einander transformirt
werden. Nimmt man das Centrum $C$ der Radien irgendwo
auf der Kugel an, welche den Tr\"ager der involutorischen
Punktreihe $k$ in deren Ordnungspunkten $O$ und $Q$ rechtwinklig
schneidet, so verwandelt sich $k$ in eine symmetrische
Punktreihe $k'$, deren Punktenpaare zu einem Durchmesser
des Kreises $k'$ symmetrisch liegen (vgl.\ 17., 29.). F\"allt $C$
mit $O$ oder $Q$ zusammen, so wird $k'$ eine \so{gerade} symmetrische
Punktreihe, von welcher ein Ordnungspunkt unendlich
fern liegt und der andere die Strecken zwischen je zwei
einander zugeordneten Punkten halbirt.

\enlargethispage{-\baselineskip}
\begin{center}
\makebox[15em]{\hrulefill}\bigskip
\end{center}

\abschnitt{\S.~4. \\[\parskip]
Harmonische Kreisvierecke; harmonische Punkte, Strahlen
und Ebenen.}\label{p4}


\hspace{\parindent}%
31. Von je zwei einander zugeordneten Punkten $P$, $R$
einer involutorischen Punktreihe wollen wir sagen, sie seien
durch die beiden Ordnungspunkte $O$, $Q$ der Punktreihe {\glqq}harmonisch
getrennt{\grqq} und bilden mit denselben eine harmonische
Punktreihe $OPQR$ oder {\glqq}vier harmonische Punkte{\grqq}.
Ist der Tr\"ager der Punktreihe ein Kreis, so nennen wir
ausserdem das Viereck $OPQR$ ein {\glqq}harmonisches Kreisviereck{\grqq}.
Demnach sind je zwei Punkte $P$, $R$ eines Kreises,
%-----File: 026.png---------------------------------
welche mit dem Schnittpunkte $C$ von zwei Tangenten desselben
in einer Geraden liegen, durch die Ber\"uhrungspunkte
$O$, $Q$ dieser Tangenten harmonisch getrennt und bilden mit
ihnen ein harmonisches Kreisviereck $OPQR$. Durch zwei
beliebige Punkte eines Kreises sind insbesondere die Halbirungspunkte
der beiden von ihnen begrenzten Kreisb\"ogen
harmonisch getrennt; diese beiden Halbirungspunkte liegen
auf einem Durchmesser des Kreises, und je zwei Punkte des
Kreises, durch welche sie harmonisch getrennt sind, liegen
symmetrisch zu dem Durchmesser. Jedes Quadrat ist ein
harmonisches Kreisviereck.

32. Die involutorische Punktreihe, von welcher $O$, $Q$
die beiden Ordnungspunkte und $P$, $R$ zwei einander zugeordnete
Punkte sind, liegt in einem durch sie bestimmten
Kugelgeb\"usch (18.). Ist $C$ das Centrum dieses Geb\"usches,
so wird die Potenz desselben dargestellt durch:
\[
CP \centerdot CR = CO^2 = CQ^2 .
\]
Der Punkt $C$ halbirt die Strecke $OQ$, wenn der Tr\"ager der
Punktreihe eine Gerade ist. Wenn also auf einer Geraden
die Punkte $P$, $R$ harmonisch durch $O$ und $Q$ getrennt sind,
so ist die Potenz des Punktenpaares $P$, $R$ im Halbirungspunkte
$C$ der Strecke $OQ$ gleich dem Quadrate der H\"alfte
dieser Strecke; der Punkt, von welchem dieser Halbirungspunkt
durch $O$ und $Q$ harmonisch getrennt ist, liegt folglich
unendlich fern.

33. Durch reciproke Radien verwandeln sich die Punktenpaare
einer involutorischen Punktreihe $k$ in diejenigen
einer involutorischen Punktreihe $k'$, und die Ordnungspunkte
von $k$ in die von $k'$ (30.). Vier harmonische Punkte $OPQR$
eines Kreises oder einer Geraden $k$ werden folglich durch
reciproke Radien allemal wieder in vier harmonische Punkte
$O'P'Q'R'$ transformirt. Nimmt man das Centrum der reciproken
Radien auf der Kugel an, welche in $O$ und $Q$ die
Linie $k$ rechtwinklig schneidet, so wird $\overline{O'Q'}$ ein Durchmesser
des Kreises $k'$ und $O'P'Q'R'$ ein zu $\overline{O'Q'}$ symmetrisch liegendes
harmonisches Kreisviereck; liegt jenes Centrum zugleich
auf der Kugel, welche in $P$ und $R$ zu $k$ normal ist,
so wird $O'P'Q'R'$ ein Quadrat. Jede harmonische Punktreihe
$OPQR$ kann folglich durch reciproke Radien in die
%-----File: 027.png---------------------------------
Eckpunkte eines Quadrates $O'P'Q'R'$ verwandelt werden;
und da je zwei Gegenpunkte des letzteren durch die anderen
beiden Gegenpunkte harmonisch getrennt sind, so ergiebt
sich der wichtige Satz: Wenn auf einer Kreislinie oder Geraden
die Punkte $P$ und $R$ harmonisch getrennt sind durch
$O$ und $Q$, so sind auch $O$ und $Q$ harmonisch getrennt durch
$P$ und $R$.

34. Wir wollen diesen Satz noch auf andere Art beweisen.
Jede Kugel, welche durch ein Punktenpaar $P$, $R$
der involutorischen Punktreihe $k$ geht, geh\"ort zu dem durch
$k$ bestimmten Kugelgeb\"usch und schneidet dessen Orthogonalkugel
rechtwinklig; insbesondere gilt dieses von der Kugel,
welche den Tr\"ager der Punktreihe $k$ in $P$ und $R$ rechtwinklig
schneidet. In dem Mittelpunkte $C_1$ dieser Kugel
haben folglich der Kreis $k$ und jene Orthogonalkugel gleiche
Potenz, und zwar ist diese Potenz gleich dem Quadrate des
Radius $C_1 P$ der Kugel (4.). Also muss $C_1$ auf der Potenzaxe
der Orthogonalkugel und des Kreises $k$ liegen (5., 8.);
diese Potenzaxe aber geht durch die Ordnungspunkte $O$ und
$Q$ der Punktreihe $k$, und es ist:
\[
C_1 O \centerdot C_1 Q = C_1 P^2 = C_1 R^2.
\]
Dieselbe Gleichung ergiebt sich unmittelbar aus (4.), wenn
der Tr\"ager der Punktreihe $k$ eine Gerade ist; sie bedeutet,
dass die Punkte $O$ und $Q$ ebenso durch $P$ und $R$ harmonisch
getrennt sind, wie $P$ und $R$ durch $O$ und $Q$. Von zwei beliebigen
Punktenpaaren eines Kreises oder einer Geraden ist
demnach entweder jedes oder keines durch das andere harmonisch
getrennt.

35. Durch drei Punkte eines Kreises oder einer Geraden
ist der vierte harmonische Punkt v\"ollig bestimmt, sobald
angegeben ist, von welchem der drei Punkte er getrennt
sein soll (31., 32.). --- Die Orthogonalkugel eines Kugelgeb\"usches
schneidet jeden Kreis, welcher durch ein Punktenpaar
$P$, $R$ des Geb\"usches geht, in zwei durch $P$ und $R$ harmonisch
getrennten Punkten $O$, $Q$ (31., 34.). --- Ein Kreis,
welcher zwei zu einander normale Kugeln schneidet, und
zwar die eine rechtwinklig, hat mit denselben vier harmonische
Punkte gemein; insbesondere schneidet jeder Durchmesser
der einen Kugel, welcher eine Secante der anderen
%-----File: 028.png--------------------------------
ist, die beiden Kugeln in vier harmonischen Punkten. Denn
die eine Kugel ist die Orthogonalkugel eines Geb\"usches,
welchem die andere Kugel und auch der Kreis angeh\"ort,
und die gemeinschaftlichen Punkte $P$, $R$ dieser letzteren bilden
ein Punktenpaar dieses Geb\"usches. --- Wenn drei Kreise
einer Kugel oder Ebene $\varkappa$ sich gegenseitig unter rechten
Winkeln schneiden, so hat jeder von ihnen mit den beiden
anderen vier harmonische Punkte gemein; zum Beweise lege
man durch zwei von den drei Kreisen Kugeln, welche zu $\varkappa$
normal sind.

36. Es sei $OPQR$ ein harmonisches Viereck in einem
Kreise $k$; die Tangenten von $k$ in den Punkten $O$ und $Q$
m\"ogen sich demgem\"ass in einem Punkte $C$ der Diagonale
$\overline{PR}$ schneiden. Dann sind die Dreiecke $OPC$ und $ROC$ \"ahnlich,
weil sie bei $C$ denselben Winkel haben und ihre Winkel
$OPC$ und $ROC$ als Peripheriewinkel \"uber dem Kreisbogen
$\stackrel{\frown}{OR}$ gleich sind; und ebenso ist $\triangle QPC \sim \triangle RQC$. Daraus
folgt:
\[
OP : RO = PC : OC \text{ und } QP : RQ = PC : QC,
\]
und weil die Tangenten $OC$ und $QC$ gleiche L\"ange haben:
\[
OP : RO = QP : RQ \text{ oder } RQ \centerdot OP = RO \centerdot QP.
\]
Die beiden Rechtecke aus den zwei Paar Gegenseiten eines
harmonischen Kreisvierecks sind demnach inhaltsgleich.

37. Wenn man den Eckpunkt $R$ eines Kreisvierecks
$OPQR$ auf dem Kreise stetig verschiebt, so nimmt von den
Seiten $RO$ und $RQ$ die eine zu und zugleich die andere ab,
und es giebt deshalb nur eine Lage des Punktes $R$, f\"ur
welche die Rechtecke aus den Gegenseiten des Kreisvierecks
$OPQR$ inhaltsgleich werden. Daraus folgt wieder der fr\"uhere
Satz, dass durch drei Kreispunkte $O$, $P$, $Q$ der vierte
harmonische, von $P$ getrennte Punkt $R$ eindeutig bestimmt
ist. Zugleich aber ergiebt sich als Umkehrung eines vorhergehenden
Satzes: Ein Kreisviereck ist harmonisch, wenn die
aus seinen Gegenseiten gebildeten Rechtecke gleichen Inhalt
haben. Auch hieraus schliesst man leicht, dass von zwei
Punktenpaaren eines Kreises entweder jedes oder keines durch
das andere harmonisch getrennt ist.

38. Indem wir uns nunmehr den harmonischen Strahlen
und Ebenen zuwenden, schicken wir folgenden H\"ulfssatz
%-----File: 029.png---------------------------------
voraus: Legt man in einer Ebene durch einen Punkt $S$ drei
Gerade $a$, $b$, $c$ und zwei Kreise $k$, $k'$, so haben die letzteren
mit den ersteren ausser $S$ noch die Eckpunkte von zwei
\"ahnlichen Dreiecken $ABC$ und $A'B'C'$ gemein. N\"amlich die
Winkel $A$, $B$, $C$ des Dreiecks $ABC$ sind als Peripheriewinkel
\"uber den B\"ogen
$\stackrel{\frown}{BC}$,% suboptimal, but we don't
$\stackrel{\frown}{CA}$,%   have any better idea
$\stackrel{\frown}{AB}$
des Kreises $k$ gleich den
resp.\ Winkeln
$\widehat{bc}$,
$\widehat{\vphantom{b}ca}$,
$\widehat{ab}$\footnote{)
  $\widehat{ab}$ bezeichnet denjenigen von $a$ und $b$ begrenzten Winkel, in
  welchem $c$ \so{nicht} liegt; und Analoges gilt von $\widehat{bc}$ und $\widehat{\vphantom{b}ca}$.});
denselben Winkeln aber sind
ebenso die Winkel $A'$, $B'$, $C'$ des Dreiecks $A'B'C'$ beziehungsweise
gleich, so dass $\angle A = A', B = B', C = C'$ und folglich
$\triangle ABC \sim \triangle A'B'C'$ wird. --- Wir k\"onnen den H\"ulfssatz
sofort zu dem folgenden Satze erweitern: Legt man in
der Ebene durch einen Punkt $S$ irgend $n$ Gerade $a$, $b$, $c$, $d\ldots$
und zwei Kreise $k$, $k'$, so haben die letzteren mit den ersteren
ausser $S$ noch die Eckpunkte von zwei \"ahnlichen $n$-ecken
$ABCD\ldots$ und $A'B'C'D'\ldots$ gemein. Denn die Winkel
dieser $n$-ecke sind beziehungsweise gleich und ihre Seiten
stehen in constantem Verh\"altnisse zu einander, so dass:
\[
AB : A'B' = BC : B'C' = CD : C'D' = \ldots
\]
Dieses constante Verh\"altniss ist wie man leicht findet gleich
demjenigen der Radien von $k$ und $k'$.

39. Vier Gerade $o$, $p$, $q$, $r$ eines Punktes $S$ heissen {\glqq}vier
harmonische Strahlen{\grqq}, wenn sie mit irgend einem durch $S$
gehenden Kreise $k$ ausser $S$ noch vier harmonische Punkte
$O$, $P$, $Q$, $R$ gemein haben; die Strahlen $p$ und $r$ sind {\glqq}harmonisch
getrennt{\grqq} durch $o$ und $q$ und {\glqq}einander zugeordnet{\grqq},
wenn die auf ihnen liegenden Punkte $P$ und $R$ durch $O$ und
$Q$ harmonisch getrennt sind. Die vier harmonischen Strahlen
$o$, $p$, $q$, $r$ haben aber nicht blos mit $k$, sondern auch mit
jedem anderen durch $S$ gehenden Kreise $k'$ ihrer Ebene ausser
$S$ noch vier harmonische Punkte $O'$, $P'$, $Q'$, $R'$ gemein. Denn
die Vierecke $OPQR$ und $O'P'Q'R'$ sind \"ahnlich (38.), und
aus der Bedingungsgleichung:
\[
OP : RO = QP : RQ \quad\text{oder}\quad RQ \centerdot OP = RO \centerdot QP
\]
f\"ur das harmonische Kreisviereck $OPQR$ folgt deshalb:
\[
O'P' : R'O' = Q'P' : R'Q' \quad\text{oder}\quad R'Q' \centerdot O'P' = R'O' \centerdot Q'P';
\]
%-----File: 030.png---------------------------------
wegen dieser letzteren Gleichung aber ist auch $O'P'Q'R'$ ein
harmonisches Viereck (37.).

40. Transformiren wir alle durch $S$ gehenden Kreise
der Ebene mittelst reciproker Radien, deren Centrum $S$ ist,
so erhalten wir alle nicht durch $S$ gehenden Geraden der
Ebene; und da vier harmonische Punkte allemal wieder in
vier harmonische Punkte, die Strahlen $o$, $p$, $q$, $r$ aber in sich
selbst transformirt werden, so ergiebt sich der wichtige
Satz: Vier harmonische Strahlen $o$, $p$, $q$, $r$ haben nicht
allein mit jedem durch ihren Schnittpunkt $S$ gehenden
Kreise, sondern auch mit jeder nicht durch $S$ gehenden Geraden
der Ebene vier harmonische Punkte gemein. Auch
leuchtet ein, dass vier Strahlen eines Punktes $S$ harmonisch
sind, wenn sie von irgend einer Geraden in vier harmonischen
Punkten geschnitten werden; die Gerade n\"amlich verwandelt
sich durch reciproke Radien vom Centrum $S$ in einen Kreis,
welcher mit den vier Strahlen ausser $S$ noch vier harmonische
Punkte gemein hat.

41. Durch drei Strahlen $o$, $p$, $q$, die in einer Ebene
durch einen Punkt $S$ gehen, ist der vierte harmonische Strahl
$r$ eindeutig bestimmt, sobald angegeben ist, von welchem
der drei Strahlen er getrennt sein soll (35.). Um ihn zu
construiren, bringe man $o$, $p$, $q$ mit einem durch $S$ gehenden
Kreise oder mit irgend einer Geraden der Ebene zum Durchschnitt
in den Punkten $O$, $P$, $Q$ und construire zu diesen den
vierten harmonischen Punkt $R$; derselbe liegt auf $r$. --- Jede
Gerade der Ebene, welche zu einem der vier harmonischen
Strahlen parallel ist, schneidet die drei \"ubrigen in \"aquidistanten
Punkten; denn wenn von vier harmonischen Punkten
einer Geraden der eine unendlich fern liegt, so halbirt der
von ihm getrennte Punkt die Strecke zwischen den \"ubrigen
beiden Punkten (32.). --- Die Halbirungslinien von zwei
Nebenwinkeln sind durch die Schenkel der Winkel harmonisch
getrennt (31.), und wenn von vier harmonischen Strahlen
zwei getrennte zu einander normal sind, so halbiren sie
die Winkel zwischen den beiden \"ubrigen Strahlen; zum Beweise
bringe man die Strahlen mit einem durch ihren Schnittpunkt
gehenden Kreise zum zweiten Male zum Durchschnitt.

42. Vier durch eine Gerade $s$ gehende Ebenen $\omega$, $\pi$, $\varkappa$, $\varrho$
heissen {\glqq}vier harmonische Ebenen{\grqq}, wenn sie von irgend
%-----File: 031.png---------------------------------
einer f\"unften Ebene $\varepsilon$ in vier harmonischen Strahlen $o$, $p$, $q$, $r$
geschnitten werden; die Ebenen $\pi$ und $\varrho$ sind {\glqq}harmonisch
getrennt{\grqq} durch $\omega$ und $\varkappa$ und einander zugeordnet, wenn die
in ihnen liegenden Strahlen $p$ und $r$ durch $o$ und $q$ harmonisch
getrennt sind. Die vier harmonischen Ebenen werden
nicht blos von $\varepsilon$, sondern auch von jeder anderen Ebene $\varepsilon'$,
die nicht durch die Gerade (oder {\glqq}Axe{\grqq}) $s$ geht, in vier
harmonischen Strahlen geschnitten; diese vier Strahlen n\"amlich
schneiden sich in einem Punkte von $s$ und gehen durch
die vier harmonischen Punkte, welche $\varepsilon'$ mit den harmonischen
Strahlen $o$, $p$, $q$, $r$ gemein hat (40.). Jede zur Axe
$s$ windschiefe Gerade und jeder die Axe in einem Punkte
schneidende Kreis hat folglich mit den vier harmonischen
Ebenen vier harmonische Punkte gemein.

43. Eine Gerade, welche zu einer der vier harmonischen
Ebenen parallel ist, schneidet die \"ubrigen drei in aequidistanten
Punkten (41.). Die harmonischen Ebenen werden
von jeder zu ihrer Axe $s$ parallelen Ebene $\varepsilon_1$ in vier parallelen
Strahlen geschnitten, welche mit den in $\varepsilon_1$ liegenden
Transversalen je vier harmonische Punkte gemein haben (42.)
und deshalb ebenfalls harmonische Strahlen genannt werden.
Vier parallele oder durch eine Axe $s$ gehende Ebenen sind
harmonisch, wenn sie von irgend einer Geraden in vier harmonischen
Punkten oder von irgend einer Ebene in vier
harmonischen Strahlen geschnitten werden. Durch drei
Ebenen einer Axe ist die vierte harmonische bestimmt.

\begin{center}
\makebox[15em]{\hrulefill}
\end{center}


\abschnitt{\S.~5.\\[\parskip]
Kugelb\"undel und Kugelb\"uschel. Orthogonale Kreise.}\label{p5}


\hspace{\parindent}%
44. Die Gesammtheit aller Kugeln und Kreise, welche
zwei verschiedenen Kugelgeb\"uschen zugleich angeh\"oren, bezeichnen
wir mit dem Namen {\glqq}Kugelb\"undel{\grqq}. Demgem\"ass
sagen wir, zwei Kugelgeb\"usche durchdringen oder schneiden
sich in einem Kugelb\"undel und haben einen B\"undel mit
einander gemein; derselbe liegt in den beiden Geb\"uschen
und ist ihr Schnitt. Durch einen beliebigen Punkt $P$ geht
allemal ein Kreis des Kugelb\"undels; dieser Kreis verbindet
den Punkt $P$ mit den Punkten $P'$ und $P''$, welche ihm in
den beiden Geb\"uschen zugeordnet sind, und liegt auf allen
%-----File: 032.png---------------------------------
durch $P$ gehenden Kugeln des B\"undels. Alle durch einen
Kreis des B\"undels gehenden Kugeln geh\"oren zu dem B\"undel.
Zwei beliebige Punkte $P$, $Q$ k\"onnen deshalb allemal durch
eine Kugel des B\"undels verbunden werden, und das Gleiche
gilt von zwei beliebigen Kreisen des B\"undels.

45. Alle Kugeln, welche zwei gegebene Kugeln oder
einen gegebenen Kreis oder eine Gerade rechtwinklig schneiden,
bilden mit ihren Schnittkreisen zusammen einen Kugelb\"undel
(13.). Wenn die Centra $C$ und $C_1$ von zwei Kugelgeb\"uschen
zusammenfallen, so besteht ihr gemeinschaftlicher
Kugelb\"undel aus allen durch $C$ gehenden Ebenen und Geraden
und ist ein gew\"ohnlicher Ebenen- oder Strahlenb\"undel
mit dem Mittelpunkte $C$. Sind dagegen, wie wir jetzt annehmen
wollen, die Centra $C$ und $C_1$ der Geb\"usche zwei
verschiedene Punkte, so enth\"alt der Kugelb\"undel keine anderen
Ebenen, als die durch die Gerade $\overline{CC_1}$ gehenden. Diese
Gerade nennen wir die {\glqq}Potenz-Axe{\grqq} oder k\"urzer die {\glqq}Axe
des Kugelb\"undels{\grqq}; durch eine Drehung um dieselbe \"andert
sich der B\"undel nicht. Da jeder Punkt, welcher mit zwei
Potenzpunkten von zwei oder mehreren Kugeln in einer Geraden
liegt, selbst ein Potenzpunkt dieser Kugeln ist (6.), so
ergiebt sich: Die Kugeln des B\"undels haben nicht blos in
jedem der Punkte $C$ und $C_1$, sondern \"uberhaupt in jedem
Punkte der Potenz-Axe $\overline{CC_1}$ gleiche Potenz.

46. In dem Kugelb\"undel durchdringen sich nicht blos
zwei, sondern unendlich viele Kugelgeb\"usche, und zwar ist
jeder Punkt seiner Axe $\overline{CC_1}$ das Centrum von einem dieser
Geb\"usche (45.). Von den Orthogonalkugeln dieser Geb\"usche
werden alle Kugeln des B\"undels rechtwinklig geschnitten.
In dem Mittelpunkte einer jeden Kugel des B\"undels haben
deshalb diese seine Orthogonalkugeln gleiche Potenz (4.),
und die Kugeln des B\"undels haben eine gemeinschaftliche
Centralebene, n\"amlich die Potenzebene der Orthogonalkugeln,
welche auf der Centrallinie der letzteren, d.~h.\ auf der Axe
$\overline{CC_1}$ normal steht (6., 7.). Diese Centralebene des B\"undels,
in welcher die Mittelpunkte aller seiner Kugeln liegen, ist
zugleich die Orthogonalebene eines durch den B\"undel gehenden
symmetrischen Kugelgeb\"usches, dessen Mittelpunkt
auf der Axe $\overline{CC_1}$ unendlich fern liegt (13.). --- Durch jeden
%-----File: 033.png---------------------------------
Punkt $P$ geht eine Orthogonalkugel des B\"undels; dieselbe
schneidet den durch $P$ gehenden Kreis des B\"undels (44.)
rechtwinklig in $P$ und ihr Mittelpunkt liegt auf der Axe $\overline{CC_1}$.

47. Um einen Kugelb\"undel zu bestimmen, kann man
entweder zwei durch ihn gehende Kugelgeb\"usche, oder zwei
seiner Orthogonalkugeln, oder seine Axe und eine seiner
Kugeln willk\"urlich annehmen. Drei beliebige Kugeln, welche
nicht eine gemeinschaftliche Potenzebene haben, bestimmen
einen durch sie gehenden Kugelb\"undel; ihre Potenz-Axe
n\"amlich ist die Axe dieses B\"undels, und jedes Kugelgeb\"usch,
welches die drei Kugeln enth\"alt, geht durch den B\"undel.
Ein Kugelb\"undel kann deshalb mit jeder nicht in ihm
enthaltenen Kugel durch ein Kugelgeb\"usch verbunden werden
(12.).

48. Wenn die Axe eines Kugelb\"undels mit irgend einer
nicht durch sie gehenden Kugel desselben einen Punkt $M$
gemein hat, so gehen durch $M$ alle Kugeln und Kreise des
B\"undels; denn sie haben in $M$ die gleiche Potenz Null. Entweder
besteht deshalb der B\"undel aus allen Kugeln und
Kreisen, welche die Axe in zwei Punkten $M$ und $N$ schneiden
oder in einem Punkte $M$ ber\"uhren, oder seine Kugeln
und Kreise haben keinen Punkt mit der Axe gemein und
ihre Potenz ist in jedem Punkte der Axe positiv. In dem
letzteren Falle giebt es in der Central-Ebene des B\"undels
einen Kreis, welcher alle Kugeln des B\"undels rechtwinklig
schneidet, den {\glqq}Orthogonalkreis{\grqq}; der Mittelpunkt desselben
liegt auf der Axe, und die Potenz des B\"undels in diesem
Mittelpunkte ist gleich dem Quadrate seines Radius (4.).
Dieser Orthogonalkreis ist der Ort aller Punktkugeln des
B\"undels und in ihm schneiden sich alle Orthogonalkugeln
desselben. Wenn dagegen alle Kugeln des B\"undels sich in
zwei Punkten schneiden, so reduciren sich auf diese Punkte
zwei Orthogonalkugeln des B\"undels; dieser selbst aber enth\"alt
keine Punktkugeln und seine Orthogonalkugeln haben
folglich keinen Punkt mit einander gemein. Der specielle
B\"undel, dessen Kugeln die Axe in einem Punkte $M$ ber\"uhren,
hat alle Kugeln, welche in $M$ die Axe rechtwinklig schneiden
und folglich einander in $M$ ber\"uhren, zu Orthogonalkugeln.

%-----File: 034.png---------------------------------

49. Die Gesammtheit aller Kugeln, welche drei verschiedenen,
nicht durch einen und denselben B\"undel gehenden
Kugelgeb\"uschen zugleich angeh\"oren, nennen wir einen
{\glqq}Kugelb\"uschel{\grqq}. Jedes der drei Geb\"usche schneidet den
B\"undel, welchen die beiden \"ubrigen mit einander gemein
haben, in diesem Kugelb\"uschel. Durch einen beliebigen
Punkt $P$ geht allemal eine Kugel des B\"uschels; dieselbe verbindet
den Punkt $P$ mit den drei Punkten $P'$, $P''$ und $P'''$,
welche ihm in den drei Geb\"uschen zugeordnet sind. Alle
Kugeln, welche drei beliebig angenommene Kugeln oder eine
Kugel und einen beliebigen Kreis rechtwinklig schneiden,
bilden einen Kugelb\"uschel (13., 45.), ebenso alle durch drei
Punkte, d.~h.\ durch einen Kreis gehenden Kugeln. Liegen
die Centra von drei Geb\"uschen in einer Geraden, so besteht
ihr gemeinsamer Kugelb\"uschel aus allen durch diese Gerade
gehenden Ebenen (vgl. 45.); bilden dagegen, wie wir jetzt
annehmen wollen, diese Centra ein Dreieck, so ist dessen
Ebene die einzige des B\"uschels und zugleich (6.) Potenz-Ebene
von je zwei Kugeln desselben. Diese Ebene heisst
die {\glqq}Potenz-Ebene des B\"uschels{\grqq}, weil seine Kugeln in jedem
Punkte der Ebene gleiche Potenz haben.

50. In dem Kugelb\"uschel durchdringen sich nicht blos
drei, sondern unendlich viele Kugelgeb\"usche und Kugelb\"undel;
und zwar ist jeder Punkt seiner Potenzebene das Centrum
von einem dieser Geb\"usche und jede Gerade derselben die
Axe von einem dieser B\"undel (49.). Die Orthogonalkugeln
und Orthogonalkreise aller durch den B\"uschel gehenden Geb\"usche
und B\"undel schneiden jede Kugel des B\"uschels rechtwinklig
und haben in deren Centrum gleiche Potenz; sie
bilden folglich einen Kugelb\"undel. Ebenso bilden die Orthogonalkugeln
eines Kugelb\"undels einen B\"uschel, weil sie drei
beliebige Kugeln des B\"undels rechtwinklig schneiden (49.).
Ueberhaupt geh\"ort zu jedem Kugelb\"uschel ein zu ihm orthogonaler
Kugelb\"undel und zu jedem B\"undel ein zu ihm orthogonaler
B\"uschel. Die Mittelpunkte aller Kugeln des B\"undels
liegen in der Potenz-Ebene des zugeh\"origen B\"uschels und
diejenigen aller Kugeln des B\"uschels liegen in der Potenz-Axe
des B\"undels.

51. Um einen Kugelb\"uschel zu bestimmen, kann man
entweder drei durch ihn gehende Geb\"usche, oder drei seiner
%-----File: 035.png---------------------------------
Orthogonalkugeln, oder seine Potenz-Ebene und eine seiner
Kugeln, oder endlich zwei seiner Kugeln willk\"urlich annehmen.
Bei der letzten Annahme ist die Potenz-Ebene der
beiden Kugeln zugleich diejenige des B\"uschels; sie enth\"alt
die Centra aller durch den B\"uschel gehenden Geb\"usche. Der
B\"uschel kann mit jeder nicht in ihm enthaltenen Kugel
durch einen Kugelb\"undel verbunden werden (47.); er liegt
in jedem Geb\"usche und jedem B\"undel, mit welchem er zwei
Kugeln gemein hat; mit zwei beliebigen Kugeln oder mit
einem beliebigen Kreise oder einem anderen Kugelb\"uschel
kann er durch ein Geb\"usch verbunden werden.

52. Die Kugeln eines B\"uschels schneiden sich entweder
in einem Kreise, oder sie ber\"uhren sich in einem Punkte,
oder sie haben keinen Punkt mit einander gemein (48.). In
dem letzteren Falle enth\"alt der B\"uschel zwei Punktkugeln
$M$, $N$, durch welche alle seine Orthogonalkugeln und Orthogonalkreise
gehen (48.). In jedem Punkte $C$ der Centrale
$\overline{MN}$ des B\"uschels hat demnach das Punktenpaar $M$, $N$ dieselbe
Potenz wie diese Orthogonalkugeln, und der Radius
derjenigen Kugel des B\"uschels, welche $C$ zum Mittelpunkt
hat, ist gleich der Quadratwurzel aus jener Potenz.

53. Ein Kugelb\"uschel wird von einem beliebigen Kreise
in einer involutorischen Punktreihe geschnitten; dieselbe liegt
in dem Kugelgeb\"usch, welches (51.) den B\"uschel mit dem
Kreise verbindet. Dieser Satz erleidet nur dann eine Ausnahme,
wenn der Kreis durch einen Punkt geht, welcher auf
allen Kugeln des B\"uschels liegt. Wird der Kreis durch die
Punktkugeln des B\"uschels gelegt, wenn solche existiren, so
sind diese die beiden Ordnungspunkte der involutorischen
Punktreihe. Durch die Punktkugeln eines B\"uschels sind
folglich je zwei Punkte harmonisch getrennt, in welchen irgend
eine Kugel des B\"uschels von einem beliebigen Orthogonalkreise
desselben geschnitten wird. Selbstverst\"andlich
wird ein Kugelb\"uschel auch von einer beliebigen Geraden in
einer involutorischen Punktreihe geschnitten, und z.~B.\ die
Centrale des B\"uschels schneidet jede Kugel desselben in zwei
Punkten, welche durch die beiden Punktkugeln, wenn solche
existiren, harmonisch getrennt sind.

54. Durch reciproke Radien verwandelt sich ein Kugelb\"undel
%-----File: 036.png---------------------------------
allemal in einen Kugelb\"undel und der B\"uschel orthogonaler
Kugeln des ersteren in denjenigen des letzteren
B\"undels; denn jedes durch einen B\"undel gehende Kugelgeb\"usch
wird in ein Kugelgeb\"usch transformirt (28.). Wenn
die Kugeln eines B\"undels sich in zwei Punkten $M$, $N$ schneiden
und einer dieser Punkte zum Centrum $M$ der reciproken
Radien gew\"ahlt wird, so verwandelt sich der Kugelb\"undel
in einen B\"undel $N'$ von Ebenen und Strahlen (vgl.\ 45.),
und der zugeh\"orige Kugelb\"uschel in einen B\"uschel concentrischer
Kugeln, deren Centrum der Punkt $N'$ ist. Dieser
dem Punkte $N$ zugeordnete Punkt r\"uckt in's Unendliche, und
die concentrischen Kugeln gehen in parallele Ebenen \"uber,
wenn $M$ und $N$ zusammenfallen. --- Hat der Kugelb\"undel
einen Orthogonalkreis, und verlegt man auf diesen das Centrum
der reciproken Radien, so besteht der zugeordnete
B\"undel aus allen Kugeln, welche die dem Orthogonalkreise zugeordnete
Gerade rechtwinklig schneiden, deren Mittelpunkte
also auf dieser Geraden liegen, sowie aus den Schnittkreisen
dieser Kugeln; die Orthogonalkugeln des B\"undels aber verwandeln
sich in die Ebenen, welche sich in jener Geraden
schneiden.

55. Zwei Kreise nennen wir {\glqq}orthogonal{\grqq}, wenn je zwei
durch sie gelegte Kugeln sich rechtwinklig schneiden. Alle
Kugeln, welche durch den einen von zwei orthogonalen
Kreisen gehen, sind demnach Orthogonalkugeln des durch
den anderen gehenden Kugelb\"uschels. Zwei orthogonale
Kreise $k$ und $k_1$ greifen in einander ein, wie zwei benachbarte
Ringe einer Kette; ihre Ebenen schneiden sich rechtwinklig
in der Verbindungslinie ihrer Mittelpunkte, weil jede
von ihnen den in der anderen liegenden Kreis rechtwinklig
schneidet. Zwei durch $k$ und $k_1$ gelegte Kugeln $\varkappa$ und $\varkappa_1$
haben allemal einen Kreis $k'$ mit einander gemein, welcher
von $k$ und $k_1$ in zwei sich harmonisch trennenden Punktenpaaren
rechtwinklig geschnitten wird. Der Kreis $k$ n\"amlich
schneidet die Kugel $\varkappa_1$ und folglich auch den auf $\varkappa_1$ liegenden
Kreis $k'$ rechtwinklig, und dasselbe gilt von $k_1$, $\varkappa$ und
$k'$; man kann folglich durch $k$ und $k_1$ zwei zu einander und
zu $k'$ normale Kugeln legen, und dass diese von $k'$ in vier
harmonischen Punkten geschnitten werden, lehrt ein fr\"uherer
Satz (35.).

%-----File: 037.png---------------------------------

56. Alle Ebenen, welche zwei orthogonale Kreise $k$, $k_1$
in vier Kreispunkten schneiden, gehen durch einen Punkt $C$,
n\"amlich durch das Centrum des durch $k$ und $k_1$ bestimmten
Kugelgeb\"usches (12.); durch denselben Punkt $C$ gehen auch
die Ebenen der orthogonalen Kreise. Eine beliebig durch $C$
gelegte Ebene schneidet die beiden orthogonalen Kreise allemal
in vier harmonischen Kreispunkten (55.). Auch die
durch $C$ gehende Centrale der Kreise $k$ und $k_1$ schneidet
dieselben in zwei sich harmonisch trennenden Punktenpaaren. --- Zwei
orthogonale Kreise verwandeln sich durch
reciproke Radien allemal wieder in zwei orthogonale Kreise.
Wenn insbesondere das Centrum der reciproken Radien auf
dem einen der beiden orthogonalen Kreise angenommen wird,
so verwandelt sich dieser in eine Gerade $g$, der andere aber
in einen Kreis, dessen Ebene zu $g$ normal ist und dessen
Mittelpunkt in $g$ liegt. Man \"uberzeugt sich leicht, dass vier
Kreispunkte, von welchen zwei auf der Geraden $g$ und die
anderen beiden auf einem zu $g$ orthogonalen Kreise liegen,
harmonische Kreispunkte sind; die letzteren beiden Punkte
haben n\"amlich zu $g$ symmetrische Lage.

57. Vier Kugelfl\"achen, von welchen jede zu den drei
anderen normal ist, schneiden sich paarweise in sechs Kreisen
und zu dreien in vier Punktenpaaren. Je zwei von den
vier Punktenpaaren liegen auf einem der sechs Kreise und
trennen sich gegenseitig harmonisch (35.). Auf jeder der
vier Kugeln liegen und durch jedes der vier Punktenpaare
gehen drei von den sechs Kreisen; dieselben schneiden sich
rechtwinklig. Jeder der sechs Kreise schneidet vier von den
\"ubrigen rechtwinklig in zwei von den vier Punktenpaaren
und ist zu dem f\"unften orthogonal. Die Ebenen der sechs
Kreise schneiden sich zu dreien in den vier Verbindungslinien
der vier Punktenpaare und sind zu zweien zu einander
normal; sie gehen alle durch einen Punkt, n\"amlich durch das
Centrum des Kugelgeb\"usches, in welchem die vier Kugeln
liegen. Wenn man eine Kugel und drei zu einander normale
Durchmesserebenen derselben durch reciproke Radien transformirt,
so erh\"alt man vier zu einander normale Kugelfl\"achen.

\begin{center}
\makebox[15em]{\hrulefill}
\end{center}
%-----File: 038.png---------------------------------

\abschnitt{\S.~6. \\[\parskip]
Kreisb\"undel und Kreisb\"uschel.}\label{p6}


\hspace{\parindent}%
58. Ein {\glqq}Kreisb\"undel{\grqq} besteht aus allen Kreisen und
Punktenpaaren einer Kugel oder Ebene, die in einem gegebenen
Punkte $C$ eine bestimmte Potenz $p$ haben. Die
Kugel oder Ebene heisst der {\glqq}Tr\"ager{\grqq}, $C$ das Centrum und
$p$ die Potenz des Kreisb\"undels. Auf einer Kugel ist ein
Kreisb\"undel bestimmt, wenn sein Centrum $C$ beliebig im
Raume angenommen wird, denn seine Kreise und Punktenpaare
liegen in den durch $C$ gehenden Ebenen und Geraden;
ebenso ist er durch drei beliebige Kugelkreise bestimmt,
deren Ebenen sich in einem Punkte $C$, nicht aber in einer
Geraden schneiden. In einer Ebene ist ein Kreisb\"undel bestimmt,
wenn sein Centrum in der Ebene, ausserdem aber
seine Potenz oder einer seiner Kreise beliebig angenommen
wird. Die Kreise und Punktenpaare eines Kugelgeb\"usches,
welche auf einer beliebigen Kugel oder Ebene desselben
liegen, bilden einen Kreisb\"undel, welcher dasselbe Centrum
und dieselbe Potenz hat wie das Geb\"usch. Durch einen
Kreisb\"undel ist das ihn enthaltende Kugelgeb\"usch v\"ollig bestimmt.
Zwei beliebige Punktenpaare des Kreisb\"undels
k\"onnen allemal durch einen Kreis desselben verbunden werden~(15.).

59. Ein Kugelb\"undel wird von jeder nicht in ihm enthaltenen
Kugel oder Ebene in einem Kreisb\"undel geschnitten;
denn er kann mit ihr durch ein Geb\"usch verbunden
werden (47.), und zu diesem geh\"ort der Kreisb\"undel (58.).
Alle Kugeln und Kreise eines zweiten Geb\"usches, welche
durch die Kreise und Punktenpaare des Kreisb\"undels gehen
(15.), liegen in einem Kugelb\"undel, n\"amlich in dem Schnitt
der beiden Geb\"usche. Die Kugeln und Kreise, welche einen
beliebigen Punkt $M$ mit den Kreisen und Punktenpaaren
eines Kreisb\"undels verbinden, schneiden sich deshalb entweder
in noch einem Punkte $N$, oder sie haben in $M$ eine
gemeinschaftliche Tangente (48.). Der Kreisb\"undel, welcher
durch drei beliebige Kreise einer Ebene geht, ist hiernach
leicht zu construiren und im Allgemeinen v\"ollig bestimmt. --- Durch
reciproke Radien verwandelt sich ein Kreisb\"undel
allemal in einen Kreisb\"undel (vgl. 54.).

%-----File: 039.png-----------------------------------

60. Ist die Potenz eines Kreisb\"undels positiv, so werden
alle seine Kreise von einem bestimmten Kreise rechtwinklig
geschnitten; dieser {\glqq}Orthogonalkreis{\grqq} liegt auf der Orthogonalkugel
des durch den Kreisb\"undel gehenden Kugelgeb\"usches
(13.) und ist der Ort aller Punktkreise des B\"undels. Ist der
Tr\"ager des Kreisb\"undels eine Kugel, so enth\"alt der Orthogonalkreis
alle Punkte derselben, deren Ber\"uhrungsebenen
durch das Centrum $C$ des B\"undels gehen. Alle Kreise einer
Kugel oder Ebene, welche einen auf ihr liegenden Kreis rechtwinklig
schneiden, geh\"oren zu einem Kreisb\"undel; derselbe ist
durch seinen Tr\"ager und den gegebenen Orthogonalkreis
v\"ollig bestimmt. --- Ist die Potenz eines Kreisb\"undels negativ,
so schneidet jeder Kreis desselben alle \"ubrigen (13.).
Ist die Potenz Null, so besteht der B\"undel aus allen durch
einen Punkt $C$ gehenden Kreisen des Tr\"agers; der Punkt $C$
ist das Centrum des B\"undels, er geh\"ort zu jedem Punktenpaare
desselben und auf ihn reducirt sich der Orthogonalkreis.
Durch reciproke Radien, deren Centrum $C$ ist, verwandelt
sich dieser specielle Kreisb\"undel in ein ebenes System,
d.~h.\ in die Gesammtheit aller Geraden und Punkte
einer Ebene.

61. Ein {\glqq}Kreisb\"uschel{\grqq} besteht aus allen Kreisen,
welche zwei Kreis\-b\"un\-deln einer Kugel oder Ebene zugleich
angeh\"oren. Die Gerade, welche die Centra der beiden B\"undel
verbindet, heisst die {\glqq}Potenzaxe{\grqq} oder k\"urzer die {\glqq}Axe{\grqq}
des Kreisb\"uschels; sie ist zugleich die Axe eines den Kreisb\"uschel
enthaltenden und durch ihn bestimmten Kugelb\"undels
(58.). Die Kreise des B\"uschels haben in jedem Punkte
der Axe gleiche Potenz und ihre Ebenen gehen durch die
Axe; jeder Punkt der Axe ist folglich das Centrum eines
durch den B\"uschel gehenden Kreisb\"undels. Alle Kreise einer
Kugel oder Ebene, welche zwei willk\"urlich auf derselben angenommene
Kreise rechtwinklig schneiden, bilden einen
Kreisb\"uschel (60.); ebenso alle Kreise einer Kugel, deren
Ebenen durch eine gegebene Gerade gehen. Die Kreise eines
Kugelb\"undels, welche auf einer Kugel oder Ebene desselben
liegen, bilden einen Kreisb\"uschel, dessen Axe mit derjenigen
des Kugelb\"undels zusammenf\"allt.

62. Ein Kugelb\"uschel wird von jeder nicht in ihm enthaltenen
Kugel oder Ebene in einem Kreisb\"uschel geschnitten,
%-----File: 040.png-----------------------------------
weil er mit derselben durch einen Kugelb\"undel verbunden
werden kann (51.). Alle Kugeln eines beliebigen Geb\"usches,
welche durch die einzelnen Kreise des Kreisb\"uschels
gehen, liegen in einem Kugelb\"uschel; in demselben durchdringen
sich das Geb\"usch und der durch den Kreisb\"uschel
bestimmte Kugelb\"undel. Alle Kugeln, welche einen beliebigen
Punkt $M$ mit den Kreisen eines Kreisb\"uschels verbinden,
schneiden sich deshalb entweder in einem Kreise
oder ber\"uhren sich in $M$. Der Kreisb\"uschel, welcher durch
zwei gegebene Kreise einer Ebene oder Kugel geht, ist hiernach
leicht zu construiren und v\"ollig bestimmt. Durch jeden
Punkt des Tr\"agers geht ein Kreis des B\"uschels.

63. Zu jedem Kreisb\"uschel erh\"alt man auf demselben
Tr\"ager einen {\glqq}orthogonalen{\grqq} Kreisb\"uschel, dessen Kreise zu
denjenigen des ersteren normal sind. N\"amlich die Orthogonalkugeln
des Kugelb\"undels, welcher durch den Kreisb\"uschel
bestimmt ist (61.), schneiden den Tr\"ager des B\"uschels in
den Kreisen des zugeh\"origen orthogonalen Kreisb\"uschels.
Jeder Kreis des einen von zwei orthogonalen B\"uscheln ist
der Orthogonalkreis eines durch den anderen gehenden Kreisb\"undels.
Wenn zwei und folglich alle Kreise des einen
B\"uschels sich in zwei Punkten $M$, $N$ schneiden, so haben
die Kreise des anderen B\"uschels keinen Punkt mit einander
gemein und zwei von ihnen reduciren sich auf die Punkte $M$
und $N$. Wenn dagegen keine zwei Kreise des ersten B\"uschels
einen Punkt mit einander gemein haben, so enth\"alt dieser
B\"uschel zwei Punktkreise (48.), durch welche alle Kreise des
anderen B\"uschels gehen. Wenn endlich die Kreise des einen
B\"uschels sich in einem Punkte $M$ ber\"uhren, so schneiden
sie in $M$ die Kreise des anderen B\"uschels rechtwinklig, und
letztere ber\"uhren sich ebenfalls in $M$.

64. Wenn zwei orthogonale Kreisb\"uschel in einer Ebene
liegen, so ist die Axe eines jeden von ihnen die Centrale
des anderen; denn im Centrum eines Kreises des einen B\"uschels
haben alle Kreise des anderen gleiche Potenz (4.)
und der Ort jenes Centrums ist folglich die Potenzaxe dieses
anderen B\"uschels. Zwei orthogonale Kreisb\"uschel einer
Kugel haben zwei sich rechtwinklig kreuzende Axen, von
welchen die eine zwei Punkte $M$, $N$ mit der Kugel gemein
hat, w\"ahrend in der anderen die Ber\"uhrungsebenen von $M$
%-----File: 041.png---------------------------------
und $N$ sich schneiden (63.); jede dieser Axen steht normal
auf der Ebene, welche die andere mit dem Mittelpunkte der
Kugel verbindet; nur dann schneiden sich die beiden Axen
rechtwinklig in einem Punkte $M$, wenn die eine und folglich
(63.) auch die andere in $M$ die Kugel ber\"uhrt.

65. Durch reciproke Radien verwandeln sich zwei orthogonale
Kreis\-b\"uschel allemal in zwei orthogonale Kreisb\"uschel;
letztere liegen in einer Ebene, wenn auf dem Tr\"ager
der ersteren das Centrum der Radien angenommen wird.
W\"ahlt man dieses Centrum beliebig auf einem Kreise, welcher
alle Kreise des einen B\"uschels in ihren beiden gemeinschaftlichen
Punkten $M$, $N$ rechtwinklig schneidet, so verwandeln
sich die orthogonalen B\"uschel in zwei andere, deren
Kreise zu einander liegen wie die Meridiane und Parallelkreise
der Erdkugel; sie verwandeln sich in einen B\"uschel
concentrischer Kreise und deren Durchmesser, wenn das
Centrum der reciproken Radien mit $M$ oder $N$ zusammenf\"allt.
Wenn endlich alle Kreise der beiden orthogonalen
B\"uschel durch einen Punkt $M$ gehen, so verwandeln sie sich
durch reciproke Radien vom Centrum $M$ in zwei ebene B\"uschel
paralleler Strahlen, deren Richtungen zu einander normal
sind.

\begin{center}
\makebox[15em]{\hrulefill}
\end{center}


\abschnitt{\S.~7. \\[\parskip]
Das sph\"arische und das cyklische Polarsystem.}\label{p7}


\hspace{\parindent}%
66. Wenn durch reciproke Radien vom Centrum $C$ und
der Potenz $p$ einem beliebigen Punkte $A$ des Raumes der
Punkt $A'$ zugeordnet ist, so nennen wir diejenige Ebene $\alpha$,
welche in $A'$ zu der Geraden $\overline{CA}$ normal ist, die {\glqq}Polar-Ebene{\grqq}
oder k\"urzer die {\glqq}Polare{\grqq} des Punktes $A$; umgekehrt
nennen wir $A$ den {\glqq}Pol{\grqq} dieser Ebene $\alpha$. Zu jedem Punkte
geh\"ort eine bestimmte Polarebene und zu jeder Ebene geh\"ort
ein Pol; und zwar ist dieser Pol durch die reciproken Radien
demjenigen Punkte der Ebene zugeordnet, welcher dem
Centrum $C$ am n\"achsten liegt. Die Gesammtheit aller dieser
zusammengeh\"origen Pole und Polaren heisst ein {\glqq}r\"aumliches
Polarsystem{\grqq}; wir bezeichnen dasselbe specieller als ein
{\glqq}sph\"arisches{\grqq}, weil es, wie wir sehen werden, zu einer Kugel
in inniger Beziehung steht. Der Punkt $C$ heisst das Centrum
%-----File: 042.png---------------------------------
und die durch $C$ gehenden Geraden und Ebenen heissen
{\glqq}Durchmesser{\grqq} und {\glqq}Durchmesser-Ebenen{\grqq} des Polarsystemes.
R\"uckt ein Punkt nach irgend einer Richtung in's Unendliche,
so f\"allt seine Polare mit der zu dieser Richtung normalen
Durchmesser-Ebene zusammen. Die Polare des Centrums $C$
liegt unendlich fern.

67. Von zwei Punkten $A$, $B'$ liegt entweder keiner
oder jeder in der Polare des anderen. Sind n\"amlich diesen
Punkten die resp.\ Punkte $A'$, $B$ durch die reciproken Radien
zugeordnet, so sind die Dreiecke $CA'B'$ und $CBA$ \"ahnlich
(20.); wenn aber $B'$ in der Polare von $A$ liegt, so ist das
Dreieck $CA'B'$ bei $A'$, also auch $CBA$ bei $B$ rechtwinklig,
und der Punkt $A$ liegt folglich in der Polar-Ebene von $B'$,
welche in $B$ zu der Geraden $\overline{CBB'}$ normal ist. --- Wir
k\"onnen den eben bewiesenen Satz auch so aussprechen: Von
zwei Ebenen geht entweder keine oder jede durch den Pol
der anderen. Wenn also eine Ebene sich dreht um einen
auf ihr liegenden Punkt, so bewegt sich ihr Pol in der Polar-Ebene
dieses Punktes; und wenn umgekehrt ein Punkt
eine Ebene beschreibt, so dreht sich seine Polare um den
Pol dieser Ebene. Beschreibt ein Punkt eine Gerade $g$, bewegt
er sich also in zwei durch $g$ gehenden Ebenen zugleich,
so dreht sich seine Polare um die beiden Pole dieser Ebenen,
d.~h.\ um die Verbindungslinie $g_1$ dieser beiden Pole; jede
der beiden Geraden $g$, $g_1$ heisst die {\glqq}Polare{\grqq} der anderen.

68. In der Polare $g_1$ einer Geraden $g$ schneiden sich
die Polar-Ebenen aller Punkte von $g$ und liegen die Pole
aller durch $g$ gehenden Ebenen (67.). Wenn also zwei Gerade
in einer Ebene liegen, so gilt dasselbe von ihren Polaren;
denn diese gehen beide durch den Pol jener Ebene.
Die Pole paralleler Ebenen liegen (66.) auf einem Durchmesser,
welcher die Ebenen rechtwinklig schneidet; die Polaren
paralleler Geraden liegen folglich auf einer Durchmesser-Ebene,
welche die Geraden rechtwinklig schneidet, und eine
beliebige Gerade kreuzt ihre Polare rechtwinklig. Die Polare
eines Durchmessers $d$ liegt unendlich fern in den zu $d$ normalen
Ebenen, und der Pol einer Durchmesser-Ebene $\delta$ liegt
unendlich fern in den zu $\delta$ normalen Geraden. Die beiden
Punkte einer Geraden und ihrer Polare, welche dem Centrum
%-----File: 043.png---------------------------------
$C$ zun\"achst liegen, sind durch die reciproken Radien einander
zugeordnet und liegen auf einem Durchmesser (vgl.\ 66.).

69. Ist die Potenz $p$ der reciproken Radien negativ, so
giebt es keinen auf seiner eigenen Polare liegenden Punkt
und keine ihre Polare schneidende Gerade. Ist dagegen $p$
positiv, so ist jeder Punkt der um das Centrum $C$ mit dem
Radius $\sqrt{p}$ beschriebenen Kugel sich selbst zugeordnet und
liegt auf seiner Polare, und jede Tangente dieser Kugel
schneidet ihre Polare rechtwinklig in dem gemeinschaftlichen
Ber\"uhrungspunkte. Wir bezeichnen in diesem Falle die Kugel
als die {\glqq}Ordnungskugel{\grqq} des r\"aumlichen Polarsystemes;
jeder Punkt derselben ist der Pol seiner eigenen Ber\"uhrungsebene.
Durch den Pol einer Ebene, welche die Ordnungskugel
schneidet, gehen die Ber\"uhrungsebenen aller Schnittpunkte
(67.); alle Punkte der Kugel, deren Ber\"uhrungsebenen
durch einen gegebenen Punkt gehen, liegen andererseits %sic
in der Polare des Punktes. Die Schnittlinie von zwei beliebigen
Ber\"uhrungsebenen der Kugel hat die Verbindungslinie der
beiden Ber\"uh\-rungs\-punkte zur Polare, und umgekehrt. Die
Axen von je zwei orthogonalen Kreisb\"uscheln der Kugel sind
demnach reciproke Polaren (64.); umgekehrt sind eine Gerade
und ihre Polare allemal die Axen von zwei orthogonalen
Kreisb\"uscheln der Kugel. Das Centrum eines Kreisb\"undels
der Kugel ist der Pol der Ebene, welche den Orthogonalkreis
des B\"undels enth\"alt (60.). Das sph\"arische Polarsystem ist
durch seine Ordnungskugel ebenso wie diese durch das Polarsystem
v\"ollig bestimmt. Ein Punkt und seine Polare
heissen deshalb auch Pol und Polare {\glqq}bez\"uglich dieser Kugel{\grqq},
und ebenso nennt man eine Gerade und ihre Polare
zwei {\glqq}reciproke Polaren bez\"uglich der Kugel{\grqq}.

70. In der Polarebene eines Punktes $A$ liegen die Polaren
aller durch $A$ gehenden Geraden (68.); zwei Ber\"uhrungsebenen
der Ordnungskugel schneiden sich demnach in
der Polare von $A$, wenn ihre Ber\"uhrungspunkte mit $A$ in
einer Geraden liegen. Zwei sich schneidende Gerade, welche
die Ordnungskugel in zwei Punkten einer durch $A$ gehenden
Secante ber\"uhren, schneiden sich folglich in einem Punkte
der Polare von $A$. Hat die Kugel mit einer Kegelfl\"ache,
deren Mittelpunkt $A$ ist, zwei Kreise gemein, so schneiden
%-----File: 044.png-------------------------------
sich die Ebenen dieser Kreise in der Polare von $A$; denn der
Schnittpunkt von je zwei in einer Ber\"uhrungsebene des Kegels
enthaltenen Tangenten der beiden Kreise liegt in der
Polare von $A$ und zugleich in den beiden Kreisebenen. Wir
k\"onnen den einen Kreis durch drei beliebige Punkte $P$, $Q$, $R$
der Kugel legen, der andere geht dann (27.) durch die Punkte
$P'$, $Q'$, $R'$, in welchen die Kugel von den Secanten $\overline{AP}$, $\overline{AQ}$,
$\overline{AR}$ zum zweiten Male geschnitten wird; in der Polare von
$A$ schneiden sich alsdann nicht blos die Ebenen $PQR$ und
$P'Q'R'$, sondern ebenso $PQR'$ und $P'Q'R$, $PQ'R$ und $P'QR'$,
sowie $P'QR$ und $PQ'R'$.

71. Bringt man also irgend zwei durch $A$ gehende Secanten
mit der Kugel zum Durchschnitt in den Punktenpaaren
$P, P'$ und $Q, Q'$, so schneiden sich die Geraden $\overline{PQ}$
und $\overline{P'Q'}$, ebenso aber $\overline{PQ'}$ und $\overline{P'Q}$ auf der Polare von $A$.
Von den Mittelpunkten der beiden Kegelfl\"achen, durch welche
zwei beliebig auf der Kugel angenommene Kreise verbunden
werden k\"onnen (27.), liegt deshalb jeder in der Polare des
anderen, und die Verbindungslinie beider hat die Schnittlinie
der beiden Kreisebenen zur Polare.

72. Wir nennen {\glqq}conjugirt{\grqq} zwei Punkte, von denen
jeder in der Polare des anderen liegt, ebenso zwei Ebenen,
von denen jede durch den Pol der anderen geht, und zwei
Gerade, von denen jede die Polare der anderen schneidet
(67., 68.). Ein Punkt und eine Gerade heissen conjugirt,
wenn die Gerade in der Polare des Punktes, also auch dieser
in der Polare der Geraden liegt. Eine Gerade und eine
Ebene endlich heissen conjugirt, wenn die Gerade durch den
Pol der Ebene und folglich die Ebene durch die Polare der
Geraden geht. Einem beliebigen Punkte $A$ sind hiernach
alle in seiner Polarebene liegenden Punkte und Geraden conjugirt,
einer Ebene alle durch ihren Pol gehenden Ebenen
und Strahlen; einer Geraden dagegen sind alle Punkte und
Ebenen ihrer Polare conjugirt, sowie alle Geraden, welche
diese Polare schneiden oder ihr parallel sind. Wenn das
Polarsystem eine Ordnungskugel hat, so sind alle Punkte,
Tangenten und Ber\"uhrungsebenen derselben sich selbst conjugirt;
denn z.~B.\ jede Ber\"uhrungsebene geht durch ihren
eigenen Pol, den Ber\"uhrungspunkt. --- Zwei Kreise der
%-----File: 045.png---------------------------------
Ordnungskugel schneiden sich nur dann rechtwinklig, wenn ihre
Ebenen conjugirt sind (60.).

73. Ist dem Punkte $A$ durch die reciproken Radien der
Punkt $A'$ zugeordnet und in dem zugeh\"origen Polarsystem
der Punkt $B$ conjugirt, so liegt die Gerade $\overline{BA'}$ in der Polare
von $A$ und schneidet den Durchmesser $\overline{CAA'}$ rechtwinklig
in $A'$. Diejenige Kugel, welche die Strecke $AB$ zum
Durchmesser hat, geht folglich auch durch $A'$ und hat im
Centrum $C$ des Polarsystemes die Potenz $CA \centerdot CA' = p$.
Folglich bilden alle Kugeln, welche eine gegebene Gerade
in je zwei conjugirten Punkten rechtwinklig schneiden, einen
Kugelb\"uschel, indem sie einerseits zu dem Kugelgeb\"usch vom
Centrum $C$ und der Potenz $p$ geh\"oren, anderseits zu dem
Kugelb\"undel, von dessen Kugeln die Gerade rechtwinklig
geschnitten wird (45). Nun wird aber ein Kugelb\"uschel von
einer Geraden in einer involutorischen Punktreihe geschnitten
(53.), wenn nicht die Gerade durch einen allen Kugeln
des B\"uschels gemeinschaftlichen Punkt geht. Die Paare
conjugirter Punkte einer jeden Geraden, welche die Ordnungskugel
des Polarsystemes nicht ber\"uhrt, bilden folglich eine
involutorische Punktreihe. Die etwa vorhandenen Ordnungspunkte
dieser Punktreihe liegen auf der Ordnungskugel des
Polarsystemes (72.) und trennen je zwei conjugirte Punkte
der Geraden harmonisch (31.). Zieht man also an eine Kugel
aus einem Punkte $A$ Secanten und bestimmt auf jeder Secante
den Punkt, welcher von $A$ durch die beiden Schnittpunkte
harmonisch getrennt ist, so erh\"alt man Punkte der
Polarebene von $A$ bez\"uglich der Kugel. --- In einer Tangente
der Ordnungskugel ist jeder Punkt dem Ber\"uhrungspunkte
conjugirt.

74. Weisen wir jedem Punkte $A$ einer nicht sich selbst
conjugirten Ebene die Gerade $a$ zu, in welcher die Ebene
von der Polare des Punktes $A$ geschnitten wird, so erhalten
wir ein {\glqq}ebenes oder cyklisches Polarsystem{\grqq}. In demselben
hat jeder Punkt $A$ die Gerade $a$ zur Polare, welche ihm in
dem sph\"arischen Polarsysteme conjugirt ist, und ebenso hat
jede Gerade den ihr conjugirten Punkt zum Pol. Zwei
Punkte oder Gerade der Ebene sind in dem ebenen Polarsysteme
conjugirt, wenn sie in dem r\"aumlichen conjugirt
%-----File: 046.png---------------------------------
sind; und umgekehrt. Die Perpendikel, welche in der Ebene
von den Punkten auf deren Polaren gef\"allt werden, schneiden
sich in einem Punkte $C_1$, dem {\glqq}Centrum{\grqq} des ebenen
Polarsystemes; dieser Punkt ist der Fusspunkt des Perpendikels,
welches von dem Centrum $C$ des r\"aumlichen Polarsystemes
auf die Ebene gef\"allt werden kann. Wenn im
ebenen Polarsysteme ein Punkt eine Gerade beschreibt, so
dreht sich seine Polare um den Pol dieser Geraden (67.).
Die etwaigen sich selbst conjugirten Punkte des ebenen Polarsystemes
liegen auf einem Kreise, dem {\glqq}Ordnungskreise{\grqq};
derselbe liegt auf der Ordnungskugel des r\"aumlichen Polarsystemes,
und seine Tangenten sind die Polaren ihrer Ber\"uhrungspunkte.
Ein dem Ordnungskreise eingeschriebenes
Viereck ist ein harmonisches Kreisviereck, wenn seine Diagonalen
conjugirt sind (31.).

75. Die Kugeln, welche die Strecken zwischen je zwei
conjugirten Punkten des ebenen Polarsystemes zu Durchmessern
haben, liegen in einem Kugelb\"undel; denn einerseits
haben sie im Centrum $C$ des r\"aumlichen Polarsystemes die
Potenz $p$ (73.), anderseits liegen sie in dem symmetrischen
Kugelgeb\"usch, in dessen Orthogonalebene das ebene Polarsystem
enthalten ist. Das Perpendikel $\overline{CC_1}$ aus dem Centrum
$C$ auf diese Ebene ist die Axe des Kugelb\"undels. Ist
$a$ die L\"ange und wie oben $C_1$ der Fusspunkt dieses Perpendikels
und bezeichnen wir mit $r$ den Radius einer beliebigen
Kugel des B\"undels, mit $d$ und $d_1$ die Abst\"ande ihres Mittelpunktes
von $C$ und $C_1$, sowie mit $p$ und $p_1$ ihre Potenz in
resp.\ $C$ und $C_1$, so ergiebt sich (2.):
\[
p = d^2 - r^2 = a^2 + d_1^2 - r^2 \quad\text{und}\quad p_1 = d_1^2 - r_1^2,
\]
woraus folgt:
\[
\quad\quad p_1 = p - a^2.
\]
Der Kreisb\"undel, in welchem der Kugelb\"undel von seiner
Orthogonalebene geschnitten wird, hat demnach den Punkt
$C_1$ zum Centrum und in ihm die Potenz $p_1 = p - a^2$. Durch
reciproke Radien vom Centrum $C_1$ und der Potenz $p_1$ ist
jedem Punkte in der Ebene sein ihm zun\"achst liegender conjugirter
Punkt zugeordnet. Wenn also die Ebene sich selbst
conjugirte Punkte enth\"alt, so ist der Ort derselben ein Kreis
vom Centrum $C_1$ und dem Halbmesser $\sqrt{p_1\vphantom{a^2}} = \sqrt{p-a^2}$; derselbe
ist der Ordnungskreis des ebenen Polarsystemes.

\begin{center}
\makebox[15em]{\hrulefill}
\end{center}
%-----File: 047.png---------------------------------

\abschnitt{\S.~8. \\[\parskip]
Kugeln und Kreise mit reellem Centrum und rein imagin\"arem Halbmesser.}\label{p8}


\hspace{\parindent}%
76. Durch reciproke Radien vom Centrum $C$ und der
Potenz $p$ ist einerseits ein Kugelgeb\"usch, anderseits ein
sph\"arisches Polarsystem bestimmt; und zwar ist die Kugel,
welche um den Mittelpunkt $C$ mit dem Radius $\sqrt{p}$ beschrieben
wird, die Orthogonalkugel des Geb\"usches (13.) und zugleich
die Ordnungskugel des Polarsystemes (69.). Diese
Kugel ist der Ort aller Punktkugeln des Geb\"usches, aller
sich selbst conjugirten Punkte und Ebenen des Polarsystemes
und aller Punkte, welche durch die reciproken Radien sich
selbst zugeordnet sind; durch sie sind die reciproken Radien,
das r\"aumliche Polarsystem und das Kugelgeb\"usch v\"ollig
bestimmt.

77. Wir wollen nun die Kugel als gegeben betrachten,
wenn ihr Mittelpunkt $C$ und die Potenz $p$ der durch sie bestimmten
reciproken Radien gegeben sind, und zwar auch
dann, wenn $p$ negativ und folglich der Halbmesser $\sqrt{p}$ rein
imagin\"ar ist. Freilich hat die Kugel in diesem Falle keine
reellen Punkte, wohl aber sind das Kugelgeb\"usch, dessen
Orthogonalkugel sie ist, und das zugeh\"orige r\"aumliche Polarsystem
reell construirbar. Wir k\"onnen, wenn $p$ negativ
ist, das Kugelgeb\"usch, das Polarsystem und die reciproken
Radien als reelle Repr\"asentanten der Kugel vom Centrum $C$
und dem imagin\"aren Radius $\sqrt{p}$ auf\/fassen. Die Einf\"uhrung
dieser imagin\"aren Orthogonalkugeln reeller Kugelgeb\"usche
gestattet uns, viele Definitionen und S\"atze ganz allgemein
auszusprechen, die sonst nur mit Einschr\"ankungen gelten
w\"urden. So k\"onnen wir von zwei Punkten, die in einem
Kugelgeb\"usch einander zugeordnet sind, nunmehr sagen, sie
seien einander {\glqq}bez\"uglich einer Kugel{\grqq}, n\"amlich der Orthogonalkugel
des Geb\"usches, zugeordnet. Von conjugirten Punkten,
Geraden und Ebenen im sph\"arischen Polarsysteme k\"onnen
wir ebenso sagen, sie seien conjugirt {\glqq}bez\"uglich einer
Kugel{\grqq}, n\"amlich bez\"uglich der Ordnungskugel des Polarsystemes;
auch nennen wir einen beliebigen Punkt den Pol
seiner Polarebene in Bezug auf dieselbe Kugel. Von zwei
%-----File: 048.png-------------------------------------
durch reciproke Radien einander zugeordneten Figuren, Linien
oder Fl\"achen endlich wollen wir sagen, sie seien einander
zugeordnet oder invers {\glqq}in Bezug auf die Kugelfl\"ache{\grqq},
auf welcher alle sich selbst zugeordneten Punkte liegen.

78. In Uebereinstimmung mit Fr\"uherem (2.) setzen wir
fest, dass eine Kugel vom Radius $\sqrt{p}$ in einem beliebigen
Punkte $A$ die Potenz $d^2-p$ hat, wenn $d$ den Abstand des
Punktes $A$ vom Centrum der Kugel bezeichnet. Ist $p$ negativ,
so hat die Kugel in jedem Punkte des Raumes positive
Potenz. --- Jeder Punkt $A$ des Raumes ist Mittelpunkt einer
Kugel, welche in dem gegebenen Punkte $C$ die Potenz $p$ hat;
ist n\"amlich $r$ der Radius dieser Kugel und $d$ der Abstand
von $A$ und $C$, so haben wir f\"ur $r$ die Gleichung:
\[
p = d^2 - r^2, \quad\text{woraus}\quad r = \sqrt{d^2 -p}.
\]
Der Radius $r$ ist reell, wenn $p$ negativ ist, oder positiv und
kleiner als $d^2$ er wird nur dann imagin\"ar, wenn $p$ positiv
und gr\"osser als $d^2$ ist. --- Die Mittelpunkte aller Kugeln
eines Kugelgeb\"usches, welches keine Orthogonalebene hat,
erf\"ullen demnach den ganzen unendlichen Raum. Ist die
Potenz des Geb\"usches negativ, so sind alle seine Kugeln
reell; ist sie dagegen positiv, so haben nur diejenigen Kugeln
des Geb\"usches reelle Halbmesser, deren Mittelpunkte ausserhalb
seiner Orthogonalkugel liegen. --- Jeder Punkt $A$ der
Centralebene eines gew\"ohnlichen Kugelb\"undels oder der Centrale
eines Kugelb\"uschels ist der Mittelpunkt einer Kugel
desselben; n\"amlich alle Orthogonalkugeln des B\"undels oder
B\"uschels haben in $A$ gleiche Potenz und die Quadratwurzel
aus dieser Potenz ist der Radius jener Kugel.

79. Zwei Kugeln bestimmen auch dann, wenn einer
oder jeder ihrer Radien imagin\"ar ist, einen durch sie gehenden
Kugelb\"uschel. Unmittelbar n\"amlich bestimmen sie als
Orthogonalkugeln von zwei Kugelgeb\"uschen einen Kugelb\"undel,
in welchem diese beiden Geb\"usche sich durchdringen;
die Orthogonalkugeln dieses B\"undels aber bilden den durch
die beiden Kugeln gehenden B\"uschel (50.). Die Centralebene
des B\"undels, welche auf der Centrale des B\"uschels
normal steht, ist die Potenzebene der beiden Kugeln, denn
letztere haben in dem Centrum einer jeden Kugel des B\"undels
gleiche Potenz. Da demnach zwei beliebige Kugeln,
%-----File: 049.png-------------------------------------
auch wenn ihre Radien rein imagin\"ar sind, eine ganz bestimmte
Potenzebene haben, so bleiben die fr\"uheren S\"atze
(8., 9.), dass im Allgemeinen drei Kugeln eine Potenzaxe
und vier Kugeln einen einzigen Potenzpunkt haben, nebst
ihren Beweisen auch ferner g\"ultig. Im Allgemeinen bestimmen
folglich auch dann drei Kugeln einen durch sie gehenden
B\"undel und vier Kugeln ein sie enthaltendes Geb\"usch,
wenn sie alle oder zum Theil imagin\"are Radien haben (vgl.\ 12., 47.).

80. Eine Punktkugel $M$ bestimmt mit einer beliebigen,
nicht durch $M$ gehenden Kugel $\varkappa$ einen Kugelb\"uschel, welcher
noch eine zweite Punktkugel $N$ enth\"alt (52.). Zu der
Potenzebene des B\"uschels liegen die Punkte $M$ und $N$ symmetrisch
(10.); ausserdem sind sie in Bezug auf die Kugel $\varkappa$
einander zugeordnet, weil die Potenz des Punktenpaares $M$, $N$
im Centrum von $\varkappa$ gleich dem Quadrate des Radius von $\varkappa$
ist (52.). Da nun die Polarebene des Punktes $M$ in Bezug
auf $\varkappa$ die Centrale $\overline{MN}$ in dem zugeordneten Punkte $N$
rechtwinklig schneidet, so ergiebt sich der Satz: {\glqq}Die Potenzebene,
welche eine Punktkugel $M$ mit einer beliebigen Kugel
$\varkappa$ bestimmt, ist parallel zu der Polarebene des Punktes $M$
in Bezug auf $\varkappa$ und halbirt das von $M$ auf diese Polarebene
gef\"allte Perpendikel{\grqq}. Alle Kugeln, in Bezug auf welche der
Punkt $M$ eine gegebene Ebene $\mu$ zur Polare hat, bilden einen
Kugelb\"uschel, von welchem $M$ und der Fusspunkt des von $M$
auf $\mu$ gef\"allten Perpendikels die beiden Punktkugeln sind.
Alle Kugeln, in Bezug auf welche dem Punkte $M$ eine Gerade
$m$ oder ein Punkt $M'$ conjugirt ist, bilden folglich
einen Kugelb\"undel resp.\ ein Geb\"usch; die Orthogonalkugel
des letzteren geht durch $M$ und $M'$ und hat die Strecke
$MM'$ zum Durchmesser.

81. In der Ebene ist durch reciproke Radien vom
Centrum $C'$ und der Potenz $p'$ einerseits ein Kreisb\"undel,
anderseits ein ebenes Polarsystem bestimmt, und zwar ist
der Kreis, welcher um den Mittelpunkt $C'$ mit dem Radius
$\sqrt{p'}$ beschrieben wird, der Orthogonalkreis des B\"undels (60.)
und zugleich der Ordnungskreis des Polarsystemes (74., 75.).
Wir wollen diesen Kreis durch seine Ebene, seinen Mittelpunkt
$C'$ und die Potenz $p'$ der reciproken Radien auch
%-----File: 050.png---------------------------------
dann als gegeben betrachten, wenn $p'$ negativ, also der
Kreisradius $\sqrt{p'}$ imagin\"ar ist. In diesem Falle sind die reciproken
Radien in der Ebene, der ebene Kreisb\"undel und das
ebene Polarsystem als reelle Repr\"asentanten des Kreises
aufzufassen.

82. Eine Kugel vom Radius $\sqrt{p}$ hat mit einer Ebene,
welche vom Centrum $C$ der Kugel den Abstand $a$ hat, einen
Kreis vom Radius $\sqrt{p'\vphantom{a^2}} = \sqrt{p-a^2}$ gemein, welcher den
Fusspunkt des von $C$ auf die Ebene gef\"allten Perpendikels
zum Mittelpunkt hat (75.). Zwei Kugeln haben allemal einen
in ihrer Potenzebene liegenden Kreis mit einander gemein,
dessen Centrum $C'$ mit denjenigen der beiden Kugeln auf
einer Geraden liegt. Denn die Potenzebene schneidet die
Centrale der Kugeln rechtwinklig in $C'$ und hat mit ihnen
folglich zwei Kreise gemein, die $C'$ zum Mittelpunkt haben;
die Radien dieser Kreise sind $\sqrt{p-a^2}$ und $\sqrt{p_1-a_1^2}$, wenn
$\sqrt{p}$ und $\sqrt{p_1}$ die Radien der beiden Kugeln und $a$ und $a_1$
die Abst\"ande ihrer Mittelpunkte von der Potenzebene bezeichnen;
weil aber die Kugeln im Punkte $C'$ gleiche Potenz
haben und folglich (78.)
\[
a^2-p = a_1^2-p_1, \quad\text{also auch}\quad \sqrt{p-a^2} = \sqrt{p_1-a_1^2}
\]
ist, so haben jene beiden Kreise gleiche Radien und sind
identisch. Es folgt aus dem soeben bewiesenen Satze, dass
alle Kugeln eines Kugelb\"uschels einen Kreis mit einander
gemein haben, welcher in der Potenzebene des B\"uschels
liegt; der Radius dieses Kreises ist entweder reell oder imagin\"ar,
das zu dem Kreise geh\"orige Polarsystem aber ist allemal
reell.

\begin{center}
\makebox[15em]{\hrulefill}
\end{center}


\abschnitt{\S.~9. \\[\parskip]
Lineare Kugelsysteme.}\label{p9}


\hspace{\parindent}%
83. Die Gesammtheit aller Kugeln, Kreise und Punktenpaare
des Raumes bezeichnen wir mit dem Namen {\glqq}Kugelsystem
von vier Dimensionen oder vierter Stufe{\grqq}; die Kugelb\"uschel,
Kugelb\"undel und -Geb\"usche dagegen wollen wir {\glqq}lineare
Kugelsysteme von ein, zwei resp.\ drei Dimensionen{\grqq}
oder {\glqq}lineare Systeme erster, zweiter resp.\ dritter Stufe{\grqq}
nennen. Von anderen Kugelsystemen unterscheiden wir die
%-----File: 051.png---------------------------------
eben genannten durch das Beiwort {\glqq}linear{\grqq}; denn w\"ahrend
jene anderen den Curven und krummen Fl\"achen vergleichbar
sind, haben diese linearen Systeme grosse Analogie mit
den geraden Linien, den Ebenen und dem r\"aumlichen Punktsystem
von drei Dimensionen. Wie eine Gerade durch zwei
und eine Ebene durch drei beliebige Punkte bestimmt ist,
so ist ein Kugelb\"uschel durch zwei, ein Kugelb\"undel durch
drei und ein Kugelgeb\"usch durch vier beliebige Kugeln bestimmt
(51., 47., 12.); und wie die drei eine Ebene bestimmenden
Punkte nicht in einer Geraden liegen d\"urfen, so
d\"urfen die drei einen B\"undel bestimmenden Kugeln nicht in
einem Kugelb\"uschel, und die vier ein Geb\"usch bestimmenden
Kugeln nicht in einem B\"undel liegen.

84. Wie eine Ebene durch jede Gerade geht, mit welcher
sie zwei Punkte gemein hat, so geht ein lineares Kugelsystem
zweiter oder dritter Stufe durch jeden Kugelb\"uschel,
mit welchem es zwei Kugeln gemein hat (51.), und ein Kugelgeb\"usch
durch jeden Kugelb\"undel, von welchem es drei
nicht in einem B\"uschel liegende Kugeln enth\"alt (47.). Alle
Geraden, welche einen Punkt mit den Punkten einer nicht
durch ihn gehenden Geraden verbinden, liegen in einer Ebene;
ebenso liegen alle Kugelb\"uschel, welche eine Kugel mit den
verschiedenen Kugeln eines nicht durch sie gehenden Kugelb\"uschels
oder -B\"undels verbinden, in einem linearen System
zweiter resp.\ dritter Stufe. Wie zwei sich schneidende Gerade
durch eine Ebene, so k\"onnen zwei Kugelb\"uschel, welche
eine Kugel mit einander gemein haben, durch einen Kugelb\"undel
verbunden werden.

85. Vier beliebige Kugelgeb\"usche haben allemal eine
und im Allgemeinen nur eine Kugel mit einander gemein;
ebenso zwei beliebige Kugelb\"undel, oder ein Kugelgeb\"usch
und ein Kugelb\"uschel. Die Orthogonalkugeln der vier Geb\"usche
haben n\"amlich einen Potenzpunkt $P$ (79.); derselbe
ist der Mittelpunkt, und die Potenz der vier Orthogonalkugeln
in $P$ ist das Quadrat des Radius jener gemeinschaftlichen
Kugel. Dieser Radius ist nur dann imagin\"ar, wenn
die vier Orthogonalkugeln alle reell sind und ihren Potenzpunkt
$P$ einschliessen (78.). --- Zwei Kugelb\"undel haben
dieselbe Kugel mit einander gemein, wie zwei Paar in ihnen
sich schneidende Kugelgeb\"usche; und ein Kugelgeb\"usch hat
%-----File: 052.png---------------------------------
mit einem Kugelb\"uschel dieselbe Kugel gemein, wie mit
drei in dem B\"uschel sich schneidenden anderen Geb\"uschen.

86. Wie zwei oder drei beliebige Ebenen sich in einer
Geraden resp.\ einem Punkte schneiden, so durchdringen sich
zwei, drei oder vier beliebige Kugelgeb\"usche in einem Kugelb\"undel,
einem Kugelb\"uschel resp.\ einer Kugel. Zwei Kugelb\"undel,
die in einem Geb\"usche liegen, haben allemal einen
Kugelb\"uschel mit einander gemein; in demselben wird das
Geb\"usch von zwei durch die beiden B\"undel gelegten anderen
Geb\"uschen geschnitten. Zwei in einem B\"undel liegende Kugelb\"uschel
haben allemal eine Kugel mit einander gemein;
denn ein Kugelgeb\"usch, welches den B\"undel in dem einen
B\"uschel durchdringt, schneidet den anderen in jener gemeinschaftlichen
Kugel (85.). Ebenso beweist man, dass ein
Kugelb\"uschel und ein -B\"undel allemal dann eine Kugel mit
einander gemein haben, wenn sie durch ein Geb\"usch verbunden
werden k\"onnen.

87. Wie die gerade Linie einfach, die Ebene zweifach
und der Raum dreifach unendlich viele Punkte enth\"alt,
ebenso enth\"alt der Kugelb\"uschel einfach, der B\"undel zweifach
und das Geb\"usch dreifach unendlich viele Kugeln (78.).
In einem B\"undel gehen durch eine beliebige Kugel $\varkappa$ desselben
einfach unendlich viele Kugelb\"uschel, von welchen jeder einfach
unendlich viele Kugeln des B\"undels enth\"alt; man erh\"alt
dieselben (86.), wenn man $\varkappa$ mit jeder Kugel eines B\"uschels,
der dem B\"undel angeh\"ort, aber nicht durch $\varkappa$ geht, durch
einen Kugelb\"uschel verbindet. L\"asst man $\varkappa$ nach und nach
mit allen Kugeln eines B\"uschels zusammenfallen, so ergiebt
sich sofort, dass der Kugelb\"undel doppelt unendlich viele
Kugelb\"uschel und folglich auch doppelt unendlich viele Kreise
enth\"alt.

88. In einem Kugelgeb\"usche gehen durch jede Kugel $\varkappa$
desselben doppelt unendlich viele Kugelb\"uschel und -B\"undel;
man erh\"alt dieselben (86.), wenn man $\varkappa$ mit jeder Kugel
und jedem B\"uschel eines B\"undels, welcher nicht durch $\varkappa$
geht, aber dem Geb\"usch angeh\"ort, durch einen B\"uschel resp.\ B\"undel
verbindet. Das Geb\"usch enth\"alt, wie sich hieraus
leicht ergiebt (vgl.\ 87.), dreifach unendlich viele Kugeln,
vierfach unendlich viele Kugelb\"uschel und Kreise, und dreifach
unendlich viele Kugelb\"undel und Punktenpaare.

%-----File: 053.png-----------------------------------

89. Durch eine beliebige Kugel $\varkappa$ gehen dreifach unendlich
viele Kugelb\"uschel, vierfach unendlich viele B\"undel
und dreifach unendlich viele Geb\"usche; man erh\"alt dieselben,
wenn man $\varkappa$ mit jeder Kugel, jedem B\"uschel und jedem
B\"undel eines nicht durch $\varkappa$ gehenden Geb\"usches durch einen
B\"uschel, einen B\"undel resp.\ ein Geb\"usch verbindet (85., 86.).
Das Kugelsystem vierter Stufe enth\"alt demnach vierfach unendlich
viele Kugeln und Kugelgeb\"usche, sechsfach unendlich
viele Kugelb\"uschel und Kreise und sechsfach unendlich viele
Kugelb\"undel und Punktenpaare. --- Durch einen B\"undel gehen
einfach und durch einen B\"uschel doppelt unendlich viele
Kugelge\-b\"u\-sche; durch einen B\"uschel gehen auch doppelt
unendlich viele B\"undel.

90. Die Gesammtheit aller Kreise und Punktenpaare
einer Kugel oder Ebene nennen wir ein {\glqq}lineares Kreissystem
dritter Stufe{\grqq}, die Kreisb\"uschel und Kreisb\"undel dagegen
bezeichnen wir als {\glqq}lineare Kreissysteme erster resp.\ zweiter
Stufe{\grqq}. Auch diese linearen Systeme sind den Geraden und
Ebenen vergleichbar. Ein Kreisb\"uschel enth\"alt einfach unendlich
viele Kreise und ist durch zwei derselben bestimmt.
Ein Kreisb\"undel enth\"alt zweifach unendlich viele Kreise,
Kreisb\"uschel und Punktenpaare; er ist bestimmt durch drei
seiner Kreise, welche nicht in einem B\"uschel liegen. Das
lineare Kreissystem dritter Stufe enth\"alt dreifach unendlich
viele Kreise und Kreisb\"undel und vierfach unendlich viele
Punktenpaare und Kreisb\"uschel. Ein lineares Kugelsystem
$n^{\text{ter}}$ Stufe wird von jeder ihm nicht angeh\"origen Kugel in
einem linearen Kreissystem $n^{\text{ter}}$ Stufe geschnitten.

\begin{center}
\makebox[15em]{\hrulefill}
\end{center}


\abschnitt{\S.~10.\\[\parskip]
Reciproke und collineare Gebilde.}\label{p10}


\hspace{\parindent}%
91. Construirt man in einem r\"aumlichen Polarsysteme
zu jedem Punkte und jeder Geraden eines beliebigen Gebildes
$\varSigma$ die Polare und zu jeder Ebene von $\varSigma$ den Pol, so erh\"alt
man ein zu $\varSigma$ {\glqq}reciprokes{\grqq} Gebilde $\varSigma_1$. Die beiden reciproken
Gebilde $\varSigma$ und $\varSigma_1$ sind auf einander {\glqq}bezogen{\grqq}, und
zwar so, dass jedem Punkte des einen eine Ebene des anderen,
n\"amlich die Polare des Punktes, entspricht, und jeder
Geraden des einen eine Gerade des anderen. Wenn $n$ Punkte
%-----File: 054.png-----------------------------------
des einen Gebildes in einer Geraden liegen, so gehen die $n$
ihnen entsprechenden oder {\glqq}homologen{\grqq} Ebenen des reciproken
Gebildes durch die entsprechende Gerade; und wenn
zwei Gerade des einen Gebildes sich schneiden, so liegen
auch die entsprechenden Geraden des andern in einer Ebene
(68.). Ist insbesondere das eine Gebilde ein ebenes, so liegt
das andere in einem Strahlenb\"undel.

92. Man nennt nun \"uberhaupt zwei B\"aume $\varSigma$ und $\varSigma_1$
{\glqq}reciprok{\grqq}, wenn sie so auf einander bezogen sind, dass
jedem Punkte von $\varSigma$ eine Ebene von $\varSigma_1$ entspricht, und jeder
Geraden oder Ebene, welche beliebige Punkte von $\varSigma$ verbindet,
eine Gerade resp.\ ein Punkt, durch welchen die entsprechenden
Ebenen von $\varSigma_1$ gehen. Zwei Gebilde heissen
reciprok, wenn sie in reciproken R\"aumen einander entsprechen.
Die Beziehungen zwischen zwei reciproken R\"aumen $\varSigma$ und
$\varSigma_1$ sind wechselseitige; auch jedem Punkte von $\varSigma_1$ entspricht
eine Ebene in $\varSigma$, und wenn ein Punkt in $\varSigma_1$ eine Gerade
oder Ebene beschreibt, so dreht sich die ihm entsprechende
Ebene in $\varSigma$ um eine Gerade resp.\ einen Punkt.

93. Zwei reciproke Fl\"achen sind so auf einander bezogen,
dass den Punkten der einen die Ber\"uhrungsebenen
der anderen entsprechen, und den Be\-r\"uh\-rungs\-ebenen der
ersteren die Punkte der letzteren. Ist also die eine Fl\"ache
{\glqq}von der $n$ten Ordnung{\grqq}, d.~h.\ hat sie mit einer nicht auf
ihr liegenden Geraden im Allgemeinen und h\"ochstens $n$ Punkte
gemein, so ist die andere {\glqq}von der $n$ten Classe{\grqq}, d.~h.\ durch
eine ihr nicht angeh\"orende Gerade gehen im Allgemeinen
und h\"ochstens $n$ von ihren Ber\"uhrungsebenen. Da beispielsweise
eine Kugelfl\"ache von der zweiten Ordnung und der
zweiten Classe ist, so ist jede zu ihr reciproke Fl\"ache von
der zweiten Classe und der zweiten Ordnung.

94. Wenn zwei R\"aume oder r\"aumliche Gebilde zu einem
und demselben dritten reciprok sind, so sind sie auf einander
{\glqq}collinear{\grqq} bezogen. Man nennt n\"amlich zwei R\"aume $\varSigma$ und
$\varSigma_1$ collinear, wenn jedem Punkte von $\varSigma$ ein Punkt von $\varSigma_1$
entspricht, und jeder Geraden oder Ebene, welche beliebige
Punkte von $\varSigma$ verbindet, eine Gerade resp.\ Ebene, welche
die entsprechenden oder {\glqq}homologen{\grqq} Punkte von $\varSigma_1$ enth\"alt.
Ebenso nennt man zwei Gebilde collinear, wenn sie
in collinearen R\"aumen einander entsprechen. Die Aehnlichkeit,
%-----File: 055.png-----------------------------------
die Congruenz und die Symmetrie sind sehr specielle
F\"alle der Collineation. Zwei collineare Fl\"achen sind von
derselben Ordnung und auch von gleicher Classe; den Punkten
und Ber\"uhrungsebenen der einen entsprechen die Punkte
resp.\ Ber\"uhrungsebenen der anderen. Wenn von zwei collinearen
Curven die eine mit einer Ebene $n$ Punkte gemein
hat, so hat die andere mit der entsprechenden Ebene gleichfalls
$n$ Punkte, und zwar die homologen $n$, gemein; liegt die
eine Curve in einer Ebene, so ist auch die andere eine
ebene Curve.

95. Wenn der eine von zwei collinearen R\"aumen einem
dritten R\"aume reciprok ist, so ist auch der andere diesem
dritten reciprok. Denn jedem Punkte des dritten Raumes
entspricht in dem ersten und dadurch auch in dem zweiten
R\"aume eine Ebene; jede dieser beiden Ebenen aber dreht
sich um eine Gerade oder einen Punkt, wenn der entsprechende
Punkt im dritten Raume eine Gerade resp.\ eine Ebene
beschreibt. --- Wenn von zwei collinearen oder insbesondere
\"ahnlichen Gebilden das eine einem dritten Gebilde reciprok
ist, so gilt dasselbe auch von dem anderen. Wenn zwei
R\"aume auf einen dritten collinear bezogen sind, so sind sie
auch zu einander collinear.

96. Zwei collineare R\"aume durchdringen sich gegenseitig,
und es kann deshalb vorkommen, dass einander entsprechende
oder {\glqq}homologe{\grqq} Elemente derselben, d.~h.\ homologe
Punkte, Strahlen oder Ebenen, zusammenfallen. Von
jedem mit seinem entsprechenden identischen Elemente der
beiden R\"aume wollen wir sagen, die collinearen R\"aume haben
das Element {\glqq}entsprechend gemein{\grqq}; und dasselbe sagen wir
von jedem Gebilde der beiden R\"aume, welches mit seinem
entsprechenden zusammenf\"allt. Beispielsweise haben zwei
\"ahnliche und \"ahnlich liegende R\"aume jede Gerade und jede
Ebene entsprechend gemein, welche durch den Aehnlichkeitspunkt
geht.

97. Zwei collineare R\"aume $\varSigma$ und $\varSigma_1$ haben {\glqq}perspective
Lage{\grqq} und heissen {\glqq}perspectiv{\grqq}, wenn sie alle
Punkte und Geraden einer Ebene $\varepsilon$, sowie alle Strahlen und
Ebenen eines Punktes $C$ entsprechend gemein haben. Mit
dem Punkte $C$, dem {\glqq}Collineationscentrum{\grqq}, liegen je zwei
einander entsprechende Punkte der collinearen R\"aume in
%-----File: 056.png---------------------------------
einer Geraden und je zwei homologe Gerade derselben in
einer Ebene; dagegen auf der {\glqq}Collineationsebene{\grqq} $\varepsilon$ schneiden
sich je zwei homologe Strahlen oder Ebenen der beiden
perspectiven R\"aume $\varSigma$ und $\varSigma_1$, weil jeder Punkt von $\varepsilon$
mit seinem entsprechenden zusammenf\"allt. Sind $C$ und $\varepsilon$,
sowie zwei beliebige einander entsprechende Elemente von
$\varSigma$ und $\varSigma_1$, z.~B.\ zwei homologe Punkte $A$ und $A_1$ gegeben,
so kann man hiernach leicht zu jedem anderen Punkte $B$
von $\varSigma$ den entsprechenden Punkt $B_1$ von $\varSigma_1$ construiren;
man bringe die Gerade $\overline{AB}$ im Punkte $S$ zum Durchschnitt
mit der Collineationsebene $\varepsilon$, dann ist $B_1$ der Schnittpunkt
der beiden Geraden $\overline{SA_1}$ und $\overline{CB}$. Eben so leicht erh\"alt man
zu jeder durch $B$ gelegten Geraden oder Ebene die entsprechende
Gerade resp.\ Ebene; dieselbe geht n\"amlich durch
$B_1$ und schneidet die erstere auf $\varepsilon$. --- R\"uckt die Collineationsebene
in's Unendliche, so sind die perspectiven R\"aume
\"ahnlich und \"ahnlich liegend, und das Collineationscentrum
$C$ ist ihr Aehnlichkeitspunkt.

98. Man kann auch Ebenen collinear oder reciprok auf
einander beziehen. Collineare Ebenen sind homologe Gebilde
von collinearen R\"aumen; sie liegen perspectiv, wenn die
collinearen R\"aume perspective Lage haben. Construirt
man in einem ebenen Polarsysteme zu jedem Punkte eines
darin angenommenen Gebildes $\varSigma$ die Polare und zu jeder
Geraden von $\varSigma$ den Pol, so erh\"alt man ein zu $\varSigma$ reciprokes
ebenes Gebilde $\varSigma_1$, und auch jedes zu $\varSigma$ collineare Gebilde
ist zu $\varSigma_1$ reciprok. Sind zwei Ebenen auf irgend eine Weise
reciprok auf einander bezogen, so entspricht jedem Punkte
der einen eine Gerade der anderen, und jeder Geraden, welche
zwei oder mehrere Punkte der einen Ebene verbindet, entspricht
ein Punkt, durch welchen die entsprechenden Geraden
der anderen Ebene gehen. Zwei Ebenen sind auf einander
collinear bezogen, wenn sie zu einer und derselben dritten
reciprok sind.

\begin{center}
\makebox[15em]{\hrulefill}
\end{center}


\abschnitt{\S.~11. \\[\parskip]
Collineare und reciproke Gebilde in Bezug auf ein Kugelgeb\"usch.}\label{p11}


\hspace{\parindent}%
99. Die Potenzebenen, welche eine beliebige Kugel $\varkappa$
mit allen Kugeln eines nicht durch $\varkappa$ gehenden Kugelb\"uschels
%-----File: 057.png-----------------------------------
bestimmt, bilden einen Ebenenb\"uschel; sie gehen n\"amlich
durch die Axe $a$ des Kugelb\"undels, welcher den Kugelb\"uschel
mit $\varkappa$ verbindet. Jede durch $a$ gehende Ebene ist die Potenzebene
von $\varkappa$ und einer bestimmten Kugel des B\"uschels
(86., 51.); der Mittelpunkt dieser Kugel liegt mit demjenigen
von $\varkappa$ auf einer zu der Ebene normalen Geraden und ist in
der Centrale des B\"uschels leicht zu construiren. Die Axe $a$
liegt in der Potenzebene des Kugelb\"uschels, kreuzt also dessen
Centrale rechtwinklig; denn durch die Axe eines Kugelb\"undels
gehen die Potenzebenen aller in dem B\"undel enthaltenen
Kugelb\"uschel. Die Axe $a$ r\"uckt in's Unendliche,
wenn der Mittelpunkt von $\varkappa$ auf der Centrale des Kugelb\"uschels
liegt oder wenn der B\"uschel aus concentrischen
Kugeln besteht (8.).

100. Die Potenzebenen und Potenzaxen, welche eine
Kugel $\varkappa$ mit allen Kugeln und Kreisen eines nicht durch $\varkappa$
gehenden Kugelb\"undels bestimmt, bilden einen Ebenen- oder
Strahlenb\"undel; sie gehen n\"amlich durch den Potenz- oder
Mittelpunkt $C$ desjenigen Geb\"usches, welches den Kugelb\"undel
mit $\varkappa$ verbindet. Man \"uberzeugt sich ohne Schwierigkeit
(86.), dass jede durch $C$ gehende Ebene zu jenen Potenzebenen
geh\"ort. Das Centrum $C$ liegt in der Axe des
Kugelb\"undels. --- Zu den Potenzebenen, welche eine Kugel
$\varkappa$ mit allen Kugeln eines nicht durch $\varkappa$ gehenden Geb\"usches
bestimmt, geh\"ort jede Ebene $\varepsilon$ des Raumes; denn der Kugelb\"uschel,
welcher $\varkappa$ mit $\varepsilon$ verbindet, hat mit dem Geb\"usch
eine Kugel $\varkappa'$ gemein (85.), und $\varepsilon$ ist die Potenzebene von
$\varkappa$ und $\varkappa'$.

101. Die Potenzebenen, welche zwei beliebige Kugeln
$\varkappa$ und $\varkappa_1$ mit den Kugeln eines nicht durch sie gehenden
Geb\"usches bestimmen, sind homologe Ebenen von zwei
perspectiv liegenden collinearen R\"aumen; und zwar ist die
Potenzebene der Kugeln $\varkappa$ und $\varkappa_1$ die Collineationsebene, und
das Centrum des Geb\"usches das Collineationscentrum dieser
perspectiven R\"aume (vgl.~97.). N\"amlich mit einer beliebigen
Kugel $\gamma$ des Geb\"usches bestimmen $\varkappa$ und $\varkappa_1$ zwei einander
entsprechende Potenzebenen, welche sich in der Potenzebene
von $\varkappa$ und $\varkappa_1$ schneiden; wenn aber $\gamma$ in dem Geb\"usche
einen Kugelb\"uschel oder -B\"undel beschreibt, so beschreiben
die beiden Potenzebenen zwei homologe Ebenenb\"uschel oder
%-----File: 058.png-----------------------------------
Ebenenb\"undel (99., 100.), deren Axen resp.\ Mittelpunkte mit
dem Centrum des Geb\"usches in einer Ebene oder Geraden
liegen, n\"amlich in der Potenzebene des Kugelb\"uschels resp.\ in
der Potenzaxe des Kugelb\"undels.

102. Der soeben bewiesene Satz gilt auch in dem besonderen
Falle, wenn $\varkappa$ und $\varkappa_1$ zwei dem Geb\"usche nicht angeh\"orige
Punktkugeln sind. Nun ist aber die Potenzebene,
welche eine Punktkugel $M$ mit der ver\"anderlichen Kugel $\gamma$
bestimmt, parallel zu der Polarebene des Punktes $M$ in Bezug
auf $\gamma$ und halbirt das von $M$ auf diese Polarebene gef\"allte
Perpendikel (80.); diese Polar- und jene Potenzebene
sind demnach homologe Ebenen von zwei \"ahnlichen und
\"ahnlich liegenden R\"aumen, von welchen $M$ der Aehnlichkeitspunkt
ist. Auch die Polarebenen von zwei Punkten in Bezug
auf die einzelnen Kugeln $\gamma$ eines Geb\"usches, dessen Orthogonalkugel
durch keinen der beiden Punkte geht, sind folglich
homologe Ebenen von zwei collinearen B\"aumen, die aber
nicht perspectiv liegen. --- Wenn ein Punkt auf der Orthogonalkugel
eines Geb\"usches liegt, so gehen seine Polarebenen
bez\"uglich aller Kugeln des Geb\"usches durch den ihm
diametral gegen\"uber liegenden Punkt der Orthogonalkugel;
denn im Centrum des Geb\"usches und dieser Orthogonalkugel
schneiden sich die Potenzebenen, welche der Punkt als Punktkugel
mit allen \"ubrigen Kugeln des Geb\"usches bestimmt.

103. Weist man dem Mittelpunkte $A$ einer ver\"anderlichen
Kugel $\gamma$ die Potenzebene $\alpha$ zu, welche $\gamma$ mit einer
gegebenen Kugel $\varkappa$ bestimmt, so beschreiben $A$ und $\alpha$ als
homologe Elemente zwei reciproke R\"aume, wenn $\gamma$ ein Kugelgeb\"usch
beschreibt; doch darf dieses Geb\"usch weder durch
$\varkappa$ gehen noch symmetrisch sein. Wenn n\"amlich $\gamma$ einen
Kugelb\"uschel oder -B\"undel des Geb\"usches beschreibt, so
durchl\"auft der Mittelpunkt $A$ eine Gerade oder Ebene und
zugleich dreht sich die Potenzebene $\alpha$ um eine Gerade resp.\ einen
Punkt. --- Ebenso erh\"alt man homologe Elemente von
zwei reciproken R\"aumen, wenn man der Polarebene eines
beliebigen Punktes in Bezug auf die ver\"anderliche Kugel $\gamma$
des Geb\"usches den Mittelpunkt von $\gamma$ als entsprechenden
Punkt zuweist (102.). --- Die Potenzebenen einer Kugel $\varkappa$
und die Polarebenen eines Punktes $M$ bez\"uglich aller Kugeln
$\gamma$ eines nicht durch $\varkappa$ oder $M$ gehenden Kugelb\"undels sind
%-----File: 059.png---------------------------------
homologe Ebenen von zwei collinearen Strahlenb\"undeln; die
Ebene, in welcher die Mittelpunkte der Kugeln $\gamma$ liegen, ist
durch den Kugelb\"undel reciprok auf jene collinearen Strahlenb\"undel
bezogen.

\begin{center}
\makebox[15em]{\hrulefill}
\end{center}


\abschnitt{\S.~12. \\[\parskip]
Harmonische Kugeln und Kreise.}\label{p12}


\hspace{\parindent}%
104. Vier Kugeln eines Kugelb\"uschels bestimmen entweder
mit keiner oder mit jeder dem B\"uschel nicht angeh\"orenden
Kugel $\varkappa$ vier harmonische Potenzebenen, und sollen im
letzteren Falle {\glqq}vier harmonische Kugeln{\grqq} heissen. N\"amlich
zwei Kugeln $\varkappa$ und $\varkappa_1$, die mit dem B\"uschel nicht in einem
und demselben Kugelb\"undel liegen, bestimmen mit jeder
Kugel des B\"uschels zwei Potenzebenen, welche auf der Potenzebene
von $\varkappa$ und $\varkappa_1$ sich schneiden; diese letztere Potenzebene
schneidet folglich die beiden Gruppen von je vier Potenzebenen,
welche $\varkappa$ und $\varkappa_1$ mit irgend vier Kugeln des
B\"uschels bestimmen, in den n\"amlichen vier Strahlen; und
jenachdem diese Strahlen harmonisch sind oder nicht, bestehen
jene beiden Gruppen aus je vier harmonischen Ebenen
oder nicht (42.).

105. Vier harmonische Kugeln bestimmen mit einer beliebigen
Kugel $\varkappa$ auch dann vier harmonische Potenzebenen,
wenn $\varkappa$ eine Punktkugel $M$ ist. Nun sind aber die Polarebenen
des Punktes $M$ bez\"uglich der vier Kugeln jenen Potenzebenen
parallel und schneiden sich wie diese in einer
Geraden (102.). Auch die Polarebenen eines beliebigen
Punktes $M$ in Bezug auf vier harmonische Kugeln sind folglich
vier harmonische Ebenen. Wenn die harmonischen
Kugeln sich in $M$ schneiden, so werden sie in diesem Punkte
von vier harmonischen Ebenen ber\"uhrt, n\"amlich von den
Polarebenen des Punktes; sie verwandeln sich folglich durch
reciproke Radien vom Centrum $M$ in vier harmonische Ebenen,
und haben mit jedem durch $M$ gelegten Kreise ausser
$M$ noch vier harmonische Punkte gemein (33., 42.).

106. Einem beliebigen Punkte $M$ sind in Bezug auf
vier harmonische Kugeln vier harmonische Punkte einer
durch $M$ gehenden Kreislinie oder Geraden zugeordnet; und
zwar (105.) einer Geraden, wenn $M$ mit den Mittelpunkten
der vier Kugeln in einer Geraden liegt. F\"allt man n\"amlich
%-----File: 060.png-----------------------------------
aus dem Punkte $M$ Perpendikel auf die Polarebenen von $M$
bez\"uglich der vier harmonischen Kugeln, so sind die Fusspunkte
dieser vier Perpendikel dem Punkte M zugeordnet
in Bezug auf die Kugeln (66.) und liegen im Allgemeinen
auf einem durch $M$ und einen gemeinschaftlichen Punkt der
vier Polarebenen gehenden Kreise, sind also (42.) vier harmonische
Punkte. Eine Ausnahme tritt ein, wenn $M$ auf
den vier Kugeln liegt oder eine Punktkugel des durch sie
gehenden B\"uschels ist. --- Die vier Strahlen, welche den
Punkt $M$ mit seinen vier zugeordneten Punkten verbinden,
sind harmonisch und gehen durch die Mittelpunkte der vier
Kugeln. Die Mittelpunkte von vier harmonischen Kugeln,
welche nicht concentrisch sind, bilden folglich eine gerade
harmonische Punktreihe.

107. Vier Kugeln eines Kugelb\"uschels sind harmonisch,
wenn bez\"uglich derselben irgend einem Punkte $M$ vier harmonische
Punkte oder vier harmonische Polarebenen zugeordnet
sind, oder wenn ihre Mittelpunkte eine harmonische
Punktreihe bilden; denn in jedem dieser F\"alle bestimmen die
vier Kugeln, wie man leicht einsieht, vier harmonische Potenzebenen
mit der Punktkugel $M$. --- Durch reciproke Radien
verwandeln sich vier harmonische Kugeln wieder in
vier harmonische Kugeln, die bei besonderer Lage des Centrums
der Radien in harmonische Ebenen \"ubergehen (105.).
Sie verwandeln sich n\"amlich in vier Kugeln eines B\"uschels
(54.), und die vier harmonischen Punkte, welche in Bezug
auf sie irgend einem Punkte $M$ zugeordnet sind, verwandeln
sich in vier harmonische Punkte, welche in Bezug auf die
anderen vier Kugeln einem Punkte $M'$ zugeordnet sind. --- Durch
drei Kugeln eines Kugelb\"uschels ist die vierte harmonische
bestimmt.

108. Die vier Kreise, welche vier harmonische Kugeln
mit einer beliebigen Kugel oder Ebene gemein haben, sollen
{\glqq}vier harmonische Kreise{\grqq} heissen; sie liegen in einem
Kreisb\"uschel und ihre Ebenen bilden, wenn sie nicht zusammenfallen,
einen harmonischen Ebenenb\"uschel (104.). Die vier
Potenzaxen, welche vier harmonische Kreise mit einer beliebigen
Kugel bestimmen, sind vier harmonische Strahlen.
Daraus folgt (vgl.~62.), dass die Kugeln, welche vier harmonische
Kreise mit einem beliebigen Punkte verbinden, vier
%-----File: 061.png---------------------------------
harmonische Kugeln sind. Durch reciproke Radien verwandeln
sich vier harmonische Kreise in vier harmonische Kreise
oder Gerade. In Bezug auf vier harmonische Kreise einer
Ebene sind einem beliebigen Punkte der Ebene vier harmonische
Punkte und zugleich vier harmonische Polaren zugeordnet
(105., 106.). Harmonische Kreise, welche sich schneiden,
werden in jedem ihrer beiden Schnittpunkte von vier
harmonischen Strahlen ber\"uhrt (105.).

\begin{center}
\makebox[15em]{\hrulefill}
\end{center}


\abschnitt{\S.~13. \\[\parskip]
Kugeln, die sich ber\"uhren. Aehnlichkeitspunkte von Kugeln.}\label{p13}


\hspace{\parindent}%
109. Wenn zwei Kugeln oder eine Kugel und eine Ebene
sich ber\"uhren, so reducirt ihr gemeinschaftlicher Kreis sich
auf einen Punkt, ist also ein Punktkreis. Eine beliebige
Kugel oder Ebene ber\"uhrt demnach h\"ochstens zwei Kugeln
eines nicht durch sie gehenden Kugelb\"uschels; denn sie
schneidet den B\"uschel in einem Kreisb\"uschel, welcher h\"ochstens
zwei Punktkreise enth\"alt (62., 63.). Die Gesammtheit
aller eine Kugel oder Ebene ber\"uhrenden Kugeln kann deshalb
als ein {\glqq}quadratisches Kugelsystem dritter Stufe{\grqq} bezeichnet
werden.

110. Durch drei gegebene Punkte oder durch einen
Kreis k\"onnen h\"och\-stens zwei Kugeln gelegt werden, welche
eine gegebene Kugel $\varkappa$ ber\"uhren. Um dieselben zu construiren,
bringe man $\varkappa$ mit irgend zwei durch die Punkte
gehenden Kugeln zum Durchschnitt, construire die Gerade $g$,
welche die Ebenen der beiden Schnittkreise mit einander
gemein haben, und lege durch $g$ Ber\"uhrungsebenen an $\varkappa$;
die Kugeln, welche die drei Punkte mit den Ber\"uhrungspunkten
dieser Ebenen verbinden, sind die gesuchten. Die
Construction wird unm\"oglich, wenn $g$ und $\varkappa$ oder, was dasselbe
ist, wenn der die drei Punkte verbindende Kreis und
$\varkappa$ sich schneiden.

111. Alle Kugeln eines Geb\"usches, welche eine dem
Geb\"usch nicht angeh\"orende Kugel $\varkappa$ ber\"uhren, werden von
noch einer Kugel $\varkappa_1$ ber\"uhrt. N\"amlich durch die zu dem
Geb\"usch geh\"origen reciproken Radien wird jede Kugel des
Geb\"usches in sich selbst, die Kugel $\varkappa$ aber in eine andere $\varkappa_1$
transformirt, und der Punkt, in welchem $\varkappa$ von irgend einer
%-----File: 062.png-----------------------------------
Kugel $\gamma$ des Geb\"usches ber\"uhrt wird, verwandelt sich in den
zugeordneten Punkt, in welchem $\varkappa_1$ dieselbe Kugel $\gamma$ ber\"uhrt.
Ist die Potenz des Geb\"usches Null, so reducirt sich die
Kugel $\varkappa_1$ auf das Centrum des Geb\"usches; ist anderseits das
Geb\"usch ein symmetrisches, so liegen $\varkappa$ und $\varkappa_1$ zu der
Orthogonalebene desselben symmetrisch und haben gleiche Radien.
Von diesen beiden speciellen F\"allen abgesehen, haben
$\varkappa$ und $\varkappa_1$ das Centrum des Geb\"usches zum Aehnlichkeitspunkt,
weil sie durch die zugeh\"origen reciproken Radien einander
zugeordnet sind (25.).

112. Bei der Lehre von den Kugeln, welche zwei oder
mehrere gegebene Kugeln ber\"uhren, spielen sonach die Aehnlichkeitspunkte
der letzteren eine Rolle, und es ist zweckm\"assig,
zun\"achst \"uber diese Aehnlichkeitspunkte das Wichtigste
anzuf\"uhren. Sind $\varkappa$ und $\varkappa_1$ zwei \"ahnliche und \"ahnlich
liegende Fl\"achen, so liegen je zwei homologe Punkte derselben
mit dem Aehnlichkeitspunkte in einer Geraden, und
je zwei homologe Sehnen sind parallel und stehen zu einander
in constantem Verh\"altnisse (vgl.~24.). Sind insbesondere
$\varkappa$ und $\varkappa_1$ zwei Kugeln, so muss demnach jeder gr\"ossten
Sehne von $\varkappa$ eine zu ihr parallele gr\"osste Sehne von $\varkappa_1$ entsprechen,
und die Endpunkte paralleler Durchmesser von $\varkappa$
und $\varkappa_1$ sowie die Mittelpunkte der Kugeln m\"ussen homologe
Punkte sein.

113. Zwei beliebige Kugeln $\varkappa$ und $\varkappa_1$ haben deshalb
nur zwei Aehnlichkeitspunkte, und zwar liegen diese auf der
Centrale der Kugeln, und durch sie gehen die zwei Paar
Geraden, welche die Endpunkte von zwei parallelen Durchmessern
der Kugeln verbinden. Die Strecken, welche einer
dieser Aehnlichkeitspunkte mit den Mittelpunkten der beiden
Kugeln begrenzt, verhalten sich zu einander wie die Radien
der Kugeln, und eben deshalb liegen die Endpunkte paralleler
Radien allemal mit einem Aehnlichkeitspunkte in einer
Geraden.

114. Man unterscheidet bei zwei Kugeln den \"ausseren
Aehnlichkeitspunkt $A$ und den inneren $J$. Der \"aussere $A$
liegt mit den Endpunkten von je zwei gleichgerichteten
parallelen Radien der Kugeln in einer Geraden, und folglich
ausserhalb der Strecke, welche die Mittelpunkte der Kugeln
begrenzen.  Im inneren Aehnlichkeitspunkte $J$ dagegen
%-----File: 063.png---------------------------------
schneiden sich die Geraden, welche die Endpunkte von je
zwei entgegengesetzt gerichteten parallelen Radien verbinden;
er liegt zwischen den Mittelpunkten der beiden Kugeln und
zwischen je zwei homologen Punkten derselben. Sind $r$ und
$r_1$ die Radien der Kugeln $\varkappa$ und $\varkappa_1$, so ist ihr
Aehnlichkeitsverh\"altniss in Bezug auf den \"ausseren Aehnlichkeitspunkt
$= r : r_1$ und in Bezug auf den inneren $= - r : r_1$; in diesem
Verh\"altniss n\"amlich stehen mit R\"ucksicht auf ihren Sinn die
Strecken zu einander, welche zwei homologe Punkte der
Kugeln mit dem betreffenden Aehnlichkeitspunkte und mit
anderen homologen Punkten bilden.

115. Eine gemeinschaftliche Ber\"uhrungs-Ebene von zwei
Kugeln geht entweder durch den \"ausseren $A$ oder durch den
inneren Aehnlichkeitspunkt $J$ derselben (114.); im letzteren
Falle liegt sie zwischen den beiden Kugeln. Wenn die Kugeln
sich \"ausserlich ber\"uhren, so f\"allt $J$ mit dem Ber\"uh\-rungs\-punkte
zusammen; wenn sie sich schneiden, so wird $J$ von
ihnen eingeschlossen, und wenn sie sich innerlich ber\"uhren,
indem die eine von der anderen eingeschlossen wird, so f\"allt
$A$ mit dem Ber\"uhrungspunkte zusammen. Umschliesst die
eine Kugel die andere, so liegen beide Aehnlichkeitspunkte
innerhalb der letzteren; sie vereinigen sich im Centrum, wenn
die Kugeln concentrisch sind. Von zwei gleichen Kugeln
liegt der \"aussere Aehnlichkeitspunkt unendlich fern, und
halbirt der innere die Strecke zwischen den beiden
Mittelpunkten. --- Zwei in einer Ebene liegende Kreise haben dieselben
zwei Aehnlichkeitspunkte wie die beiden Kugeln, von
denen sie gr\"osste Kreise sind.

116. Wenn zwei Kugeln $\varkappa$ und $\varkappa_1$ von einer dritten $\gamma$
rechtwinklig geschnitten werden, so fallen ihre Aehnlichkeitspunkte
zusammen mit den Mittelpunkten der beiden Kegelfl\"achen,
durch welche (27.) die zwei Schnittkreise $k$ und $k_1$
verbunden werden k\"onnen. Verwandelt man n\"amlich die
Kugel $\varkappa$ durch reciproke Radien, welche den Mittelpunkt
von einer dieser Kegelfl\"achen zum Centrum haben und die
Kugel $\gamma$ in sich selbst transformiren, so erh\"alt man eine
Kugel, welche im Kreise $k_1$ die Kugel $\gamma$ rechtwinklig schneidet
und deshalb mit $\varkappa_1$ identisch ist; jener Mittelpunkt ist
folglich (25.) ein Aehnlichkeitspunkt von $\varkappa$ und $\varkappa_1$.
Zugleich ergiebt sich der Satz: Zwei Kugeln k\"onnen durch reciproke
%-----File: 064.png-----------------------------------
Radien, deren Centrum $C$ ihr \"ausserer oder innerer
Aehnlichkeitspunkt ist, in einander transformirt werden, vorausgesetzt
dass sie sich nicht in $C$ ber\"uhren. Die beiden Kugeln,
in Bezug auf welche demnach zwei gegebene Kugeln $\varkappa$ und
$\varkappa_1$ einander zugeordnet sind (77.) und deren Mittelpunkte
mit den Aehnlichkeitspunkten von $\varkappa_1$ und $\varkappa_1$ zusammenfallen,
liegen \"ubrigens in dem durch $\varkappa$ und $\varkappa_1$ gehenden Kugelb\"uschel,
weil sie alle Orthogonalkugeln desselben rechtwinklig schneiden;
sie halbiren, wenn $\varkappa$ und $\varkappa_1$ sich schneiden, die von
diesen Kugeln gebildeten Winkel.

117. Wir wollen sagen, auf den Kugeln $\varkappa$ und $\varkappa_1$ liegen
zwei Punkte $P$ und $P'$ {\glqq}invers bez\"uglich des Aehnlichkeitspunktes
$C${\grqq}, wenn sie mit $C$ in einer Geraden liegen, ohne
sich zu entsprechen. Alle Paare von solchen inversen Punkten
haben in $C$ gleiche Potenz (116.) und sind Punktenpaare
eines Kugelgeb\"usches, welchem alle zu $\varkappa$ und $\varkappa_1$ rechtwinkligen
Kugeln angeh\"oren. Es k\"onnen deshalb zwei Paare
inverser Punkte allemal durch einen Kreis und drei Paare
durch eine Kugel dieses Geb\"usches verbunden werden (15.);
jede solche Kreislinie oder Kugel des Geb\"usches aber schneidet
die Kugeln $\varkappa$ und $\varkappa_1$ unter gleichen Winkeln (22.), weil sie
durch die zum Geb\"usche geh\"origen reciproken Radien in sich
selbst, zugleich aber $\varkappa$ in $\varkappa_1$ \"ubergeht. Die Kugeln $\varkappa$ und $\varkappa_1$
werden in je zwei invers liegenden Punkten von einer dritten
Kugel ber\"uhrt und von unendlich vielen anderen unter gleichen
Winkeln geschnitten.

118. Wenn zwei Kugeln $\varkappa$ und $\varkappa_1$ von einer dritten
ber\"uhrt werden, so liegen die beiden Ber\"uhrungspunkte $P$ und
$P$ mit einem Aehnlichkeitspunkte von $\varkappa$ und $\varkappa_1$ in einer
Geraden und bez\"uglich desselben invers. Denn alle Kugeln,
welche $\varkappa$ in $P$ ber\"uhren, bilden einen Kugelb\"uschel, und es
k\"onnen deshalb nur zwei von ihnen zugleich die Kugel $\varkappa_1$
ber\"uhren (109.); die Ber\"uhrungspunkte dieser beiden Kugeln
aber liegen invers zu $P$ in Bezug auf die Aehnlichkeitspunkte
von $\varkappa$ und $\varkappa_1$ (117.). --- Wenn zwei Kugeln $\varkappa$ und $\varkappa_1$ von
einer dritten $\gamma$ unter gleichen Winkeln geschnitten werden,
so liegen die beiden Schnittkreise $k$ und $k_1$ invers bez\"uglich
eines Aehnlichkeitspunktes von $\varkappa$ und $\varkappa_1$ und letzterer ist
der Mittelpunkt von einer der beiden durch $k$ und $k_1$ gehenden
Kegelfl\"achen. N\"amlich durch reciproke Radien, welche
%-----File: 065.png-----------------------------------
die Mittelpunkte dieser beiden Kegelfl\"achen zu Centren haben
und die Kugel $\gamma$ in sich selbst transformiren, verwandelt
sich $\varkappa$ in zwei andere Kugeln, welche die Kugel $\gamma$ im Kreise
$k_1$ unter denselben Winkeln schneiden wie $\varkappa_1$ und von welchen
folglich die eine mit $\varkappa_1$ zusammenf\"allt (vgl.~116.). Alle Kugeln,
welche zwei gegebene Kugeln unter gleichen Winkeln schneiden
oder auch ber\"uhren, geh\"oren also zu zwei Kugelgeb\"uschen,
deren Centra die Aehnlichkeitspunkte der gegebenen
Kugeln sind.

119. Drei Kugeln bestimmen paarweise sechs Aehnlichkeitspunkte,
n\"am\-lich drei \"aussere und drei innere; dieselben
liegen in der Central-Ebene der drei Kugeln, und zwar zu
zweien auf den drei Centrallinien derselben. Die Endpunkte
von irgend drei gleichgerichteten parallelen Radien der Kugeln
liegen mit den drei \"ausseren Aehnlichkeitspunkten in einer
Ebene (114.), und letztere liegen folglich in einer Geraden.
Die Endpunkte von drei ungleich gerichteten, parallelen
Radien dagegen liegen in einer Ebene, welche durch einen
\"ausseren und zwei innere Aehnlichkeitspunkte geht; die beiden
inneren Aehnlichkeitspunkte, welche eine der Kugeln mit den
beiden anderen Kugeln bestimmt, liegen folglich mit dem
\"ausseren Aehnlichkeitspunkte dieser beiden letzteren in einer
Geraden. Ueberhaupt liegen die sechs Aehnlichkeitspunkte
der drei Kugeln zu dreien in vier Geraden, den vier {\glqq}Aehnlichkeits-Axen{\grqq}
der Kugeln; sie bilden die Eckpunkte eines vollst\"andigen
Vierseits, dessen drei Diagonalen sich paarweise
in den Mittelpunkten der drei Kugeln schneiden. Die vier
Aehnlichkeits-Axen fallen zusammen, wenn die Mittelpunkte
der drei Kugeln in einer Geraden liegen. --- Drei Kreise
einer Ebene haben dieselben sechs Aehnlichkeitspunkte wie
die drei Kugeln, von welchen sie gr\"osste Kreise sind.

120. Jede gemeinschaftliche Ber\"uhrungs-Ebene von drei
Kugeln geht durch eine Aehnlichkeits-Axe derselben (115.).
Wenn drei Kugeln von einer vierten ber\"uhrt werden, so
gehen die Verbindungslinien der drei Ber\"uh\-rungs\-punkte
durch drei Aehnlichkeits-Punkte, und geht folglich ihre Ebene
durch eine Aehnlichkeits-Axe der Kugeln (118.). Alle Kugeln,
welche drei gegebene ber\"uhren oder unter gleichen Winkeln
schneiden, geh\"oren zu vier Kugelb\"undeln, deren Axen die
Aehnlichkeits-Axen der drei gegebenen Kugeln sind (118.).
%-----File: 066.png-----------------------------------

121. Vier Kugeln, deren Mittelpunkte nicht in einer
Ebene liegen, bestimmen paarweise zw\"olf Aehnlichkeitspunkte;
dieselben liegen zu sechsen in den vier Ebenen,
welche die Mittelpunkte von je drei der vier Kugeln verbinden,
und zu dreien in 16 Geraden (119.), den Aehnlichkeits-Axen.
Die Endpunkte von vier parallelen Radien
der Kugeln liegen zu zweien auf sechs Geraden, welche durch
sechs Aehnlichkeitspunkte, und zu dreien in vier Ebenen,
welche durch vier Aehnlichkeits-Axen der Kugel gehen; und
zwar schneiden sich diese vier Aehnlichkeits-Axen in jenen
sechs Aehnlichkeitspunkten und bilden mit ihnen zusammen
ein vollst\"andiges ebenes Vierseit. Je nachdem nun die vier
parallelen Radien gleichgerichtet sind oder nicht, ergiebt
sich daraus Folgendes. Die sechs \"ausseren Aehnlichkeitspunkte
der vier Kugeln liegen in einer Ebene und zu dreien
in vier Geraden. Die drei inneren Aehnlichkeitspunkte, welche
drei von den Kugeln mit der vierten, und die drei \"ausseren,
welche sie mit einander bestimmen, liegen zusammen in
einer Ebene und zu dreien in vier Geraden. Endlich die
vier inneren Aehnlichkeitspunkte, welche zwei von den vier
Kugeln mit den beiden anderen, und die beiden \"ausseren,
welche diese zwei Kugelpaare f\"ur sich bestimmen, liegen
zusammen in einer Ebene und zu dreien in vier Aehnlichkeits-Axen.

122. Die zw\"olf Aehnlichkeitspunkte von vier beliebigen
Kugeln liegen also (121.) zu dreien in sechzehn Geraden,
den Aehnlichkeits-Axen, und zu sechsen in zw\"olf Ebenen,
welche je vier der 16 Geraden enthalten; vier von den zw\"olf
Ebenen verbinden die Mittelpunkte der vier Kugeln, die
\"ubrigen acht m\"ogen {\glqq}Aehnlichkeits-Ebenen{\grqq} der vier Kugeln
genannt werden. Jede der 16 Aehnlichkeits-Axen geht durch
drei von den zw\"olf Aehnlichkeitspunkten und liegt in drei
von den 12 Ebenen. Und wie in jeder dieser 12 Ebenen
sechs von den 12 Punkten und vier von den 16 Geraden
liegen, ebenso gehen durch jeden von den 12 Punkten sechs
von den 12 Ebenen und vier von den 16 Geraden. Ueberhaupt
lehrt eine genauere Untersuchung, dass diese merkw\"urdige
Configuration von 12 Punkten, 16 Geraden und 12
Ebenen sich selbst reciprok ist.

123. Wenn vier Kugeln, deren Mittelpunkte nicht in
%-----File: 067.png-----------------------------------
einer Ebene liegen, von einer f\"unften ber\"uhrt werden, so
liegen die vier Ber\"uhrungspunkte zu zweien auf sechs Geraden,
welche durch sechs Aehnlichkeitspunkte, und zu dreien
in vier Ebenen, welche durch vier Aehnlichkeitsaxen der
vier Kugeln gehen (118., 120.); diese sechs Aehnlichkeitspunkte
und vier Axen liegen in einer Aehnlichkeits-Ebene
der Kugeln (122.). Alle Kugeln, welche vier gegebene Kugeln
ber\"uhren oder unter gleichen Winkeln schneiden, liegen in
acht Kugelb\"uscheln, deren Potenz-Ebenen die acht Aehnlichkeits-Ebenen
der vier Kugeln sind (118., 120.). Bestimmt
man vier Punkte auf den vier Kugeln so, dass der eine von
ihnen zu den drei anderen invers liegt in Bezug auf drei
von den 12 Aehnlichkeitspunkten, so liegen diese vier Punkte
auf einer zu jenen acht B\"uscheln geh\"origen Kugel; und zwar
geh\"ort diese leicht construirbare Kugel zu demjenigen von
den acht Kugelb\"uscheln, welcher die Ebene der drei Aehnlichkeitspunkte
zur Potenz-Ebene hat.

124. Bringt man diese Ebene zum Durchschnitt mit
den Ebenen der Kreise, welche die vier gegebenen Kugeln
mit der f\"unften gemein haben, so erh\"alt man die Axen der
vier Kreisb\"uschel, in welchen die vier gegebenen Kugeln den
einen der acht Kugelb\"uschel schneiden\footnote{) %
  Construirt man bez\"uglich der vier Kugeln die Polar-Ebenen
  ihres Potenzpunktes, so gehen auch diese Ebenen durch die Axen der
  vier Kreisb\"uschel; denn die Orthogonalkugel der vier gegebenen Kugeln
  geh\"ort zu jedem der acht Kugelb\"uschel.}).
Die vier Kugeln
werden im Allgemeinen und h\"ochstens von zwei Kugeln des
Kugelb\"uschels ber\"uhrt, und zwar in denjenigen leicht construirbaren
Punkten, deren Ber\"uhrungs-Ebenen durch die
Axen der vier Kreisb\"uschel gehen. Sonach giebt es im Allgemeinen
und h\"ochstens sechzehn Kugeln, welche vier gegebene
Kugeln ber\"uhren; dieselben haben paarweise die acht
Aehnlichkeits-Ebenen der vier Kugeln zu Potenz-Ebenen. --- F\"unf
gegebene Kugeln werden im Allgemeinen und h\"ochstens
von sechzehn Kugeln unter gleichen Winkeln geschnitten;
in jeder dieser sechzehn Kugeln durchdringen sich vier leicht
angebbare Kugelgeb\"usche, in welchen die erste der f\"unf gegebenen
Kugeln den vier \"ubrigen zugeordnet ist.

\begin{center}
\makebox[15em]{\hrulefill}
\end{center}
%-----File: 068.png-----------------------------------

\abschnitt{\S.~14.\\[\parskip]
Ber\"uhrung und Schnitt von Kreisen auf einer Kugelfl\"ache.}\label{p14}


\hspace{\parindent}%
125. Zwei sich nicht ber\"uhrende Kreise $k$, $k_1$ einer
Kugel $\gamma$ k\"onnen durch zwei Kegelfl\"achen verbunden werden
(27.). Die Mittelpunkte dieser beiden Kegelfl\"achen sind die
Aehnlichkeitspunkte der beiden Kugeln, welche in $k$ und $k_1$
rechtwinklig von $\gamma$ geschnitten werden (116.); sie sind conjugirt
in Bezug auf $\gamma$ und liegen auf der Polare der Geraden,
in welcher die Ebenen von $k$ und $k_1$ sich schneiden (71.).
Wir wollen sie die {\glqq}Kegel-Centra{\grqq} der Kreise $k$, $k_1$ nennen.
Da sie mit den Polen der beiden Kreis-Ebenen bez\"uglich der
Kugel $\gamma$ in einer Geraden liegen, so k\"onnen sie auch folgendermassen
construirt werden. Man bringe die Kreise $k$, $k_1$ mit
einer durch ihre beiden Pole gehenden Ebene zum Durchschnitt
und verbinde die vier Schnittpunkte; dann liegen
zwei von den sechs Verbindungslinien in den Ebenen von $k$
und $k_1$, und die \"ubrigen vier schneiden sich paarweise in den
Kegelcentren von $k$ und $k_1$. Nun liegt aber jeder zu $k$ und $k_1$
rechtwinklige Kreis der Kugel $\gamma$ mit jenen beiden Polen in
einer Ebene (60., 72.). Von den Verbindungslinien der vier
Punkte, welche die Kreise $k$ und $k_1$ mit irgend einem sie
rechtwinklig schneidenden Kreise gemein haben, gehen folglich
je zwei durch die beiden Kegelcentra von $k$ und $k_1$. --- Wenn
$k$ und $k_1$ sich ber\"uhren, so f\"allt das eine ihrer Kegelcentren
mit dem Ber\"uhrungspunkte zusammen, und die zugeh\"orige
Kegelfl\"ache zerf\"allt in die Ebenen von $k$ und $k_1$.

126. Eine Ebene, welche durch ein Kegelcentrum der
beiden Kreise $k$, $k_1$ geht und einen derselben ber\"uhrt, ber\"uhrt
auch den anderen. Jeder die Kreise $k$ und $k_1$ ber\"uhrende
Kreis liegt mit einem Kegelcentrum von $k$ und $k_1$ in einer
Ebene; die Verbindungslinie seiner beiden Ber\"uhrungspunkte
geht durch dieses Centrum (27.). Jedes Kegelcentrum von
$k$ und $k_1$ ist das Centrum reciproker Radien, durch welche
diese beiden Kreise sich in einander verwandeln; doch darf
jenes Centrum kein gemeinsamer Ber\"uhrungspunkt von $k$
und $k_1$ sein. Die durch $k$ und $k_1$ gehende Kugel $\gamma$ und jeder
Kreis derselben, welcher mit dem Kegelcentrum in einer Ebene
liegt, wird durch die reciproken Radien in sich selbst verwandelt.
Da nun die Winkel durch diese Transformation sich
%-----File: 069.png-----------------------------------
nicht \"andern, so ergiebt sich: Zwei Kreise $k$, $k_1$ einer Kugel $\gamma$
werden von denjenigen Kugelkreisen, deren Ebenen durch
die Kegelcentra von $k$ und $k_1$ gehen, unter gleichen Winkeln
geschnitten.

127. Wenn auf einer Kugel $\gamma$ zwei Kreise $k$, $k_1$ von
einem dritten $l$ unter gleichen Winkeln geschnitten werden,
so gehen durch eines oder jedes der beiden Kegelcentra
von $k$ und $k_1$ zwei von den Verbindungslinien der vier Schnittpunkte;
und zwar durch jedes, wenn die Winkel rechte sind.
Transformirt man n\"amlich den Kreis $k$ durch zweierlei reciproke
Radien, in deren Centren sich je zwei jener Verbindungslinien
schneiden und welche den Kreis $l$ in sich selbst
verwandeln, so erh\"alt man auf der Kugel $\gamma$ zwei Kreise $k'$
und $k''$, welche von $l$ in denselben Punkten und unter denselben
Winkeln geschnitten werden wie $k_1$. Es muss deshalb
einer, oder, wenn die Winkel rechte sind, jeder dieser
beiden Kreise mit $k_1$ identisch sein, woraus der Satz folgt (vgl.\
118.). --- Alle Kreise einer Kugel $\gamma$, welche zwei auf $\gamma$ liegende
Kreise $k$, $k_1$ unter gleichen Winkeln schneiden oder ber\"uhren,
liegen demnach in zwei Kreisb\"undeln, deren Centra die beiden
Kegelcentra von $k$ und $k_1$ sind.

128. Die sechs Kegelcentra, welche drei Kreise einer
Kugel $\gamma$ paarweise bestimmen, liegen zu dreien in vier Geraden
und bilden die sechs Eckpunkte eines vollst\"andigen Vierseits;
denn sie sind die Aehnlichkeitspunkte der drei Kugeln, welche
in den drei Kreisen rechtwinklig von $\gamma$ geschnitten werden
(125., vgl.\ 119.). Die Ebene des Vierseits hat in Bezug
auf $\gamma$ den Schnittpunkt der drei Kreis-Ebenen zum Pol, weil
die Pole von je zwei dieser Ebenen mit zwei von den sechs
Kegelcentren in einer Geraden liegen (125.). Die vier Geraden,
welche je drei der sechs Kegelcentra enthalten, nennen wir
die {\glqq}Kegel-Axen{\grqq} der drei Kreise; wenn die Ebenen der drei
Kreise sich in einer Geraden schneiden, so fallen ihre vier
Kegelaxen zusammen mit der Polare dieser Geraden.

129. Alle Kreise einer Kugel $\gamma$, welche drei beliebig
auf $\gamma$ angenommene Kreise unter gleichen Winkeln schneiden
oder ber\"uhren, liegen in vier Kreisb\"uscheln, deren Axen mit
den vier Kegel-Axen der drei Kreise zusammenfallen (127.).
Ein Kreis von $\gamma$, dessen Ebene durch eine dieser vier Kegel-Axen
geht, schneidet entweder keinen der drei Kreise oder
%-----File: 070.png-----------------------------------
schneidet sie alle unter gleichen Winkeln (126.). Jede Ebene,
welche durch eine der vier Kegel-Axen geht und einen der
gegebenen Kreise ber\"uhrt, muss sie alle drei ber\"uhren. Auf
einer Kugel $\gamma$ giebt es demnach im Allgemeinen und h\"ochstens
acht Kreise, welche drei auf $\gamma$ gegebene Kreise ber\"uhren;
die Construction derselben liegt auf der Hand. --- Ebenso
giebt es im Allgemeinen und h\"ochstens acht Kreise,
welche vier beliebig auf $\gamma$ angenommene Kreise unter gleichen
Winkeln schneiden; die Ebenen derselben sind die acht
Aehnlichkeits-Ebenen der vier Kugeln, welche in den vier Kreisen
rechtwinklig von $\gamma$ geschnitten werden (vgl.\ 122.). Der Beweis
ergiebt sich leicht aus dem Vorhergehenden. --- Die
Construction aller Kreise, welche drei in der Ebene gegebene
Kreise ber\"uhren oder vier Kreise der Ebene unter gleichen
Winkeln schneiden, kann durch reciproke Radien auf die
vorhergehenden Constructionen zur\"uckgef\"uhrt werden.

\begin{center}
\makebox[15em]{\hrulefill}
\end{center}


\abschnitt{\S.~15. \\[\parskip]
Die Dupin'sche Cyclide.}\label{p15}


\hspace{\parindent}%
130. Eine einfach unendliche Schaar von Kugeln, welche
durch stetige Bewegung einer ver\"anderlichen Kugel beschrieben
ist, wird im Allgemeinen von einer Fl\"ache $\Phi$ eingeh\"ullt,
die eine Schaar von kreisf\"ormigen Kr\"ummungslinien besitzt.
N\"amlich jede Kugel der Schaar wird von $\Phi$ l\"angs der Kreislinie
ber\"uhrt, welche sie mit der unmittelbar benachbarten
Kugel der Schaar gemein hat; und weil die Normalen, welche
in den Punkten dieser Linie auf $\Phi$ errichtet werden k\"onnen,
sich im Centrum der Kugel schneiden, so ist die Kreislinie
eine Kr\"ummungslinie\footnote{) %
  Jede Kr\"ummungslinie einer Fl\"ache hat die charakteristische
  Eigenschaft, dass die in ihren Punkten auf der Fl\"ache errichteten Normalen
  eine abwickelbare Fl\"ache bilden, dass also jede dieser Normalen
  die ihr unmittelbar benachbarte schneidet.})
von $\Phi$. Wird die Fl\"ache $\Phi$ durch
reciproke Radien in eine andere $\Phi_1$ transformirt, so gehen
jene Kr\"ummungslinien \"uber in kreisf\"ormige Kr\"ummungslinien
von $\Phi_1$; denn $\Phi_1$ umh\"ullt diejenige Schaar von Kugeln, in
welche die von $\Phi$ eingeh\"ullte Schaar sich verwandelt. Deshalb
besitzen insbesondere diejenigen Fl\"achen, welche durch
reciproke Radien in Rotationsfl\"achen verwandelt werden
%-----File: 071.png---------------------------------
k\"onnen, ebenso wie die letzteren eine Schaar von kreisf\"ormigen
Kr\"ummungslinien.

131. Eine der merkw\"urdigsten unter diesen Fl\"achen
ist die von \so{Dupin} entdeckte {\glqq}Cyclide{\grqq}. Dieselbe wird von
einer ver\"anderlichen Kugel $\gamma$ umh\"ullt, welche bei ihrer stetigen
Bewegung drei gegebene Kugeln $\varkappa$, $\varkappa_1$, $\varkappa_2$ fortw\"ahrend ber\"uhrt.
Die Central-Ebene dieser drei Kugeln ist eine Sym\-metrie-Ebene
der Cyclide, weil zu ihr die Kugeln symmetrisch liegen.
Wenn die drei Kugeln, welche \"ubrigens nicht in einem
Kugelb\"uschel liegen d\"urfen, eine gemeinschaftliche Centrale
haben, so wird die Cyclide von einer um diese Centrale
rotirenden Kugel $\gamma$ umh\"ullt, und ist eine Rotations-Cyclide,
deren Rotations-Axe die Centrale ist. Die Cyclide wird zu
einem geraden Kegel oder Cylinder, wenn die gegebenen drei
Kugeln $\varkappa$, $\varkappa_1$, $\varkappa_2$ in Ebenen ausarten.

132. Eine Dupin'sche Cyclide verwandelt sich durch
reciproke Radien allemal in eine Dupin'sche Cyclide; denn
die ver\"anderliche Kugel $\gamma$, welche die drei Kugeln $\varkappa$, $\varkappa_1$, $\varkappa_2$
fortw\"ahrend ber\"uhrt und die Cyclide umh\"ullt, verwandelt
sich in eine ver\"anderliche Kugel $\gamma'$, welche die zugeordneten
drei Kugeln $\varkappa$, $\varkappa_1$, $\varkappa_2$ best\"andig ber\"uhrt, und folglich auch
eine Dupin'sche Cyclide, die zugeordnete n\"amlich, umh\"ullt.
Nun haben die drei Kugeln $\varkappa$, $\varkappa_1$, $\varkappa_2$ entweder einen gemeinschaftlichen
Orthogonalkreis $k$ oder sie schneiden sich in
mindestens einem Punkte $M$ (47., 48.). Im ersteren Falle
verwandeln sie sich durch reciproke Radien, deren Centrum
beliebig auf $k$ angenommen wird, in drei andere Kugeln,
deren Mittelpunkte in einer Geraden liegen; die von der ver\"anderlichen
Kugel $\gamma$ beschriebene Cyclide verwandelt sich
folglich in eine Rotations-Cyclide (131.). Im zweiten Falle
werden die Kugeln $\varkappa$, $\varkappa_1$, $\varkappa_2$ durch reciproke Radien vom
Centrum $M$ in drei Ebenen transformirt, und die Cyclide verwandelt
sich in einen geraden Kegel oder Cylinder. Letzterer
kann als ein Specialfall der Rotations-Cyclide aufgefasst werden,
weil er von einer um die Axe rotirenden Ebene umh\"ullt wird.

133. Jede Dupin'sche Cyclide kann also durch reciproke
Radien, deren Centrum passend gew\"ahlt wird, in eine Rotations-Cyclide
verwandelt werden; sie hat deshalb folgende,
f\"ur die Rotations-Cyclide evidente Eigenschaften. Die Dupin'sche
Cyclide wird von zwei verschiedenen Kugelschaaren
%-----File: 072.png-----------------------------------
umh\"ullt und besitzt zwei Schaaren kreisf\"ormiger Kr\"ummungslinien,
in welchen sie von den Kugeln der beiden Kugelschaaren
ber\"uhrt wird. Jede Kugel der einen Schaar ber\"uhrt
alle Kugeln der anderen Schaar in den Punkten einer kreisf\"ormigen
Kr\"ummungslinie. Mindestens eine der beiden Kugelschaaren
hat einen Orthogonalkreis, welcher alle ihre Kugeln
rechtwinklig schneidet; derselbe entspricht der Axe der
Rotations-Cyclide. Die Cyclide wird durch reciproke Radien,
deren Centrum irgendwo auf diesem Orthogonalkreise angenommen
wird, allemal in eine Rotations-Cyclide verwandelt.
Zwei Kr\"ummungslinien der Cyclide k\"onnen durch eine Kugel
verbunden werden, wenn sie zu derselben Schaar geh\"oren;
im anderen Falle schneiden sie sich in einem Punkte rechtwinklig.
In jedem Punkte der Cyclide schneiden sich zwei
Kr\"ummungslinien der beiden Schaaren rechtwinklig. Jede durch
eine Kr\"ummungslinie gehende Kugel oder Ebene hat mit der
Cyclide noch eine zweite Kr\"ummungslinie von derselben Schaar
gemein; dieselbe f\"allt nur dann mit der ersteren zusammen,
wenn die Cyclide von der Kugel ber\"uhrt wird.

134. Wie die Rotations-Cyclide so hat auch jede andere
Dupin'sche Cyclide entweder keinen Doppelpunkt, oder zwei
{\glqq}Knotenpunkte{\grqq}, in welchen alle Kr\"ummungslinien der einen
Schaar sich schneiden, oder einen {\glqq}Cuspidalpunkt{\grqq}, in welchem
dieselben sich ber\"uhren. Von diesen drei Hauptarten der
Cyclide erh\"alt man wesentlich verschiedene Formen, wenn
man die zugeh\"orige Rotations-Cyclide transformirt durch reciproke
Radien, deren Centrum einmal ausserhalb, einmal
auf und einmal innerhalb der Rotations-Cyclide angenommen
wird. Die zweite und dritte Hauptart k\"onnen durch reciproke
Radien auf einem geraden Kegel oder Cylinder conform abgebildet
werden (132.). In jedem Knotenpunkte der zweiten
Hauptart werden die durch ihn gehenden Kr\"ummungslinien
der Cyclide von den Strahlen eines Rotations-Kegels ber\"uhrt.
Wenn eine Cyclide sich in das Unendliche erstreckt, was
nach dem Vorhergehenden bei jeder der drei Hauptarten eintreten
kann, so besitzt sie zwei gerade Kr\"ummungslinien, die
sich rechtwinklig kreuzen; durch dieselben gehen die Ebenen
aller \"ubrigen Kr\"ummungslinien (133.).

135. Die Kr\"ummungslinien einer Rotations-Cyclide heissen
Meridiane oder Parallelkreise, jenachdem ihre Ebenen durch
%-----File: 073.png-----------------------------------
die Rotations-Axe gehen oder auf ihr senkrecht stehen. Die
Meridiane haben in jedem Punkte der Rotations-Axe gleiche
Potenz, liegen also in einem Kugelb\"undel, dessen Axe die
Rotations-Axe ist; die Parallelkreise dagegen geh\"oren zu
demjenigen B\"undel, dessen Kugeln von der Rotations-Axe
rechtwinklig geschnitten werden. Jeder dieser beiden B\"undel
geht durch die Orthogonalkugeln des anderen; denn die Orthogonalkugeln
des zweiten B\"undels reduciren sich auf die Ebenen
der Meridiane, und diejenigen des ersteren gehen durch die
Parallelkreise, indem sie alle Meridiane rechtwinklig schneiden.
Durch einen beliebigen Punkt gehen zwei Kugeln,
welche die Rotations-Cyclide in Kreisen ber\"uhren; diese Kreise
sind zwei Meridiane, wenn der Punkt von der Cyclide einfach
eingeschlossen ist, zwei Parallelkreise, wenn er garnicht oder
zweifach von ihr umschlossen wird, dagegen ein Meridian
und ein Parallelkreis, wenn er auf der Cyclide liegt. Aus
diesen S\"atzen ergeben sich die folgenden (vgl.\ 133. und 54.).

136. Alle Kr\"ummungslinien der Dupin'schen Cyclide,
welche zu der einen oder der anderen Schaar geh\"oren, und
alle durch sie gehenden Kugeln liegen in einem Kugelb\"undel;
die Ebenen dieser Kr\"ummungslinien schneiden sich folglich
in der Axe dieses B\"undels. Mit den Kugeln des B\"undels
hat die Cyclide im Allgemeinen je zwei Kr\"ummungslinien
der Schaar gemein (vgl.\ 133.); denn wenn eine dieser Kugeln
durch einen Punkt $P$ der Cyclide geht, so enth\"alt sie auch
den durch $P$ gehenden Kreis des B\"undels (44.). Die Orthogonalkugeln
des B\"undels liegen mit den Kr\"ummungslinien
der zweiten Schaar in einem zweiten Kugelb\"undel, dessen
Orthogonalkugeln wiederum in dem ersten B\"undel liegen.
Die Central-Ebene eines jeden der beiden B\"undel steht auf
der Axe desselben normal und geht durch die Axe des anderen
B\"undels; denn sie geh\"ort zu den Orthogonalkugeln des ersteren
B\"undels; sie ist eine Symmetrie-Ebene dieses B\"undels und
folglich auch der Cyclide.

137. Die Dupin'sche Cyclide hat demnach zwei zu einander
normale Symmetrie-Ebenen (vgl.\ 131.). Jede derselben
schneidet eine der beiden Schaaren von Kr\"ummungslinien
und deren Potenz-Axe rechtwinklig, und geht durch zwei
Kr\"ummungslinien und die Potenz-Axe der anderen Schaar.
Durch eine der beiden Potenz-Axen gehen zwei singul\"are
%-----File: 074.png-----------------------------------
Ber\"uhrungs-Ebe\-nen, welche die Cyclide l\"angs zwei Kreisen ber\"uhren
(135.); wenn aber die Cyclide sich in das Unendliche
erstreckt (vgl.\ 134.), so wird sie in jeder der beiden Potenz-Axen
von einer singul\"aren Ebene ber\"uhrt. Verbindet man
die Kr\"ummungslinien der einen oder der anderen Schaar mit
einem beliebigen Punkte $P$ durch Kugelfl\"achen, so schneiden
sich diese in einem Kreise des zugeh\"origen Kugelb\"undels (44.);
die beiden durch $P$ gehenden Kreise der zwei Kugelb\"undel
aber schneiden sich rechtwinklig in $P$, weil jeder von ihnen
auf einer Orthogonalkugel des anderen liegt.

138. Jede der beiden Kugelschaaren, welche eine Dupin'sche
Cyclide umh\"ullen, kann durch eine ver\"anderliche Kugel $\gamma$
beschrieben werden, die bei ihrer stetigen Bewegung drei
beliebige Kugeln $\varkappa$, $\varkappa_1$, $\varkappa_2$ der anderen Schaar fortw\"ahrend
ber\"uhrt. Bei dieser Bewegung aber gehen die Verbindungslinien
der drei Ber\"uhrungspunkte best\"andig durch drei Aehnlichkeitspunkte
der Kugeln $\varkappa$, $\varkappa_1$, $\varkappa_2$ und ihre Ebene geht
durch eine Aehnlichkeits-Axe derselben (120.). Die drei Ber\"uhrungspunkte
liegen auf der kreisf\"ormigen Kr\"ummungslinie,
in welcher die bewegliche Kugel $\gamma$ die Cyclide ber\"uhrt (133.);
jene Aehnlichkeits-Axe der Kugeln $\varkappa$, $\varkappa_1$, $\varkappa_2$ ist demnach die
Potenz-Axe der von $\gamma$ beschriebenen Kugelschaar (136.). Die
Ber\"uhrungspunkte beschreiben auf $\varkappa$, $\varkappa_1$ und $\varkappa_2$ drei Kr\"ummungslinien
der anderen Schaar; der Schnittpunkt ihrer drei
Ber\"uhrungsebenen ist Potenzpunkt von $\varkappa$, $\varkappa_1$, $\varkappa_2$ und $\gamma$, und
beschreibt, indem $\gamma$ sich bewegt, die Potenz-Axe von $\varkappa$, $\varkappa_1$
und $\varkappa_2$. Construirt man also bez\"uglich irgend einer Kugel $\gamma$
(oder $\varkappa$) der einen Schaar die Polare der Potenz-Axe dieser
Schaar, so liegt diese Polare mit dem Kreise, in welchem
die Cyclide von der Kugel ber\"uhrt wird, und mit der Potenz-Axe
der anderen Schaar in einer Ebene.

139. Die Potenz-Axen der beiden eine Dupin'sche Cyclide
umh\"ullenden Kugelschaaren haben demnach folgende Eigenschaften.
Sie sind conjugirt bez\"uglich aller Kugeln der beiden
Schaaren und kreuzen sich rechtwinklig (137.). Jede von
ihnen ist die Potenz-Axe von je drei Kugeln der einen Schaar
und zugleich Aehnlichkeits-Axe von je drei Kugeln der anderen.
Durch jede der beiden Potenz-Axen gehen die Ebenen aller
zu einer Schaar geh\"origen Kr\"ummungslinien; auf ihr liegen
die Mittelpunkte aller Kegelfl\"achen, welche die Cyclide
%-----File: 075.png-----------------------------------
in je einer Kr\"ummungslinie der anderen Schaar ber\"uhren
oder in je zwei solchen schneiden.

140. Um eine Kugel $\gamma$ zu construiren, welche drei gegebene
Kugeln $\varkappa$, $\varkappa_1$, $\varkappa_2$ ber\"uhrt, suchen wir zun\"achst die
Potenz-Axe und die vier Aehnlichkeits-Axen der drei Kugeln.
Sodann bestimmen wir von einer dieser Aehnlichkeits-Axen
die zu der Potenz-Axe parallelen Polaren in Bezug auf $\varkappa$, $\varkappa_1$
und $\varkappa_2$, verbinden diese Polaren mit der Potenz-Axe durch
drei Ebenen und bringen letztere mit resp.\ $\varkappa$, $\varkappa_1$ und $\varkappa_2$ zum
Durchschnitt. Sind die drei Schnittkreise reell, so geht durch
jeden Punkt derselben eine die Kugeln $\varkappa$, $\varkappa_1$ und $\varkappa_2$ ber\"uhrende
Kugel $\gamma$; und zwar liegen die drei Ber\"uhrungspunkte von $\gamma$
auf jenen drei Kreisen, ihre Ebene geht durch die Aehnlichkeits-Axe
und ihre Verbindungslinien gehen durch die drei
auf derselben liegenden Aehnlichkeits-Punkte von $\varkappa$, $\varkappa_1$ und $\varkappa_2$.
Die drei Ber\"uhrungspunkte und damit zugleich die ber\"uhrende
Kugel $\gamma$ sind hiernach leicht zu construiren.

141. Die Construction eines Kreises, welcher drei in
einer Ebene gegebene Kreise $k$, $k_1$, $k_2$ ber\"uhrt, wird auf die
vorhergehende zur\"uckgef\"uhrt, indem man die Kreise als gr\"osste
Kreise von drei Kugeln auf\/fasst. Man construire also bez\"uglich
der drei Kreise die Pole von einer ihrer vier Aehnlichkeits-Axen,
verbinde diese Pole mit dem Potenzpunkt von
$k$, $k_1$ und $k_2$, und bringe die drei Verbindungslinien mit den
resp.\ drei Kreisen zum Durchschnitt. Die Schnittpunkte,
wenn solche existiren, k\"onnen zu dreien durch zwei Kreise
verbunden werden, welche in ihnen die gegebenen drei Kreise
ber\"uhren. Es giebt im Allgemeinen und h\"ochstens acht
Kreise, welche drei in der Ebene beliebig angenommene Kreise
ber\"uhren.

142. Es giebt im Allgemeinen und h\"ochstens sechzehn
Kugeln, welche vier gegebene Kugeln $\varkappa$, $\varkappa_1$, $\varkappa_2$, $\varkappa_3$ ber\"uhren
(vgl.\ 124.). Um zwei derselben zu construiren, suche man
bez\"uglich der vier Kugeln die Pole von einer ihrer acht
Aehnlichkeits-Ebenen, verbinde diese vier Pole mit dem Potenzpunkte
der Kugeln $\varkappa$, $\varkappa_1$, $\varkappa_2$, $\varkappa_3$ und bringe die vier
Verbindungslinien mit den resp.\ vier Kugeln zum Durchschnitt.
Wenn Schnittpunkte existiren, so k\"onnen dieselben zu vieren
durch zwei Kugeln verbunden werden, welche in ihnen die
vier gegebenen Kugeln ber\"uhren. Der Beweis dieser
%-----File: 076.png-----------------------------------
Construction bleibe als n\"utzliche Uebung dem Leser \"uberlassen
(vgl.\ 124., 140.).

143. Im Allgemeinen giebt es vier Dupin'sche Cycliden,
welche drei beliebig angenommene Kugeln $\varkappa$, $\varkappa_1$, $\varkappa_2$ einh\"ullen
(140.); die vier Aehnlichkeits-Axen dieser Kugeln sind die
zweiten Potenz-Axen der vier Cycliden. Doch kann je nach
der Lage der drei Kugeln auch der Fall eintreten, dass weniger
als vier oder auch gar keine Schaaren sie ber\"uhrender Kugeln
existiren. Wenn z.~B.\ eine der drei Kugeln die zweite ein- und
die dritte ausschliesst, so giebt es keine Kugel, welche
sie alle drei ber\"uhrt.

144. Alle Kugeln eines Kugelb\"undels, welche eine beliebige,
nicht zu dem B\"undel geh\"orige Kugel $\varkappa$ ber\"uhren,
umh\"ullen eine Dupin'sche Cyclide. Denn sie werden nicht
blos von $\varkappa$, sondern von unendlich vielen Kugeln $\varkappa_1$, $\varkappa_2$, $\ldots$
ber\"uhrt, welche eine zweite die Cyclide einh\"ullende Kugelschaar
bilden, und zwar erh\"alt man eine dieser Kugeln $\varkappa_1,
\varkappa_2, \ldots$, wenn man durch den Kugelb\"undel ein Geb\"usch legt
und durch die zu dem Geb\"usche geh\"origen reciproken Radien
die Kugel $\varkappa$ transformirt (111.). Die Kugeln $\varkappa_1$, $\varkappa_2$, $\ldots$ der
zweiten Schaar sind der Kugel $\varkappa$ zugeordnet in Bezug auf
die Orthogonalkugeln des B\"undels.

\begin{center}
\makebox[15em]{\hrulefill}
\end{center}


\abschnitt{\S.~16.\\[\parskip]
Lineare Kugelsysteme, die zu einander normal sind.}\label{p16}


\hspace{\parindent}%
145. Ein Kugelb\"undel und der zu ihm geh\"orige B\"uschel
orthogonaler Kugeln stehen in den folgenden Wechselbeziehungen
zu einander. Alle Orthogonalkugeln des B\"undels
bilden den B\"uschel und alle Orthogonalkugeln des B\"uschels
bilden den B\"undel (50.).  Jede Kugel von einem dieser
beiden linearen Kugelsysteme ist die Orthogonalkugel eines
durch das andere gehenden Geb\"usches; und jedes Geb\"usch,
welches durch eines der beiden Systeme geht, hat eine in
dem anderen liegende Orthogonalkugel. Mit anderen Worten:
Wenn ein Kugelb\"uschel oder -B\"undel durch die Orthogonalkugel
eines Geb\"usches geht, so geht das letztere durch alle
Orthogonalkugeln des ersteren; und umgekehrt. Weil aber
ein B\"undel der Schnitt von zwei Geb\"uschen ist, so ergiebt
sich weiter: Wenn von zwei Kugelb\"undeln der eine durch
%-----File: 077.png---------------------------------
zwei und folglich durch alle Orthogonalkugeln des anderen geht,
so geht der letztere durch alle Orthogonalkugeln des ersteren.
Von zwei Kugelgeb\"uschen geht entweder keines oder jedes
durch die Orthogonalkugel des anderen; wenn n\"amlich das
eine durch die Orthogonalkugel des anderen geht, so geht
dieses durch alle Orthogonalkugeln eines seine Orthogonalkugel
enthaltenden B\"undels des ersteren Geb\"usches und folglich
auch durch die Orthogonalkugel dieses Geb\"usches.

146. Wir k\"onnen die vorhergehenden S\"atze in dem
folgenden Satze zusammenfassen: Von zwei linearen Kugelsystemen
geht entweder keines oder jedes durch alle Orthogonalkugeln
des anderen. In dem letzteren Falle, wenn
also das eine und folglich jedes der beiden Systeme alle
Orthogonalkugeln des anderen enth\"alt, wollen wir diese
linearen Kugelsysteme {\glqq}zu einander normal{\grqq} nennen. Zu
einem Kugelgeb\"usche sind demnach normal alle durch seine
Orthogonalkugel gehenden Kugelb\"uschel, B\"undel und Geb\"usche;
durch jede andere Kugel geht ein bestimmter, zu
dem Geb\"usche normaler Kugelb\"uschel, und durch jeden
die Orthogonalkugel nicht enthaltenden B\"uschel kann allemal
ein zu dem Geb\"usche normaler Kugelb\"undel gelegt
werden. Ein Kugelb\"undel ist zu unendlich vielen anderen
Kugelb\"undeln normal; dieselben durchdringen sich in den
Orthogonalkugeln jenes B\"undels, und durch jede andere Kugel
des Raumes geht einer von ihnen. Zu einem Kugelb\"uschel
sind unendlich viele Geb\"usche normal; dieselben durchdringen
sich in den Orthogonalkugeln des B\"uschels, und durch jede
andere Kugel geht eines von ihnen.

147. Von zwei zu einander normalen Kugelb\"undeln gehen
durch einen beliebigen Punkt des Raumes zwei sich rechtwinklig
schneidende Kreise; durch jeden dieser beiden Kreise
geht n\"amlich eine Orthogonalkugel des anderen (44., 146.).
Die Central-Ebene des einen B\"undels geh\"ort zu den Orthogonalkugeln
desselben; sie ist folglich eine Ebene des anderen
B\"undels und geht durch dessen Axe, w\"ahrend sie zu der
Axe des ersteren B\"undels normal ist. Die beiden Axen der
zu einander normalen B\"undel kreuzen sich demnach rechtwinklig,
und ihre Central-Ebenen schneiden sich rechtwinklig;
jede der beiden Axen liegt in einer der beiden Central-Ebenen
und ist zu der anderen normal. Schneiden sich die Kugeln
%-----File: 078.png---------------------------------
des einen B\"undels in zwei Punkten, die (48.) auch zusammenfallen
k\"onnen, so hat der andere B\"undel einen durch diese
Punkte gehenden Orthogonalkreis; denn auf die beiden Punkte
reduciren sich zwei Orthogonalkugeln des ersteren B\"undels,
sie sind also Punktkugeln des letzteren. Wenn anderseits
jeder der beiden B\"undel einen Orthogonalkreis hat, so sind
diese beiden Kreise zu einander orthogonal, weil jeder von
ihnen die durch den anderen gehenden Kugeln rechtwinklig
schneidet. In einem sehr speciellen Falle, den wir nicht
weiter ber\"ucksichtigen wollen, reduciren sich die Orthogonalkreise
beider B\"undel auf einen Punkt, in welchem sich die
Axen der B\"undel rechtwinklig schneiden. --- Die beiden,
eine Dupin'sche Cyclide einh\"ullenden Kugelschaaren liegen
in zwei zu einander normalen Kugelb\"undeln (136.).

148. Zwei zu einander normale lineare Kugelsysteme
verwandeln sich durch reciproke Radien allemal wieder in
zwei zu einander normale lineare Kugelsysteme (54.). Ueberhaupt
bilden ja zwei sich schneidende Kugeln dieselben
Winkel mit einander wie die beiden Kugeln oder Ebenen, in
welche sie durch reciproke Radien \"ubergehen (22.). Nimmt
man das Centrum der reciproken Radien auf dem Orthogonalkreise
des einen von zwei zu einander normalen Kugelb\"undeln
an, so verwandeln sich diese B\"undel in zwei andere
zu einander normale B\"undel, die eine besonders einfache
gegenseitige Lage haben; n\"amlich die Axe des einen derselben
enth\"alt die Mittelpunkte aller Kugeln des anderen (54.), und
durch Drehung um diese Axe \"andern sich die B\"undel nicht.
Wenn insbesondere das Centrum der reciproken Radien mit
einem Punkte zusammenf\"allt, durch welchen alle Kugeln des
einen von den normalen B\"undeln gehen, so verwandelt sich
dieser B\"undel in einen B\"undel von Strahlen und Ebenen,
der andere aber in einen B\"undel von Kugeln und Kreisen,
deren Mittelpunkte auf einem Strahle jenes Strahlenb\"undels
liegen (147., 54.), und auch in diesem Falle \"andern sich die
beiden B\"undel durch eine Drehung um diese Mittelpunktsgerade nicht.

149. Alle Kugeln eines B\"undels $B$, welche eine beliebige
Kugel $\varkappa$ unter einem gegebenen schiefen Winkel schneiden,
bilden mit jeder sie schneidenden Kugel des durch $\varkappa$
gehenden und zu $B$ normalen B\"undels $B_1$ gleiche Winkel,
%-----File: 079.png---------------------------------
und umh\"ullen im Allgemeinen eine Dupin'sche Cyclide, deren
zweite Kugelschaar in dem B\"undel $B_1$ liegt. Bei dem Beweise
dieses Satzes d\"urfen wir annehmen, dass entweder die
Axe $a$ des B\"undels $B$ durch die Mittelpunkte aller Kugeln
von $B_1$ geht, oder dass der B\"undel $B_1$ ein Strahlenb\"undel
ist und dass ein Strahl $s$ desselben die Mittelpunkte aller
Kugeln von $B$ enth\"alt; denn auf diese beiden F\"alle l\"asst
sich der allgemeine Fall durch reciproke Radien zur\"uckf\"uhren
(148., 147.). In dem ersteren Falle erh\"alt man alle
Kugeln des B\"undels $B$, welche mit $\varkappa$ den gegebenen schiefen
Winkel bilden, wenn man eine beliebige derselben um die
Axe $a$ rotiren l\"asst; jene Kugeln umh\"ullen eine Rotations-Cyclide,
und die Richtigkeit des Satzes leuchtet ohne Weiteres
ein. In dem zweiten Falle ist $\varkappa$ eine Ebene, welche den
Strahl $s$ in dem Mittelpunkte $M$ des Strahlenb\"undels $B_1$
schneidet, und man erh\"alt alle jene Kugeln des B\"undels $B$,
wenn man das Centrum von einer derselben den Strahl $s$
durchlaufen und zugleich ihren Radius proportional mit dem
Abstande des Centrums vom Punkte $M$ sich \"andern l\"asst.
Auch in diesem zweiten Falle leuchtet die Richtigkeit des
Satzes sofort ein; jene Kugeln aber umh\"ullen im Allgemeinen
einen Rotationskegel mit der Axe $s$ und dem Mittelpunkte
$M$, welche nur dann nicht reell existirt, wenn der Punkt $M$
von den Kugeln eingeschlossen wird oder auf denselben liegt.

150. Alle Kugeln eines Geb\"usches $\varGamma$, welche eine beliebige
Kugel $\varkappa$ unter einem gegebenen schiefen Winkel
schneiden, bilden mit jeder sie schneidenden Kugel des durch
$\varkappa$ gehenden und zu $\varGamma$ normalen B\"uschels gleiche Winkel, und
ber\"uhren im Allgemeinen zwei Kugeln dieses B\"uschels. Bei
dem Beweise dieses Satzes unterscheiden wir zwei F\"alle,
jenachdem n\"amlich $\varkappa$ mit der Orthogonalkugel $\omega$ des Geb\"usches
einen Punkt gemein hat oder nicht. In dem ersteren
Falle verwandeln wir $\varkappa$ und $\omega$ durch reciproke Radien in
zwei Ebenen $\varkappa'$ und $\omega'$; dann geht das Geb\"usch \"uber in
ein zu $\omega'$ symmetrisches Geb\"usch $\varGamma'$. Da nun die Radien
aller Kugeln von $\varGamma'$, welche die Ebene $\varkappa'$ unter dem gegebenen
Winkel schneiden, proportional sind den Abst\"anden
ihrer Mittelpunkte von der Geraden $\overline{\varkappa'\,\omega'}$, so bilden diese
Kugeln mit einer beliebig durch diese Schnittlinie von $\varkappa'$
und $\omega'$ gelegten Ebene gleiche Winkel, und ber\"uhren zwei
%-----File: 080.png-----------------------------------
durch $\overline{\varkappa'\;\omega'}$ gehende Ebenen, wenn sie mit $\overline{\varkappa'\;\omega'}$ keinen Punkt
gemein haben. F\"ur diesen ersten Fall ist damit der obige
Satz bewiesen. --- Wenn zweitens die Kugeln $\varkappa$ und $\omega$ keinen
Punkt mit einander gemein haben, so enth\"alt der durch sie
gehende Kugelb\"uschel zwei Punktkugeln $M$, $N$. Durch reciproke
Radien vom Centrum $M$ verwandeln sich alsdann $\varkappa$
und $\omega$ in zwei concentrische Kugeln $\varkappa'$ und $\omega'$ (54.), und
das Geb\"usch wird in ein anderes transformirt, welches den
Mittelpunkt von $\varkappa'$ und $\omega'$ zum Centrum hat. Alle Kugeln
dieses neuen Geb\"usches aber, welche $\varkappa'$ unter dem gegebenen
Winkel schneiden, haben wie man leicht einsieht gleiche
Radien, und der Ort ihrer Mittelpunkte ist eine mit $\varkappa'$ concentrische
Kugel; sie ber\"uhren folglich zwei Kugeln und
bilden gleiche Winkel mit jeder dritten sie schneidenden
Kugel des durch $\varkappa'$ und $\omega'$ gehenden B\"uschels concentrischer
Kugeln. Damit ist auch f\"ur diesen zweiten Fall, welcher
insbesondere dann eintritt, wenn $\omega$ einen imagin\"aren Halbmesser
hat, der Satz bewiesen.

151. Die Kugeln eines Geb\"usches, welche eine Kugel $\varkappa$
unter einem gegebenen Winkel schneiden, sind im Allgemeinen
identisch mit denjenigen Kugeln des Geb\"usches,
welche eine gewisse andere Kugel $\lambda$ ber\"uhren (150., vgl.\ 111.).
Die Potenz-Ebenen, welche sie mit irgend zwei dem Geb\"usche
nicht angeh\"orenden Kugeln bestimmen, umh\"ullen zwei collineare
Fl\"achen (101.); die eine dieser Fl\"achen aber f\"allt mit
$\lambda$ zusammen, wenn $\lambda$ die eine jener beiden Kugeln ist, und
die andere Fl\"ache ist folglich eine zu der Kugel $\lambda$ collineare
Fl\"ache zweiter Ordnung und zweiter Classe (94.). Insbesondere
umh\"ullen die Ebenen der Kreise, in welcher $\varkappa$ von
jenen Kugeln unter dem gegebenen Winkel geschnitten wird,
eine zu $\lambda$ collineare Fl\"ache zweiter Ordnung und zweiter
Classe. Auch die Polar-Ebenen eines beliebigen Punktes
in Bezug auf alle jene Kugeln umh\"ullen eine zu $\lambda$ collineare
Fl\"ache (102.); die Mittelpunkte der Kugeln aber liegen auf
einer zu $\lambda$ reciproken Fl\"ache zweiter Classe und zweiter
Ordnung (103.), falls das Geb\"usch kein symmetrisches ist. --- Ein
beliebiger dem Geb\"usche angeh\"orender Kugelb\"uschel
enth\"alt im Allgemeinen und h\"ochstens zwei Kugeln, welche
$\lambda$ ber\"uhren (109.) und somit die Kugel $\varkappa$ unter dem gegebenen
schiefen Winkel schneiden.

%-----File: 081.png-----------------------------------

152. Weil die Kugeln eines B\"undels, welche eine beliebige
Kugel $\varkappa$ unter einem gegebenen schiefen Winkel
schneiden, im Allgemeinen eine Dupin'sche Cyclide umh\"ullen
(149.), so wollen wir ihre Gesammtheit eine {\glqq}Dupin'sche
Kugelschaar{\grqq} nennen. Die Potenz-Ebenen, welche die Kugeln
dieser Schaar mit beliebigen, dem B\"undel nicht angeh\"orenden
Kugeln bestimmen, umh\"ullen zwei collineare Kegelfl\"achen
(100., 101.); diese Kegelfl\"achen sind von der zweiten Ordnung
und zweiten Classe, weil eine derselben ein Rotationskegel
wird, wenn die eine der beiden Kugeln alle Kugeln der
Schaar ber\"uhrt. Auch die Polar-Ebenen eines beliebigen
Punktes bez\"uglich aller Kugeln der Dupin'schen Schaar umh\"ullen
eine Kegelfl\"ache zweiter Ordnung und zweiter Classe;
die Mittelpunkte jener Kugeln aber liegen im Allgemeinen
auf einer Curve zweiter Ordnung, welche auf jene Kegelfl\"achen
reciprok bezogen ist.

\begin{center}
\makebox[15em]{\hrulefill}
\end{center}


\abschnitt{\S.~17. \\[\parskip]
Kugeln, die sich unter gegebenen Winkeln schneiden.}\label{p17}


\hspace{\parindent}%
153. Alle Kugeln, welche eine Kugel $\varkappa$ unter einem
gegebenen schiefen Winkel schneiden, bilden ein {\glqq}quadratisches
Kugelsystem dritter Stufe{\grqq}, d.~h.\ ein beliebiger Kugelb\"uschel
enth\"alt im Allgemeinen und h\"ochstens zwei derselben.
Bei dem Beweise dieses Satzes d\"urfen wir annehmen,
dass der B\"uschel entweder aus concentrischen Kugeln bestehe
oder aus Ebenen, die alle durch eine Gerade gehen; denn
durch reciproke Radien kann der allgemeine Fall auf diese
besonderen beiden F\"alle zur\"uckgef\"uhrt werden (54.). In
dem ersteren dieser F\"alle sei $M$ der Mittelpunkt der concentrischen
Kugeln, $C$ derjenige von $\varkappa$ und $P$ ein beliebiger
Punkt der Kugel $\varkappa$. Dann bildet $\varkappa$ mit der durch $P$ gehenden
Kugel des B\"uschels dieselben Winkel, wie der Radius $CP$ mit
der Geraden $MP$. Legt man also durch die Punkte $C$ und
$M$ einen Kreis, dessen \"uber dem Bogen $CM$ stehenden Peripheriewinkel
dem gegebenen Winkel $w$ gleich sind, und bestimmt
sodann die Schnittpunkte $P$, $P'$ dieses Kreises und
der Kugel $\varkappa$, so gehen durch $P$ und $P'$ die beiden einzigen
Kugeln des B\"uschels, welche mit $\varkappa$ den Winkel $w$ bilden;
man erh\"alt aber keine solche Schnittpunkte, wenn der Radius $r$
%-----File: 082.png-----------------------------------
von $\varkappa$ gr\"osser als $CM$ und $\sin w > CM : r$ ist. --- In dem
zweiten Falle legen wir durch das Centrum $C$ der Kugel $\varkappa$
eine Ebene, welche die Ebenen des B\"uschels rechtwinklig schneidet,
und bezeichnen mit $M$ den gemeinschaftlichen Punkt der
Schnittlinien, sowie mit $P$ einen der Punkte, welchen die
Ebene mit $\varkappa$ gemein hat. Die Kugel $\varkappa$ bildet dann mit der
durch $P$ gehenden Ebene des B\"uschels und mit der um $M$
mit dem Radius $MP$ beschriebenen Kugel zwei spitze Winkel,
die sich zu einem rechten erg\"anzen; diejenigen zwei
Lagen des Punktes $P$, f\"ur welche der erstere dieser Winkel
einem gegebenen Winkel gleich wird, ergeben sich deshalb
ebenso, wie im ersteren Falle.

154. Eine Kugel $\varkappa$ wird unter dem schiefen Winkel $w$
auch von unendlich vielen Ebenen geschnitten; dieselben umh\"ullen
eine mit $\varkappa$ concentrische Kugel. Transformirt man
diese Ebenen durch reciproke Radien, deren Centrum irgend
ein Punkt $C$ ist und welche die Kugel $\varkappa$ in sich selbst verwandeln,
so ergiebt sich: Alle durch einen Punkt $C$ gehenden
Kugeln, welche eine Kugel $\varkappa$ unter dem schiefen Winkel $w$
schneiden, umh\"ullen eine andere Kugel $\lambda$. Dieser Satz ist in
einem fr\"uheren (150.) enthalten; denn alle durch $C$ gehenden
Kugeln und Kreise bilden ein specielles Kugelgeb\"usch, und
der Punkt $C$ kann als eine von ihnen ber\"uhrte Punktkugel
aufgefasst werden.

155. Die Ebenen, welche zwei Kugeln $\varkappa$ und $\varkappa_1$ unter
den respectiven schiefen Winkeln $w$ und $w_1$ schneiden, umh\"ullen
im Allgemeinen zwei Rotationskegel; denn sie sind
die gemeinschaftlichen Ber\"uhrungsebenen von zwei bestimmten,
mit $\varkappa$ und $\varkappa_1$ concentrischen Kugeln (154.). Alle durch
einen Punkt $C$ gehenden Kugeln, welche mit den Kugeln $\varkappa$
und $\varkappa_1$ die resp.\ Winkel $w$ und $w_1$ bilden, umh\"ullen im Allgemeinen
zwei Dupin'sche Cycliden, von welchen $C$ ein Knotenpunkt
ist; denn durch reciproke Radien vom Centrum $C$ verwandeln
sie sich in die gemeinschaftlichen Ber\"uhrungs-Ebenen
von zwei anderen Kugeln, oder auch, wenn $C$ auf $\varkappa$ oder $\varkappa_1$
liegt, in diejenigen Ber\"uhrungs-Ebenen einer Kugel, welche
eine Ebene unter einem gegebenen schiefen Winkel schneiden.
Die beiden Dupin'schen Cycliden sind allemal reell vorhanden,
wenn $C$ auf $\varkappa$ oder $\varkappa_1$ liegt.

156. Das quadratische Kugelsystem dritter Stufe, dessen
%-----File: 083.png-----------------------------------
Kugeln mit einer Kugel $\varkappa$ einen gegebenen Winkel bilden,
hat mit einem Geb\"usche ein {\glqq}quadratisches Kugelsystem
zweiter Stufe{\grqq} und mit einem Kugelb\"undel eine Dupin'sche
Kugelschaar gemein (150., 152.). Von der Dupin'schen Kugelschaar
liegen in einem beliebigen Kugelgeb\"usch im Allgemeinen
und h\"och\-stens zwei Kugeln; denn das Geb\"usch schneidet
den die Schaar enthaltenden B\"undel in einem Kugelb\"uschel,
und dieser hat mit der Schaar dieselben Kugeln
gemein wie mit dem quadratischen Kugelsystem dritter Stufe.
Auf \"ahnliche Weise ergiebt sich, dass das quadratische Kugelsystem
zweiter Stufe mit einem beliebigen Kugelb\"undel im
Allgemeinen und h\"ochstens zwei Kugeln, mit einem Geb\"usche
aber eine Dupin'sche Kugelschaar gemein hat. Insbesondere
bilden alle durch einen Punkt $C$ gehenden Kugeln des quadratischen
Systemes zweiter Stufe eine Dupin'sche Kugelschaar,
weil sie dem Geb\"usche vom Centrum $C$ und der Potenz Null
angeh\"oren. Die Kugeln dieses quadratischen Systemes zweiter
Stufe umh\"ullen im Allgemeinen zwei Kugeln (150.).

157. Die Kugeln $\gamma$, welche zwei Kugeln $\varkappa$ und $\varkappa_1$ beziehungsweise
unter den schiefen Winkeln $w$ und $w_1$ schneiden,
bilden zwei quadratische Kugelsysteme zweiter Stufe;
die beiden sie enthaltenden Kugelgeb\"usche sind normal zu
dem durch $\varkappa$, und $\varkappa_1$ gehenden Kugelb\"uschel. Verbinden wir
n\"amlich eine jener Kugeln $\gamma$ mit dem Kugelb\"undel, von welchem
$\varkappa$ und $\varkappa_1$ zwei Orthogonalkugeln sind, durch ein Geb\"usch $\varGamma$,
so ist dieses zu dem B\"uschel $\varkappa\varkappa_1$ normal; alle Kugeln von $\varGamma$,
welche mit $\varkappa$ den Winkel w bilden, schneiden folglich $\varkappa_1$
unter demselben Winkel $w_1$, wie jene eine Kugel $\gamma$ (150.),
und bilden ein quadratisches Kugelsystem zweiter Stufe. Die
s\"ammtlichen Kugeln $\gamma$ aber bilden zwei solche Kugelsysteme
und liegen in zwei verschiedenen Kugelgeb\"uschen, weil diejenigen
unter ihnen, welche durch irgend einen Punkt von $\varkappa$
gehen, nicht blos eine, sondern zwei Dupin'sche Kugelschaaren
bilden (155., 156.).

158. Alle Kugeln, welche drei in keinem B\"uschel liegende
Kugeln $\varkappa$, $\varkappa_1$, $\varkappa_2$ unter den resp.\ schiefen Winkeln
$w$, $w_1$, $w_2$ schneiden, bilden im Allgemeinen vier Dupin'sche
Kugelschaaren und liegen in vier Kugelb\"undeln, welche zu
dem durch $\varkappa$, $\varkappa_1$, und $\varkappa_2$ gehenden B\"undel normal sind. Sie
liegen n\"amlich, weil sie $\varkappa$ und $\varkappa_1$ unter den Winkeln $w$
%-----File: 084.png-----------------------------------
und $w_1$ schneiden, in zwei zu dem B\"uschel $\varkappa\varkappa_1$ normalen Geb\"uschen
(157.), und weil sie $\varkappa$ und $\varkappa_1$ unter den Winkeln $w$
und $w_2$ schneiden, in zwei zu dem B\"uschel $\varkappa\varkappa_2$ normalen
Geb\"uschen; sie liegen folglich in den vier Kugelb\"undeln,
welche die ersteren beiden Geb\"usche mit den letzteren beiden
gemein haben. Jeder dieser vier B\"undel geht durch die gemeinschaftlichen
Orthogonalkugeln der B\"uschel $\varkappa\varkappa_1$ und $\varkappa\varkappa_2$.
und ist folglich zu dem B\"undel $\varkappa\:\varkappa_1\:\varkappa_2$
normal; alle seine
Kugeln aber, welche die Kugel $\varkappa$ unter dem Winkel $w$ schneiden,
bilden eine Dupin'sche Kugelschaar (152.) und schneiden
die Kugeln $\varkappa_1$ und $\varkappa_2$ unter dem resp.\ Winkeln $w_1$ und $w_2$. --- Uebrigens
ist es, wie wir schon f\"ur den Fall der Ber\"uhrung,
wenn $w = w_1 = w_2 = 0$ ist, hervorgehoben haben (143.),
bei besonderer Lage der Kugeln $\varkappa$, $\varkappa_1$, $\varkappa_2$ m\"oglich, dass
weniger als vier oder dass garkeine Schaaren von Kugeln
existiren, welche mit $\varkappa$, $\varkappa_1$ und $\varkappa_2$ die gegebenen Winkel
bilden.

159. Vier in keinem B\"undel liegende Kugeln $\varkappa$, $\varkappa_1$, $\varkappa_2$, $\varkappa_3$
werden im Allgemeinen und h\"ochstens von sechzehn Kugeln
unter den respectiven schiefen Winkeln $w$, $w_1$, $w_2$, $w_3$ geschnitten.
N\"amlich diese sechzehn Kugeln liegen, weil sie
$\varkappa$, $\varkappa_1$, und $\varkappa_2$ unter den Winkeln $w$, $w_1$ und $w_2$ schneiden,
in vier Dupin'schen Kugelschaaren, und zugleich, weil sie
mit $\varkappa$ und $\varkappa_4$ die Winkel $w$ und $w_4$ bilden, in zwei Kugelgeb\"uschen;
sie bilden also die acht Kugelpaare, welche diese
beiden Geb\"usche mit jenen vier Schaaren gemein haben (156.).

\begin{center}
\makebox[15em]{\hrulefill}
\end{center}


\abschnitt{\S.~18. \\[\parskip]
Kreise auf einer Kugel, die sich unter gegebenen Winkeln schneiden.}\label{p18}


\hspace{\parindent}%
160. Die Geometrie der Kreise auf einer Kugel (oder
Ebene) $\gamma$ l\"asst sich zur\"uckf\"uhren auf die Geometrie des Kugelgeb\"usches,
von welchem $\gamma$ die Orthogonalkugel ist. Insbesondere
bilden zwei sich schneidende Kreise der Kugel $\gamma$
mit einander dieselben Winkel, wie die beiden durch sie
gehenden und zu $\gamma$ rechtwinkligen Kugeln. Doch ziehen wir
es vor, die n\"achstfolgenden S\"atze direct, anstatt mit H\"ulfe
des Geb\"usches, zu beweisen.

161. Von zwei auf einer Kugel liegenden Kreisb\"undeln
geht entweder keiner oder jeder durch den Orthogonalkreis
%-----File: 085.png-----------------------------------
des anderen; denn nur dann, wenn die Centra der beiden
B\"undel conjugirt sind bez\"uglich der Kugel, tritt der letztere
Fall ein (vgl.~69.). Wenn ein Kreisb\"undel und ein Kreisb\"uschel
auf einer und derselben Kugel liegen, so geht entweder
keiner oder jeder von ihnen durch alle Orthogonalkreise
des anderen; der letztere Fall tritt ein, wenn das Centrum
des B\"undels und die Axe des B\"uschels conjugirt sind
in Bezug auf die Kugel (69., 72.). Wir wollen nun zwei Kreisb\"undel
einer Kugel oder Ebene, und ebenso einen Kreisb\"undel
und einen Kreisb\"uschel {\glqq}zu einander normal{\grqq} nennen, wenn der
eine von ihnen durch jeden Orthogonalkreis des anderen geht.

162. Zwei solche zu einander normale Kreissysteme,
m\"ogen sie nun auf einer Kugel oder in einer Ebene liegen,
verwandeln sich durch reciproke Radien allemal wieder in
zwei zu einander normale Kreissysteme. Zu einem Kreisb\"uschel
k\"onnen einfach unendlich viele normale Kreisb\"undel construirt
werden; dieselben durchdringen sich in den Orthogonalkreisen
des B\"uschels, und durch jeden anderen Kreis ihres Tr\"agers
geht einer von ihnen. Zu einem Kreisb\"undel sind doppelt
unendlich viele Kreisb\"uschel normal; dieselben haben den
Orthogonalkreis des B\"undels mit einander gemein, und durch
jeden anderen Kreis ihres Tr\"agers geht einer von ihnen.

163. Wenn ein Kreisb\"undel und ein Kreisb\"uschel zu
einander normal sind, so bilden alle Kreise des ersteren,
welche irgend einen Kreis des letzteren unter einem gegebenen
schiefen Winkel schneiden, auch mit jedem anderen
sie schneidenden Kreise des B\"uschels gleiche Winkel, und
ber\"uhren im Allgemeinen zwei Kreise des B\"uschels. Bei dem
Beweise dieses Satzes d\"urfen wir annehmen, dass entweder
der B\"uschel aus Parallelkreisen einer Kugel besteht oder ein
gew\"ohnlicher Strahlenb\"uschel ist; denn auf diese beiden F\"alle
l\"asst sich der allgemeine Fall zur\"uckf\"uhren (65.). Im ersteren
Falle liegen die Mittelpunkte der Parallelkreise mit dem Centrum
des Kreisb\"undels auf einem Durchmesser der Kugel; man
erh\"alt alle Kreise des B\"undels, welche mit einem der Parallelkreise
den gegebenen Winkel bilden, wenn man einen beliebigen
derselben um jenen Durchmesser rotiren l\"asst, und die Richtigkeit
des Satzes leuchtet ohne weiteres ein. In dem zweiten
Falle liegt der Kreisb\"undel in der Ebene des Strahlenb\"uschels
und enth\"alt alle Kreise der Ebene, deren Mittelpunkte auf
%-----File: 086.png---------------------------------
einem bestimmten Strahle dieses B\"uschels liegen; die Radien
derjenigen Kreise des B\"undels, welche mit irgend einem anderen
Strahle des B\"uschels den gegebenen Winkel bilden, sind
folglich proportional zu den Abst\"anden ihrer Mittelpunkte von
dem Mittelpunkte des B\"uschels; diese Kreise bilden deshalb
mit jedem sie schneidenden Strahle des B\"uschels gleiche
Winkel, und werden, wenn sie das Centrum des B\"uschels
nicht einschliessen, von zwei Strahlen desselben ber\"uhrt.

164. Alle Kugeln, welche eine Kugel $\varkappa$ unter dem schiefen
Winkel $w$ und eine andere Kugel $\gamma$ rechtwinklig schneiden,
bilden ein quadratisches Kugelsystem zweiter Stufe und umh\"ullen
im Allgemeinen zwei Kugeln (156.). Daraus folgt
(160.), wenn $\varkappa$ und $\gamma$ sich rechtwinklig schneiden: Alle Kreise
der Kugel $\gamma$, welche einen auf $\gamma$ angenommenen Kreis $k$
unter dem schiefen Winkel $w$ schneiden, bilden ein quadratisches
Kreissystem zweiter Stufe, d.~h.\ in einem Kreisb\"uschel
von $\gamma$ liegen im Allgemeinen und h\"ochstens zwei derselben.
Durch eine Drehung um den Durchmesser von $\gamma$, welcher zu der
Ebene des Kreises $k$ normal ist, \"andert dieses quadratische
Kreissystem sich nicht. Die Ebenen aller Kreise dieses
Systemes umh\"ullen eine Fl\"ache zweiter Ordnung und zweiter
Classe (151.); dieselbe ist eine Rotationsfl\"ache und hat den
eben erw\"ahnten Durchmesser zur Rotationsaxe.

165. Alle Kugeln, welche zwei Kugeln $\varkappa$, $\varkappa_1$ unter den
resp.\ schiefen Winkeln $w$, $w_1$ und eine dritte Kugel $\gamma$ rechtwinklig
schneiden, liegen in zwei Kugelb\"undeln und bilden
zwei Dupin'sche Kugelschaaren (157., 156.). Alle Kreise
der Kugel $\gamma$, welche zwei auf $\gamma$ angenommene Kreise $k$, $k_1$
unter den resp.\ Winkeln $w$, $w_1$ schneiden, liegen folglich in
zwei Kreisb\"undeln und bilden zwei quadratische Schaaren
von Kreisen. Jede dieser beiden Schaaren hat mit einem
beliebigen Kreisb\"undel von $\gamma$ im Allgemeinen und h\"ochstens
zwei Kreise gemein (156.), ihre Kreise bilden mit jedem
sie schneidenden Kreise des durch $k$ und $k_1$ gehenden B\"uschels
gleiche Winkel und ber\"uhren im Allgemeinen zwei
Kreise dieses B\"uschels (163.).

166. Drei beliebige Kreise $k$, $k_1$, $k_2$ einer Kugel $\gamma$ werden
im Allgemeinen und h\"ochstens von acht Kreisen der Kugel
unter den respectiven schiefen Winkeln $w$, $w_1$, $w_2$ geschnitten.
Diese acht Kreise liegen, weil sie mit $k$ und $k_1$ die
%-----File: 087.png-----------------------------------
Winkel $w$ und $w_1$ bilden, in zwei quadratischen Kreisschaaren,
zugleich aber, weil sie $k$ und $k_2$ unter den resp.\ Winkeln
$w$ und $w_2$ schneiden, in zwei Kreisb\"undeln (165.);
sie bilden also die vier Kreispaare, welche diese beiden
B\"undel mit jenen beiden Kreisschaaren gemein haben.

\begin{center}
\makebox[15em]{\hrulefill}
\end{center}
\newpage

\section*{\centering Einleitung in die analytische Geometrie der Kugelsysteme.}
%\abschnitt{\large \so{Einleitung in die analytische Geometrie der Kugelsysteme.}}

\abschnitt{\S.~19.\\[\parskip]
Kugelcoordinaten. Complexe, Congruenzen und Schaaren von Kugeln.}\label{p19}


\hspace{\parindent}%
167. Wir wollen nunmehr unseren Untersuchungen ein
rechtwinkliges Coordinatensystem zu Grunde legen. Es seien
$\xi$, $\eta$, $\zeta$ die Coordinaten des Mittelpunktes einer Kugel vom
Radius $r$, und $x$, $y$, $z$ diejenigen eines Punktes $A$, welcher
von jenem Mittelpunkte den Abstand $d$ hat. Dann wird die
Potenz der Kugel im Punkte $A$ dargestellt durch:
\[
d^2 - r^2 = (x-\xi)^2 + (y-\eta)^2 + (z-\zeta)^2 - r^2,
\]
und insbesondere die Potenz $p$ im Coordinaten-Anfange durch:
\[
\tag{1}
  p = \xi^2 + \eta^2 + \zeta^2 - r^2.
\]
Wir haben also den Satz:
\begin{list}{}{\leftmargin1em\rightmargin2em\topsep0em}\item
{\glqq}Die Potenz einer Kugel im Punkte $(x, y, z)$ wird
dargestellt durch:
\[
\tag{2}
(x-\xi)^2 + (y-\eta)^2 + (z-\zeta)^2 - r^2
= x^2 + y^2 + z^2 - 2\xi x - 2\eta y - 2\zeta z + p,
\]
wenn $(\xi,\eta,\zeta)$ ihr Mittelpunkt, $r$ ihr Radius ist und
$p$ ihre Potenz im Anfangspunkte der Coordinaten.{\grqq}
\end{list}
Liegt der Punkt $(x, y, z)$ auf der Kugelfl\"ache, so ist
$d = r$, die Potenz ist Null, und wir erhalten aus (2) die
Gleichung der Kugel in der Form:
\[
\tag{3}
x^2 + y^2 + z^2 - 2\xi x - 2\eta y - 2\zeta z + p = 0.
\]

168. Die Kugel ist v\"ollig bestimmt, wenn die rechtwinkligen
Coordinaten $\xi$, $\eta$, $\zeta$ ihres Mittelpunktes und ihre
Potenz $p$ im Anfangspunkte der Coordinaten gegeben sind.
Wir k\"onnen $\xi$, $\eta$, $\zeta$ und $p$ die vier {\glqq}bestimmenden Gr\"ossen{\grqq}
%-----File: 088.png---------------------------------
oder {\glqq}Coordinaten{\grqq} der Kugel nennen, und mit $(\xi, \eta, \zeta, p)$
die Kugel selbst bezeichnen. Die Einf\"uhrung der vierten
Kugelcoordinate $p$ anstatt des Radius $r$ empfiehlt sich schon
deshalb, weil die Gleichung der Kugel in Bezug auf $\xi$, $\eta$, $\zeta$, $p$
linear ist, in Bezug auf $\xi$, $\eta$, $\zeta$, $r$ dagegen quadratisch.
Uebrigens kann der Radius $r$ leicht aus den Kugelcoordinaten
berechnet werden mittelst der Gleichung (1):
\[
r^2 = \xi^2 + \eta^2 + \zeta^2 - p.
\]
Die Kugel $(\xi, \eta, \zeta, p)$ ist eine Punktkugel, wenn $p = \xi^2+\eta^2+\zeta^2$
ist; sie artet in eine Ebene aus, wenn ihre Coordinaten unendlich
werden.

169. Zwei Kugeln $(\xi, \eta, \zeta, p)$ und $(\xi_1, \eta_1, \zeta_1, p_1)$ haben
gleiche Potenz in einem Punkte $(x, y, z)$, wenn dessen Coordinaten
der Gleichung:
\[
-2\,\xi x - 2\,\eta y - 2\,\zeta z + p = -2\,\xi_1 x - 2\,\eta_1 y - 2\,\zeta_1 z + p_1
\]
oder:
\[
(\xi-\xi_1)\,x + (\eta-\eta_1)\,y + (\zeta-\zeta_1)\,z = \frac{(p-p_1)}{2}
\]
gen\"ugen. Diese Gleichung repr\"asentirt die Potenz-Ebene der
beiden Kugeln, welche alle Potenzpunkte derselben enth\"alt
und zu der Centrale der Kugeln normal ist. --- Die beiden
Kugeln $(\xi, \eta, \zeta, p)$ und $(\xi_1, \eta_1, \zeta_1, p_1)$ schneiden sich rechtwinklig,
wenn:
\[
\tag*{(4)}
\xi\xi_1 + \eta\eta_1 + \zeta\zeta_1 = \frac{p+p_1}{2}
\]
ist. Denn auf diese Gleichung reducirt sich die folgende:
\begin{gather*}
(\xi^2 + \eta^2 + \zeta^2 - p) + (\xi_1^2 + \eta_1^2 + \zeta_1^2 - p_1)
\\
= (\xi-\xi_1)^2 + (\eta-\eta_1)^2 + (\zeta-\zeta_1)^2,
\end{gather*}
welche die Summe der Quadrate beider Kugelradien gleich
dem Quadrate des Abstandes der Centra setzt; auch erh\"alt
man jene Gleichung (4) leicht, wenn man die Potenz der einen
Kugel im Centrum der anderen gleich dem Quadrate des
Radius dieser anderen Kugel setzt.

170. Fassen wir die Kugelcoordinaten $\xi$, $\eta$, $\zeta$, $p$ als
ver\"anderliche Gr\"ossen auf, so k\"onnen wir jede beliebige Kugel
durch sie darstellen; nehmen wir insbesondere $\xi$, $\eta$, $\zeta$, $p$ unendlich
gross an, aber so dass ihre Verh\"altnisse endliche
Werthe erhalten, so stellen wir durch sie eine Ebene dar,
%-----File: 089.png-----------------------------------
welche auf den Coordinaten-Axen die Strecken $\frac{p}{2\xi}$, $\frac{p}{2\eta}$ und $\frac{p}{2\zeta}$
abschneidet. Da jede der vier Coordinaten $\xi$, $\eta$, $\zeta$, $p$ unabh\"angig
von den \"ubrigen unendlich viele Werthe annehmen kann,
so giebt es vierfach unendlich viele Kugeln, und alle Kugeln
des Raumes bilden eine Mannigfaltigkeit von vier Dimensionen.

171. Werden die ver\"anderlichen Coordinaten $\xi$, $\eta$, $\zeta$, $p$
irgend einer Bedingungsgleichung unterworfen, so k\"onnen sie
nicht mehr jede beliebige Kugel, sondern nur noch dreifach
unendlich viele Kugeln darstellen. Wir nennen die Gesammtheit
aller Kugeln, deren Coordinaten einer gegebenen Gleichung
gen\"ugen, ein Kugelsystem von drei Dimensionen oder
dritter Stufe, oder auch nach Pl\"ucker einen {\glqq}Kugelcomplex{\grqq},
und wollen sagen, der Complex werde durch die Gleichung
{\glqq}dargestellt{\grqq} oder {\glqq}repr\"asentirt{\grqq}. Der Complex heisst algebraisch
oder transcendent, je nachdem die Gleichung algebraisch
oder transcendent ist; im ersteren Falle nennen wir
ihn linear, quadratisch, cubisch oder vom $n^{\text{ten}}$ Grade, wenn
seine Gleichung in Bezug auf $\xi$, $\eta$, $\zeta$, $p$ linear, quadratisch,
cubisch resp.\ vom $n^{\text{ten}}$ Grade ist. So z.~B.\ bilden alle Kugeln, deren Mittelpunkte auf einer gegebenen Fl\"ache liegen,
einen Kugelcomplex; derselbe wird durch die Gleichung
der Fl\"ache dargestellt. Alle Kugeln vom gegebenen Radius
$r$ bilden einen quadratischen Kugelcomplex, dessen Gleichung
$p = \xi^2 + \eta^2 + \zeta^2 - r^2$ ist; insbesondere bilden alle
Punktkugeln einen quadratischen Complex.

172. Alle Kugeln, deren Coordinaten zwei verschiedenen
Gleichungen gen\"ugen, bilden im Allgemeinen ein Kugelsystem
zweiter Stufe oder nach Pl\"ucker's Bezeichnung eine {\glqq}Kugelcongruenz{\grqq}.
Diese Congruenz besteht aus allen gemeinschaftlichen
Kugeln der beiden durch die Gleichungen repr\"a\-sen\-tirten
Kugelcomplexe; letztere durchdringen oder {\glqq}schneiden{\grqq}
sich in der Congruenz, falls sie sich nicht in derselben {\glqq}ber\"uhren{\grqq}.
Drei Kugelcomplexe, welche keine Kugelcongruenz
und auch keinen Theil einer Congruenz mit einander gemein
haben, durchdringen sich in einem Kugelsystem erster Stufe,
welches wir auch eine {\glqq}Kugelschaar{\grqq} nennen; diese Kugelschaar
besteht aus den einfach unendlich vielen gemeinschaftlichen
Kugeln der drei Complexe und wird durch die drei
Gleichungen der Complexe dargestellt.

%-----File: 090.png-----------------------------------

173. Ueberhaupt bilden alle Kugeln, deren Coordinaten $i$
Bedingungsgleichungen gen\"ugen, im Allgemeinen ein Kugelsystem
von $4-i$ Dimensio\-nen oder $4-i^{\text{ter}}$ Stufe. Sie
k\"onnen jedoch in besonderen F\"allen eine Mannigfaltigkeit
von mehr als $4-i$ Dimensionen bilden, auch wenn, wie
wir voraussetzen, keine der $i$ Gleichungen eine Folge der
\"ubrigen ist. Diese Ausnahmef\"alle sind demjenigen vergleichbar,
in welchem drei Fl\"achen eine krumme oder gerade
Linie mit einander gemein haben anstatt discreter Punkte,
wie in dem allgemeinen Falle. Ein Kugelsystem heisst
algebraisch, wenn alle seine Gleichungen algebraisch sind;
es heisst linear, wenn seine Gleichungen algebraisch und vom
ersten Grade sind in Bezug auf die Kugelcoordinaten.

174. Ein linearer Kugelcomplex ist nichts anderes als
ein Kugelgeb\"usch, und zwar insbesondere ein symmetrisches
Geb\"usch, wenn seine Gleichung die Form:
\[
  A\xi + B\eta + C\zeta + D = 0
\]
hat. Diese Gleichung n\"amlich repr\"asentirt die Symmetrie-Ebene
des Ge\-b\"u\-sches, in welcher die Mittelpunkte aller
seiner Kugeln liegen. Im Allgemeinen enth\"alt die Gleichung
des linearen Complexes auch die vierte Kugel-Coordinate $p$,
und kann auf die Form:
\[
\tag{5}
  p = a\xi + b\eta + c\zeta + d
\]
gebracht werden; weil aber dann die Potenz einer beliebigen
Kugel des Complexes im Punkte $(x,\, y,\, z)$ dargestellt wird durch:
\[
  x^2 + y^2 + z^2 - 2\xi x - 2\eta y - 2\zeta z
+ ( a\xi + b\eta + c\zeta + d ),
\]
so haben alle Kugeln des Complexes im Punkte
$\left( \frac{a}{2}, \frac{b}{2}, \frac{c}{2} \right)$
die Potenz
\[
  \textstyle\frac{1}{4} (a^2 + b^2 + c^2) + d.
\]
Die Gleichung~(5) stellt also ein Kugelgeb\"usch dar vom
Centrum
$\left( \frac{a}{2}, \frac{b}{2}, \frac{c}{2} \right)$
und der Potenz
$\frac{1}{4} (a^2 + b^2 + c^2) + d$; die
Orthogonalkugel dieses Geb\"usches aber hat die Coordinaten
$\left( \frac{a}{2}, \frac{b}{2}, \frac{c}{2}, -d \right)$
wie sich durch Vergleichung von (5) mit (4)
ohne Weiteres ergiebt. Die Constanten $a$, $b$, $c$, $d$ der Gleichung
(5) k\"onnen so bestimmt werden, dass das Geb\"usch durch vier
%-----File: 091.png-----------------------------------
beliebig angenommene Kugeln geht (vgl.~12.). --- Eine lineare
Kugelcongruenz ist ein Kugelb\"undel, und eine lineare Kugelschaar
ist ein Kugelb\"uschel; der Beweis folgt aus dem obigen
Satze und aus den Definitionen der linearen Kugelsysteme.
Dass vier Kugelgeb\"usche eine und im Allgemeinen nur eine
Kugel mit einander gemein haben, beweist man durch Auf\/l\"osung
ihrer vier linearen Gleichungen. Auch die \"ubrigen
S\"atze des \S~9 \"uber lineare Kugelsysteme k\"onnen hier mittelst
einfacher Rechnungen bewiesen werden.

175. Sind $\xi$, $\eta$, $\zeta$, $p$ und $\xi'$, $\eta'$, $\zeta'$, $p'$ die Coordinaten
einer beliebigen Kugel in Bezug auf zwei verschiedene rechtwinklige
Coordinaten-Systeme, so werden bekanntlich die
Mittelpunkts-Coordinaten $\xi$, $\eta$, $\zeta$ durch lineare Functionen
von $\xi'$, $\eta'$, $\zeta'$ ausgedr\"uckt; aber auch die Potenz $p$ im Coordinatenanfangspunkte
des ersten Systemes ist alsdann eine
lineare Function von $\xi'$, $\eta'$, $\zeta'$ und $p'$. Denn es wird (167.)
\[
  p = a^2 + b^2 + c^2 - 2\xi'a + \eta'b + \zeta'c + p',
\]
wenn $a$, $b$, $c$ die Coordinaten jenes Anfangspunktes in Bezug
auf das zweite Coordinatensystem bezeichnen. Bei dem
Uebergange von einem rechtwinkligen Coordinatensysteme
zu einem anderen bleibt deshalb der Grad der Gleichungen
algebraischer Kugelsysteme unge\"andert.

176. Durch reciproke Radien, deren Potenz $= k$ und
deren Centrum der Coordinaten-Anfang ist, wird jedem Punkte
$(x,\, y,\, z)$ ein Punkt $(x_1,\, y_1,\, z_1)$ zugeordnet, so dass:
\[
  x: x_1 = y: y_1 = z: z_1 \text{ und }
 (x^2 + y^2 + z^2) \cdot (x_1^2 + y_1^2 + z_1^2) = k^2
\]
und demgem\"ass:
\[
\tag{6}
  x = \frac{kx_1}{x_1^2 + y_1^2 + z_1^2}, \;
  y = \frac{ky_1}{x_1^2 + y_1^2 + z_1^2}, \;
  z = \frac{kz_1}{x_1^2 + y_1^2 + z_1^2}
\]
wird. Durch diese Substitution geht die Gleichung:
\[
  x^2 + y^2 + z^2 - 2\xi x - 2\eta y - 2\zeta z + p = 0,
\]
einer Kugel $(\xi, \eta, \zeta, p)$ \"uber in diejenige einer anderen Kugel
$(\xi_1, \eta_1, \zeta_1, p_1)$, n\"amlich in:
\[
  p_1 - 2\xi_1 x_1 - 2\eta_1 y_1 - 2\zeta_1 z_1
+ x_1^2 + y_1^2 + z_1^2 = 0,
\]
wenn gesetzt wird:
\[
\tag{7}
  \xi_1 = \frac{k\xi}{p}, \quad
  \eta_1 = \frac{k\eta}{p}, \quad
  \zeta_1 = \frac{k\zeta}{p}, \quad
   p_1 = \frac{k^2}{p}, \quad
\]
Die Kugel $(\xi, \eta, \zeta, p)$ wird also durch die reciproken Radien
%-----File: 092.png-----------------------------------
in die Kugel ($\xi_1$, $\eta_1$, $\zeta_1$, $p_1$) transformirt, und wir erhalten
aus (7) die Substitution:
\[
  \xi   = \frac{k   \xi_1}{p_1},  \quad
  \eta  = \frac{k  \eta_1}{p_1},  \quad
  \zeta = \frac{k \zeta_1}{p_1},  \quad
  p     = \frac{k^2      }{p_1}.
\]
Setzen wir in irgend eine Gleichung $n^{\text{ten}}$ Grades f\"ur
$\xi$, $\eta$, $\zeta$, $p$
diese Werthe ein und multipliciren sodann die Gleichung
mit $p_1^n$, so erhalten wir eine Gleichung $n^{\text{ten}}$ Grades f\"ur
$\xi_1$, $\eta_1$, $\zeta_1$, $p_1$.
Ein Kugelcomplex $n^{\text{ten}}$ Grades verwandelt sich
also durch reciproke Radien in einen Kugelcomplex $n^{\text{ten}}$
Grades. Dieser Satz gilt f\"ur jede beliebige Lage des Centrums
der reciproken Radien, weil der Anfangspunkt der Coordinaten
nach diesem Centrum hin verlegt werden kann\footnote{)
  Zwei Kugeln ($\xi$, $\eta$, $\zeta$, $p$) und
  ($\xi_1$, $\eta_1$, $\zeta_1$, $p_1$) sind einander zugeordnet
  in Bezug auf eine beliebige dritte
  ($\xi_0$, $\eta_0$, $\zeta_0$, $p_0$), wenn:
\[
  \frac{\xi  -  \xi_0}{\xi_1  -  \xi_0}
= \frac{\eta - \eta_0}{\eta_1 - \eta_0}
= \frac{\zeta-\zeta_0}{\zeta_1-\zeta_0}
= \frac{ p   -    p_0}{ p_1   -    p_0}
= \frac{r_0^2}{k_1} = \frac{k}{r_0^2},
\]
  worin
\[
  r_0^2 = \xi_0^2 + \eta_0^2 + \zeta_0^2 - p_0,\
  k = \xi_0^2 + \eta_0^2 + \zeta_0^2
  - 2\xi\xi_0 - 2\eta\eta_0 - 2\zeta\zeta_0 + p
\]
  und
\[
  k_1 = \xi_0^2 + \eta_0^2 + \zeta_0^2
      - 2\xi_1\xi_0 + 2\eta_1\eta_0 - 2\zeta_1\zeta_0 + p_1
\]
  ist. Den Beweis dieser Formeln unterdr\"ucken wir der K\"urze wegen.}).

177. Ist die Gleichung einer Kugel gegeben in der Form:
\[
  \alpha_0(x^2 + y^2 + z^2)
- 2\alpha_1 x - 2\alpha_2 y - 2\alpha_3 z + \alpha_4 = 0,
\]
so k\"onnen wir deren f\"unf Coefficienten $\alpha_i$ als die Coordinaten
der Kugel auf\/fassen und die Kugel durch
($\alpha_0$, $\alpha_1$, $\alpha_2$, $\alpha_3$, $\alpha_4$)
oder k\"urzer durch $\alpha$ darstellen; denn diese Coefficienten und
sogar die Verh\"altnisse derselben bestimmen die Kugel vollst\"andig.
Diese etwas allgemeineren Kugelcoordinaten $\alpha_i$ sind
mit den vorigen verkn\"upft durch die einfachen Gleichungen:
\[
\tag{8}
  \xi   = \frac{\alpha_1}{\alpha_0}, \quad
  \eta  = \frac{\alpha_2}{\alpha_0}, \quad
  \zeta = \frac{\alpha_3}{\alpha_0}, \quad
  p     = \frac{\alpha_4}{\alpha_0};
\]
wird $\alpha_0 = 1$ gesetzt, so werden
$\alpha_1$, $\alpha_2$, $\alpha_3$, $\alpha_4$
identisch mit
den gew\"ohnlichen Kugelcoordinaten
$\xi$, $\eta$, $\zeta$, $p$.
Durch die
reciproken Radien (6) verwandelt sich $\alpha$ in eine Kugel $\beta$,
deren Coordinaten aus den Gleichungen:
\[
\tag{9}
  \beta_0 =    \alpha_4, \quad
  \beta_1 = k  \alpha_1, \quad
  \beta_2 = k  \alpha_2, \quad
  \beta_3 = k  \alpha_3, \quad
  \beta_4 = k^2\alpha_0
\]
berechnet werden k\"onnen. Ein Kugelcomplex $n^{\text{ten}}$ Grades
wird dargestellt durch eine \so{homogene} Gleichung $n^{\text{ten}}$ Grades
%-----File: 093.png-----------------------------------
zwischen $\alpha_0$, $\alpha_1$, $\alpha_2$, $\alpha_3$, $\alpha_4$; er verwandelt sich durch die
reciproken Radien in einen Kugelcomplex $n^{\text{ten}}$ Grades, weil
seine Gleichung durch die Substitution (9) in eine homogene
Gleichung $n^{\text{ten}}$ Grades f\"ur $\beta_0$, $\beta_1$, $\beta_2$, $\beta_3$,
$\beta_4$ \"ubergeht.

178. Die Kugel $\alpha$ hat den Punkt $\left(\frac{\alpha_1}{\alpha_0},
       \frac{\alpha_2}{\alpha_0},
       \frac{\alpha_3}{\alpha_0}\right)$ zum
Centrum und ihr Radius $r$ ergiebt sich (168.) aus der Gleichung:
\[
\tag{10}
\alpha_0^2 r^2 = \alpha_1^2 + \alpha_2^2 + \alpha_3^2 - \alpha_0\alpha_4.
\]
Sie ist eine Punktkugel, wenn $\alpha_1^2 + \alpha_2^2 + \alpha_3^2 = \alpha_0\alpha_4$, und
artet in eine Ebene aus, wenn $\alpha_0 = 0$ ist; im letzteren Falle
schneidet die Ebene auf den Coordinatenaxen die Strecken
$\frac{\alpha_4}{2\alpha_1}$, $\frac{\alpha_4}{2\alpha_2}$ und
$\frac{\alpha_4}{2\alpha_3}$ ab. Zwei Kugeln $\alpha$ und $\beta$ schneiden sich
rechtwinklig, wenn:
\[
\tag{11}
\alpha_1\beta_1 + \alpha_2\beta_2 + \alpha_3\beta_3 =
\textstyle\frac{1}{2} (\alpha_4\beta_0 + \alpha_0\beta_4)
\]
ist (169.). Eine Kugel $\alpha$ ist nur dann zu sich selbst rechtwinklig,
wenn sie sich auf einen Punkt reducirt; denn
f\"ur $\beta_i = \alpha_i$ geht (11) \"uber in $\alpha_1^2 + \alpha_2^2 + \alpha_3^2 =
\alpha_0\alpha_4$. --- Alle Kugeln $\alpha$, deren Coordinaten der linearen homogenen
Gleichung:
\[
\tag{12}
a_0\alpha_0 + a_1\alpha_1 + a_2\alpha_2 +
a_3\alpha_3 + a_4\alpha_4 = 0
\]
gen\"ugen, bilden einen linearen Kugelcomplex, d.~h.\ ein Kugelgeb\"usch;
f\"ur die Orthogonalkugel $\beta$ dieses Geb\"usches erhalten
wir durch Vergleichung von (12) mit (11) die Coordinaten:
\[
\tag{13}
  \beta_0 = -2 a_4,\quad
  \beta_1 = a_1,\quad
  \beta_2 = a_2,\quad
  \beta_3 = a_3,\quad
  \beta_4 = -2 a_0,
\]
und es ist
$\left(-\frac{a_1}{2a_4},
       -\frac{a_2}{2a_4},
       -\frac{a_3}{2a_4}\right)$ das Centrum und
$\frac{1}{4 a_4^2} (a_1^2 + a_2^2 + a_3^2 - 4a_0a_4)$ die Potenz des Geb\"usches (vgl.\ 174.).

179. Eine Kugel $\gamma$ liegt mit zwei gegebenen Kugeln
$\alpha$ und $\alpha'$ in einem Kugelb\"uschel, wenn ihre Coordinaten den
Gleichungen:
\[
\tag{14}
\gamma_0 = \lambda\alpha_0 + \lambda' \alpha_0',\;
\gamma_1 = \lambda\alpha_1 + \lambda' \alpha_1',\;
\ldots,\;
\gamma_4 = \lambda\alpha_4 + \lambda' \alpha_4'
\]
gen\"ugen. Denn durch Elimination der willk\"urlichen Constanten
$\lambda$ und $\lambda'$ aus den f\"unf Gleichungen (14) ergeben
sich drei lineare homogene Gleichungen f\"ur die Coordinaten
von $\gamma$, und diese drei Gleichungen repr\"asentiren den durch
$\alpha$ und $\alpha'$ gehenden Kugelb\"uschel. Die beiden Kugeln:
\[
(\lambda\alpha_0\pm\lambda'\alpha_0', \;
 \lambda\alpha_1\pm\lambda'\alpha_1', \;
 \lambda\alpha_2\pm\lambda'\alpha_2', \;
 \lambda\alpha_3\pm\lambda'\alpha_3', \;
 \lambda\alpha_4\pm\lambda'\alpha_4')
\]
%-----File: 094.png---------------------------------
sind durch die Kugeln $\alpha$ und $\alpha'$ harmonisch getrennt; denn
man findet ohne Schwierigkeit, dass ihre Mittelpunkte durch
diejenigen von $\alpha$ und $\alpha'$ harmonisch getrennt sind, und dass
die vier Kugeln mit einer f\"unften Kugel $\beta$ vier harmonische
Potenz-Ebenen bestimmen. Uebrigens kann der Satz auch
als Definition harmonischer Kugeln betrachtet werden. --- Wenn
in (14) das Verh\"altniss der Parameter $\lambda$ und $\lambda'$ sich
stetig \"andert, so beschreibt die Kugel $\gamma$ den durch $\alpha$ und $\alpha'$
gehenden Kugelb\"uschel.

\begin{center}
\makebox[15em]{\hrulefill}
\end{center}


\abschnitt{\S.~20. \\[\parskip]
Projective Verwandtschaft linearer Kugelsysteme.}\label{p20}


\hspace{\parindent}%
180. Wir wollen mit $R$ und $R'$ zwei R\"aume bezeichnen, in
jedem derselben ein rechtwinkliges Coordinatensystem annehmen,
und eine beliebige Kugel $\alpha$ von $R$ mittelst ihrer Coordinaten
durch $(\alpha_0, \alpha_1, \alpha_2,\alpha_3,\alpha_4)$ sowie eine Kugel $\alpha'$ von $R'$ durch
$(\alpha_0', \alpha_1', \alpha_2',\alpha_3',\alpha_4')$ darstellen. Durch die bilineare Gleichung:
\[
\tag*{(A)}
\begin{Bmatrix}
\phantom{+}(a_{00}\alpha_0 + a_{01}\alpha_1 + a_{02}\alpha_2 + a_{03}\alpha_3 + a_{04}\alpha_4)\,\alpha_0' \\
         + (a_{10}\alpha_0 + a_{11}\alpha_1 + a_{12}\alpha_2 + a_{13}\alpha_3 + a_{14}\alpha_4)\,\alpha_1' \\
\hdotsfor{1} \\
\hdotsfor{1} \\
         + (a_{40}\alpha_0 + a_{41}\alpha_1 + a_{42}\alpha_2 + a_{43}\alpha_3 + a_{44}\alpha_4)\,\alpha_4'
\end{Bmatrix} = 0
\]
sind dann mit jeder Kugel des einen Raumes unendlich viele
Kugeln des anderen {\glqq}verkn\"upft{\grqq}, indem ihre Coordinaten der
Gleichung (A) gen\"ugen. Und zwar sind mit einer bestimmten
Kugel $\alpha$ des Raumes $R$ alle Kugeln eines in $R'$ liegenden
Geb\"usches verkn\"upft. Die Orthogonalkugel $\beta'$ dieses Geb\"usches
hat (178.) die Coordinaten:
\[
\tag*{(B)}
\left\{
\begin{aligned}
\beta_0' &=          -2\,( a_{40}\alpha_0 + a_{41}\alpha_1 + \hdots + a_{44}\alpha_4), \\
\beta_1' &= \phantom{-2\,(}a_{10}\alpha_0 + a_{11}\alpha_1 + \hdots + a_{14}\alpha_4, \\
\beta_2' &= \phantom{-2\,(}a_{20}\alpha_0 + a_{21}\alpha_1 + \hdots + a_{24}\alpha_4, \\
\beta_3' &= \phantom{-2\,(}a_{30}\alpha_0 + a_{31}\alpha_1 + \hdots + a_{34}\alpha_4, \\
\beta_4' &=          -2\,( a_{00}\alpha_0 + a_{01}\alpha_1 + \hdots + a_{04}\alpha_4);
\end{aligned}
\right.
\]
wir wollen sagen, diese Kugel $\beta'$ von $R'$ {\glqq}entspreche{\grqq} der
Kugel $\alpha$ des Raumes $R$ und sei ihr {\glqq}homolog{\grqq}. Ganz \"ahnliche
lineare Gleichungen erh\"alt man f\"ur die Coordinaten
%-----File: 095.png-----------------------------------
der Kugel $\beta$ von $R$, welche einer beliebigen Kugel $\alpha'$ von $R'$
entspricht, wenn man die bilineare Gleichung (A) identificirt
mit der Gleichung:
\[
\tag{C}
-\textstyle\frac{1}{2}\beta_4\alpha_0 + \beta_1\alpha_1 +
\beta_2\alpha_2 + \beta_3\alpha_3 -\textstyle\frac{1}{2}\beta_0\alpha_4
=0;
\]
diese letztere Gleichung n\"amlich ist die Bedingung daf\"ur,
dass die Kugel $\beta$ zu einer jeden mit $\alpha'$ verkn\"upften Kugel $\alpha$
normal ist (178.).

181. Durch die lineare Substitution (B), deren Determinante
wir als von Null verschieden vorraussetzen, %[sic!]
ist einer jeden Kugel $\alpha$ des Raumes $R$ die ihr entsprechende Kugel $\beta'$
von $R'$ zugewiesen, zugleich aber jedem Kugelcomplex $n^{\text{ten}}$
Grades des einen Raumes ein ihm entsprechender Kugelcomplex
$n^{\text{ten}}$ Grades des anderen. Denn eine homogene Gleichung
$n^{\text{ten}}$ Grades f\"ur $\beta_0'$, $\beta_1'$, $\beta_2'$, $\beta_3'$, $\beta_4'$ geht durch die
Substitution (B) \"uber in eine homogene Gleichung $n^{\text{ten}}$ Grades
f\"ur $\alpha_0$, $\alpha_1$, $\alpha_2$, $\alpha_3$, $\alpha_4$. Wenn insbesondere eine der
homologen Kugeln $\alpha$ und $\beta'$ ein Kugelgeb\"usch beschreibt, so
beschreibt auch die andere ein Kugelgeb\"usch. Auch jedem
Kugelb\"undel oder -B\"uschel des einen Raumes entspricht
folglich ein Kugelb\"undel resp.\ -B\"uschel des anderen. Wir
nennen diese eindeutige Beziehung, welche durch die Substitution
(B) zwischen den vierfach unendlichen Kugelsystemen
der R\"aume $R$ und $R'$ hergestellt wird, eine {\glqq}projective{\grqq},
und wollen auch von zwei einander entsprechenden Kugel-Complexen,
-Congruenzen oder -Schaaren der R\"aume sagen,
sie seien {\glqq}projectiv{\grqq} auf einander bezogen. --- Zwei Kugelsysteme
vierter Stufe, welche zur Deckung gebracht werden
k\"onnen, sind allemal projectiv; denn die Substitution:
\[
  \beta_0'=\alpha_0,\quad
  \beta_1'=\alpha_1,\quad
  \beta_2'=\alpha_2,\quad
  \beta_3'=\alpha_3,\quad
  \beta_4'=\alpha_4
\]
ist in (B) enthalten. Ebenso sind zwei Kugelsysteme projectiv,
wenn sie durch reciproke Radien in einander transformirt
werden k\"onnen (177.). Auch beweist man leicht,
dass zwei Kugelsysteme, welche zu einem und demselben
dritten projectiv sind, zu einander projectiv sein m\"ussen.

182. Wenn die Coordinaten $\alpha_i$ in der Gleichung (C)
dieselbe Kugel repr\"asentiren, wie in den f\"unf Gleichungen (B),
so sind $\beta'$ und $\beta$ zwei Kugeln, denen zwei mit einander
verkn\"upfte Kugeln $\alpha$ und $\alpha'$ entsprechen, und jede der Kugeln $\beta'$
und $\beta$ ist die Orthogonalkugel des Geb\"usches, welches mit
%-----File: 096.png---------------------------------
der der anderen entsprechenden Kugel verkn\"upft ist. Wir
erhalten aber in diesem Falle, indem wir die f\"unf Coordinaten
$\alpha_i$ aus den sechs Gleichungen (B) und (C) eliminiren,
f\"ur die Coordinaten $\beta_i$ und $\beta_i'$ die bilineare Gleichung:
\[
\tag*{(D)}
0=
\begin{vmatrix}
a_{00} & a_{01} & a_{02} & a_{03} & a_{04} & - \frac{1}{2}\,\beta_4' \\
a_{10} & a_{11} & a_{12} & a_{13} & a_{14} &                \beta_1' \\
a_{20} & a_{21} & a_{22} & a_{23} & a_{24} &                \beta_2' \\
a_{30} & a_{31} & a_{32} & a_{33} & a_{34} &                \beta_3' \\
a_{40} & a_{41} & a_{42} & a_{43} & a_{44} & - \frac{1}{2}\,\beta_0' \\
- \frac{1}{2}\,\beta_4 & \beta_1 & \beta_2 & \beta_3 & - \frac{1}{2}\,\beta_0 & 0
\end{vmatrix}
\]

Wenn also zwei Kugeln $\alpha$ und $\alpha'$ durch die Gleichung (A)
verkn\"upft sind, so sind die ihnen entsprechenden Kugeln $\beta'$
und $\beta$ durch die bilineare Gleichung (D) verkn\"upft. Dieser
Satz gilt auch umgekehrt, weil aus (B) und (D) die Gleichung (C)
folgt.

183. Die bilineare Gleichung (A) z\"ahlt 25 Constanten $a_{ik}$,
und die Verkn\"upfung der Kugeln von $R$ und $R'$ ist nebst
der projectiven Beziehung der beiden Kugelsysteme vierter
Stufe v\"ollig bestimmt, wenn die 24 Verh\"altnisse dieser 25
Constanten gegeben sind. Wir erhalten nun f\"ur diese Verh\"altnisse
eine lineare Gleichung, wenn wir in (A) die Coordinaten
von irgend zwei mit einander verkn\"upften Kugeln
einsetzen. Die projective Beziehung der beiden Kugelsysteme
ist deshalb im Allgemeinen v\"ollig bestimmt, wenn 24 Paare
von mit einander verkn\"upften Kugeln der R\"aume $R$ und $R'$
willk\"urlich angenommen werden. Dabei ist zu bemerken,
dass eine Kugel des einen Raumes mit einem Geb\"usche des
anderen verkn\"upft ist und der Orthogonalkugel desselben
entspricht, sobald sie mit vier beliebigen Kugeln des Geb\"usches
verkn\"upft ist. Um zwei Kugelsysteme vierter Stufe
projectiv auf einander zu beziehen, kann man demnach in
jedem derselben sechs Kugeln, von welchen keine f\"unf in
einem Kugelgeb\"usche liegen, willk\"urlich annehmen, und sodann
den sechs Kugeln des einen Systemes die sechs des
anderen beziehungsweise als entsprechende zuweisen; die projective
Beziehung der Systeme ist dadurch v\"ollig bestimmt.

184. Wir k\"onnen diesen wichtigen Satz auch mit H\"ulfe
der Gleichungen (B) beweisen. Dividiren wir n\"amlich durch
%-----File: 097.png------------------------------------
die erste dieser f\"unf Gleichungen die vier \"ubrigen und setzen
in die so gewonnenen vier Gleichungen die Verh\"altnisse der
Coordinaten von zwei einander entsprechenden Kugeln $\alpha$
und $\beta'$ ein, so erhalten wir vier lineare Gleichungen f\"ur die
24 Verh\"altnisse der Constanten $\alpha_{ik}$; sechs Paare homologer
Kugeln sind also im Allgemeinen ausreichend zur Bestimmung
dieser 24 Verh\"altnisse und damit der projectiven Beziehung
der beiden Kugelsysteme. --- Auf \"ahnliche Weise ergeben
sich die folgenden S\"atze: Um zwei Geb\"usche, B\"undel oder
B\"uschel von Kugeln projectiv auf einander zu beziehen, kann
man in jedem derselben f\"unf, vier resp.\ drei Kugeln willk\"urlich
annehmen und diese Kugeln einander paarweise als
entsprechende zuweisen; die projective Beziehung ist dadurch
im Allgemeinen v\"ollig bestimmt. N\"amlich durch die Gleichungen
von zwei B\"uscheln z.~B., die projectiv auf einander
bezogen werden sollen, sind je drei der Coordinaten $\alpha_i$ resp.\ $\beta_i'$
als lineare Functionen der \"ubrigen, etwa von $\alpha_0'$, $\alpha_1$ resp.\ $\beta_0'$,
$\beta_1'$, bestimmt, so dass die ersten beiden Gleichungen (B)
die Form annehmen:
\[
\beta_0'=a\,\alpha_0+ b\,\alpha_1;\quad
\beta_1'=c\,\alpha_0+d\,\alpha_1,
\quad\text{woraus}\quad
\frac{\beta_1'}{\beta_0'}=\frac{c\,\alpha_0+d\,\alpha_1}{a\,\alpha_0+b\,\alpha_1}.
\]

Setzen wir in diese letzte Gleichung die Coordinaten-Verh\"altnisse
$\frac{\alpha_1}{\alpha_0}$ und $\frac{\beta_1'}{\beta_0'}$ von zwei einander entsprechenden
Kugeln der B\"uschel ein, so erhalten wir f\"ur die drei Verh\"altnisse
der Constanten $a, b, c, d$ eine lineare Gleichung;
drei Paare homologer Kugeln der B\"uschel gen\"ugen deshalb
zur Bestimmung dieser drei Verh\"altnisse und somit der projectiven
Beziehung der B\"uschel.

185. Da congruente B\"uschel auch projectiv sind (181.),
so folgt aus dem soeben bewiesenen Satze: Wenn zwei projective
Kugelb\"uschel drei Kugeln {\glqq}entsprechend gemein{\grqq} haben,
d.~h. wenn drei Kugeln des einen mit den ihnen entsprechenden
Kugeln des anderen zusammen fallen, so haben die B\"uschel
alle ihre Kugeln entsprechend gemein und sind identisch.
Ebenso ergiebt sich: Zwei projective Kugelb\"undel
sind identisch, wenn sie vier Kugeln, von welchen keine drei
in einem B\"uschel liegen, entsprechend gemein haben. Zwei
projective Geb\"usche endlich haben alle ihre Kugeln entsprechend
gemein (sind also identisch), wenn f\"unf Kugeln
%-----File: 098.png-----------------------------------
des einen, von welchen keine vier in einem B\"undel liegen,
mit den ihnen entsprechenden Kugeln des anderen zusammenfallen;
denn die projective Beziehung der Geb\"usche ist durch die
f\"unf Paare homologer Kugeln v\"ollig bestimmt (184.) und kann in
dem vorliegenden Falle keine andere sein als die der Congruenz.

186. Nehmen wir nunmehr an, die R\"aume $R$ und $R'$
seien auf ein und dasselbe Coordinatensystem bezogen, so
ergiebt sich ohne Weiteres: Alle mit sich selbst verkn\"upften
Kugeln bilden einen quadratischen Kugelcomplex; derselbe
wird durch die Gleichung (A) dargestellt, wenn darin
$\alpha_i'=\alpha_i$ f\"ur $i = 0, 1, 2, 3, 4$ gesetzt wird. Nur dann erleidet
dieser Satz eine Ausnahme, wenn $a_{ik} =-a_{ki}$ f\"ur $i$ und
$k = 0, 1, 2, 3, 4$, und folglich $a_{ii} = 0$ ist; denn in diesem
Falle ist durch (A) jede beliebige Kugel $\alpha$ mit sich selbst
verkn\"upft und zu der ihr entsprechenden Kugel $\beta'$ normal.
Wir k\"onnen den Satz auch so aussprechen: In zwei projectiven
Kugelsystemen vierter Stufe bilden diejenigen Kugeln,
welche zu den ihnen entsprechenden normal sind, im Allgemeinen
je einen quadratischen Kugelcomplex.

187. In den projectiven Kugelsystemen der R\"aume $R$
und $R'$ f\"allt die Kugel $\alpha$ mit ihrer entsprechenden $\beta'$ zusammen,
wenn die Coordinaten von $\beta'$ sich verhalten wie diejenigen
von $\alpha$ (177.). Setzen wir nun in den Gleichungen (B):
\[
\beta_0' = \varkappa\alpha_0, \quad
\beta_1' = \varkappa\alpha_1, \quad \ldots \quad
\beta_4' = \varkappa\alpha_4
\]
und eliminiren sodann aus ihnen die Coordinaten $\alpha_i$ so erhalten
wir f\"ur die Constante $\varkappa$ die Gleichung f\"unften Grades:
\[
\left|\begin{array}{lllll}
a_{00} & a_{01} & a_{02} & a_{03} & a_{04}+\frac{\varkappa}{2} \\
a_{10} & a_{11}-\varkappa & a_{12} & a_{13} & a_{14} \\
a_{20} & a_{21} & a_{22}-\varkappa & a_{23} & a_{24} \\
a_{30} & a_{31} & a_{32} & a_{33}-\varkappa & a_{34} \\
a_{40}+\frac{\varkappa}{2} & a_{41} & a_{42} & a_{43} & a_{44}
\end{array}\right|
=0
\]
Zu jeder Wurzel dieser Gleichung geh\"ort eine sich selbst
entsprechende Kugel $\alpha$, und zwar ergeben sich deren Coordinaten
$\alpha_i$, abgesehen von einem gemeinschaftlichen Factor,
aus vier der Gleichungen (B), wenn darin $\beta_i'=\varkappa\alpha_i$ gesetzt
wird. Es giebt also im Allgemeinen f\"unf Kugeln, welche
mit den ihnen entsprechenden zusammenfallen.

%-----File: 099.png---------------------------------

188. Eine beliebige Kugel $\gamma$ kann sowohl zum Raume $R$
als auch zu $R'$ gerechnet werden, und ihr entsprechen deshalb
im Allgemeinen zwei verschiedene Kugeln: eine in $R'$
und eine in $R$. Nur dann fallen f\"ur jede Lage der Kugel $\gamma$
die beiden ihr entsprechenden Kugeln zusammen, wenn die
bilineare Gleichung (A) bei einer Vertauschung von $\alpha_0, \alpha_1,
\alpha_2, \alpha_3, \alpha_4$ mit resp.\ $\alpha_0', \alpha_1', \alpha_2', \alpha_3', \alpha_4'$ unge\"andert bleibt, wenn
also entweder $a_{ik}=a_{ki}$ oder $a_{ik}=-a_{ki}$ ist f\"ur $i$ und $k =
0, 1, 2, 3, 4$. Mit dem ersteren dieser beiden F\"alle besch\"aftigen
wir uns im n\"achsten $\S$, und beschr\"anken uns hier auf eine
einzige Bemerkung zu demselben. N\"amlich wenn $\beta$ die Orthogonalkugel
des Geb\"usches ist, welches durch die Gleichung (A)
mit irgend einer Kugel $\alpha$ verkn\"upft ist, so ist umgekehrt $\alpha$
die Orthogonalkugel des durch die Gleichung (D), nicht
aber durch (A) mit $\beta$ verkn\"upften Geb\"usches, mag nun
$a_{ik}=a_{ki}$ sein oder nicht.

\begin{center}
\makebox[15em]{\hrulefill}
\end{center}

\abschnitt{\S.~21. \\[\parskip]
Quadratische Complexe, Congruenzen und Schaaren
von Kugeln.}\label{p21}


\hspace{\parindent}%
189. Indem wir die Gleichungen (A), (B) und (D) des
\S~20 auch ferner unseren Untersuchungen zu Grunde legen,
nehmen wir nunmehr an, dass $a_{ik}=a_{ki}$ ist f\"ur $i$ und $k =
0, 1, 2, 3, 4$. Die Coordinaten aller Kugeln $\gamma$, welche durch
die bilineare Gleichung (A) mit sich selbst verkn\"upft sind,
gen\"ugen alsdann der quadratischen Gleichung:
\begin{gather*}
a_{00} \gamma_0^2 + 2 \, a_{01}\gamma_0\gamma_1 + 2 \, a_{02}\gamma_0\gamma_2 +\ldots
\tag{E} \\
\ldots + a_{33}\gamma_3^2 + 2 \, a_{34}\gamma_3\gamma_4 + a_{44}\gamma_4^2 =0.
\end{gather*}
Diese Gleichung enth\"alt dieselben 15 Constanten $a_{ik}$, wie
die Gleichungen (A), (B) und (D), und repr\"asentirt einen
ganz beliebigen quadratischen Kugelcomplex. Wenn dieser
Complex gegeben ist, so ist deshalb auch die durch (A) bewirkte
Verkn\"upfung sowie die durch (B) hergestellte projective
Beziehung der Kugeln v\"ollig bestimmt. Durch vierzehn
willk\"urlich angenommene Kugeln kann ein quadratischer
Kugelcomplex gelegt werden.

190. Von zwei durch die bilineare Gleichung (A) verkn\"upften
Kugeln $\alpha$ und $\alpha'$ wollen wir sagen, sie seien {\glqq}conjugirt{\grqq}
in Bezug auf den quadratischen Kugelcomplex (E),
%-----File: 100.png---------------------------------
weil sie zu demselben in analoger Beziehung stehen, wie zu
einer Fl\"ache zweiter Ordnung zwei bez\"uglich derselben conjugirte
Punkte. Setzen wir n\"amlich in der Gleichung (E):
\[
\gamma_i=\lambda\alpha_i + \lambda'\alpha_i'
\quad\text{f\"ur}\quad
i = 0, 1, 2, 3, 4 \quad\text{(vgl. 179.),}
\]
so erhalten wir f\"ur die Parameter $\lambda$, $\lambda'$ derjenigen
%corrected misprinted index in line above ($lambda_1$ for $lambda$)
Kugeln des Complexes, welche mit $\alpha$ und $\alpha'$ in einem B\"uschel liegen,
eine quadratische Gleichung von der Form als $a\lambda^2+a'(\lambda')^2 = 0$;
denn der Coefficient von $2\lambda\lambda'$ wird Null wegen der Gleichung
%corrected misprinted 'Coefficiant' to 'Coefficient' in line above
(A). Die quadratische Gleichung ergiebt f\"ur $\frac{\lambda'}{\lambda}$ zwei
Werthe $\pm b$, die sich nur durch das Vorzeichen unterscheiden;
die zugeh\"origen Kugeln des Complexes aber haben die Coordinaten
$\gamma_i=\lambda\alpha_i\pm b\lambda\alpha_i'$ und sind (179.) durch die Kugeln
$\alpha$ und $\alpha'$ harmonisch getrennt. Also je zwei durch die
Gleichung (A) verkn\"upfte Kugeln trennen diejenigen beiden
Kugeln des Complexes (E) harmonisch, welche mit ihnen
in einem B\"uschel liegen.

191. Wenn die Kugel $\alpha$ dem quadratischen Complexe (E)
angeh\"ort, so verschwindet in der Gleichung $a\lambda^2 + a'(\lambda')^2 = 0$
der Coefficient $a$, und die Wurzeln $\pm b$ werden beide Null; der
Kugelb\"uschel schneidet dann nicht den Kugelcomplex, sondern
{\glqq}ber\"uhrt{\grqq} ihn in der Kugel $\alpha$. Liegen die conjugirten Kugeln $\alpha$
und $\alpha'$ beide in dem quadratischen Complexe, so enth\"alt dieses
alle Kugeln des B\"uschels $\alpha\alpha'$; denn alsdann verschwindet
sowohl $a$ wie $a'$, und $b$ wird ein willk\"urlicher Parameter.

192. Das Kugelgeb\"usch, dessen Kugeln in Bezug auf
den Complex (E) einer gegebenen Kugel $\alpha$ conjugirt, d.~h.\ mit
$\alpha$ durch die Gleichung (A) verkn\"upft sind, wollen wir
die {\glqq}Polare{\grqq} von $\alpha$ bez\"uglich des quadratischen Complexes
nennen. Liegt $\alpha$ in dem Complexe, so wird dieser von dem
Geb\"usche, d.~h.\ von jedem durch $\alpha$ gehenden Kugelb\"uschel desselben
(191.), in $\alpha$ {\glqq}ber\"uhrt{\grqq}. Mit diesem ber\"uhrenden Geb\"usche
hat der Complex eine quadratische Kugelcongruenz gemein,
welche entweder gar keine von $\alpha$ verschiedene reelle Kugel
oder einfach unendlich viele durch $\alpha$ gehende Kugelb\"uschel
enth\"alt und durch einen solchen B\"uschel beschrieben werden
kann (191.). Wenn der Complex einen durch $\alpha$ gehenden
Kugelb\"undel enth\"alt, so liegt dieser B\"undel auch in der
quadratischen Congruenz, und letztere zerf\"allt in diesen und
einen anderen B\"undel.

%-----File: 101.png---------------------------------

193. Die Polaren aller Kugeln eines B\"uschels durchdringen
sich in einem B\"undel und die Polaren aller Kugeln
dieses B\"undels durchdringen sich in jenem B\"uschel; wir
wollen deshalb den B\"undel die {\glqq}Polare{\grqq} des B\"uschels und
den B\"uschel die Polare des B\"undels nennen. Die Polaren
aller Kugeln eines Geb\"usches gehen durch eine Kugel, von
welcher das Geb\"usch die Polare ist. Die Richtigkeit dieser
S\"atze folgt daraus, dass die Gleichung (A) sich nicht \"andert,
wenn $\alpha_i$ mit $\alpha_i'$ vertauscht wird. --- Hat beispielsweise die
Gleichung (A) die einfache Form:
\[
-\frac{1}{2}\,\alpha_4\alpha_0' + \alpha_1\alpha_1' + \alpha_2\alpha_2'
+ \alpha_3\alpha_3' - \frac{1}{2}\,\alpha_0\alpha_4' = 0,
\]
so sind je zwei conjugirte Kugeln zu einander normal (178.),
jede Kugel ist die Orthogonalkugel ihrer Polare, und ein
beliebiger B\"uschel ist die Polare des zu ihm orthogonalen
B\"undels; der quadratische Complex aber hat die Gleichung:
\[
\alpha_1^2 + \alpha_2^2 + \alpha_3^2 - \alpha_0\alpha_4 =0
\]
und besteht aus allen Punktkugeln des Raumes.

194. Im Allgemeinen erf\"ullen die Punktkugeln des
quadratischen Kugelcomplexes nicht den ganzen unendlichen
Raum, sondern eine Fl\"ache vierter Ordnung, welche von
Casey\footnote{) Casey, on Cyclides and Sphero-Quartics (Philos.\ Transactions,
vol.\ CLXI), London 1871.})
und Darboux\footnote{) Darboux, Sur une classe remarquable de courbes et de
surfaces alg\'ebriques, Paris 1873.})
eine {\glqq}Cyclide{\grqq} genannt worden ist.
Wir erhalten die Gleichung dieser Fl\"ache in rechtwinkligen
Punktcoordinaten $\xi$, $\eta$, $\zeta$, wenn wir (177.) in der
Complexgleichung (E) setzen:
\[
\frac{\gamma_1}{\gamma_0} = \xi, \quad
\frac{\gamma_2}{\gamma_0} = \eta, \quad
\frac{\gamma_3}{\gamma_0} = \zeta, \quad
\frac{\gamma_4}{\gamma_0} = p = \xi^2+\eta^2+\zeta^2.
\]
Der Ort aller Punktkugeln des quadratischen Complexes wird
demnach dargestellt durch die Gleichung vierten Grades:
\[
u_2 + 2\,u_1 (\xi^2+\eta^2+\zeta^2)
+ a_{44} (\xi^2+\eta^2+\zeta^2)^2 = 0,
\]
worin:
\begin{align*}
u_2 &= a_{00} + 2\,a_{01}\xi + 2\,a_{02}\eta + 2\,a_{03}\zeta + a_{11}\xi^2
+ \ldots + 2\,a_{23} \eta\zeta + a_{33}\zeta^2, \\
u_1 &= a_{04} + a_{14}\xi + a_{24}\eta + a_{34}\zeta
\end{align*}
%-----File: 102.png-----------------------------------
ist. Von anderen Fl\"achen vierter Ordnung unterscheidet sich
diese Cyclide vor Allem dadurch, dass sie mit einer beliebigen
Kugel eine Raumcurve vierter Ordnung gemein hat, durch
welche Fl\"achen zweiter Ordnung gelegt werden k\"onnen.
Setzen wir n\"amlich $\xi^2 + \eta^2 + \zeta^2$ gleich einer linearen
Function $u$ von $\xi$, $\eta$, $\zeta$, so haben wir die Gleichung einer
beliebigen Kugel, zugleich aber geht die Gleichung der Cyclide
\"uber in die Gleichung $u_2 + 2u_1 u + a_{44}\,u\,u =0$ einer Fl\"ache
zweiter Ordnung, welche mit der Kugel eine auf der Cyclide
liegende Raumcurve vierter Ordnung gemein hat. Diese
Raumcurve kann in zwei Kreise zerfallen.

195. Die Ebenen eines quadratischen Kugelcomplexes
umh\"ullen im Allgemeinen eine Fl\"ache zweiter Classe. Weil
n\"amlich (190.) ein Kugelb\"uschel, der nicht ganz dem Complexe
angeh\"ort, h\"ochstens zwei Kugeln desselben enth\"alt, so
hat insbesondere ein Ebenenb\"uschel im Allgemeinen h\"ochstens
zwei Ebenen mit dem Complexe gemein. Wird in der Complexgleichung
$\gamma_0 = 0$ gesetzt, so erh\"alt man (178.) die Gleichung
der Fl\"ache zweiter Classe in Ebenencoordinaten $2\gamma_1$,
$2\gamma_2$, $2\gamma_3$, $-\gamma_4$.

196. Die Gleichung des quadratischen Kugelcomplexes
kann nach einem bekannten algebraischen Satze\footnote{) %
  S.\ die Abhandlungen von Jacobi, Hermite und Borchardt in dem
  Journal f\"ur d.\ r.\ u.\ a.\ Mathematik Bd.~53, S.~270--283; vgl.\
  Gundelfinger in Hesse's analyt.\ Geometrie des Raumes, 3.~Aufl.,
  S.~449--461.})
auf unendlich
viele Arten auf die kanonische Form:
\[
k_0 P_0^2 + k_1 P_1^2 + k_2 P_2^2 + k_3 P_3^2 + k_4 P_4^2 = 0
\]
gebracht werden, worin die $k_i$ reelle Constanten und die $P_i$
reelle lineare Functionen der Kugelcoordinaten bezeichnen;
und zwar repr\"asentiren die Gleichungen $P_i = 0$ f\"unf Kugelgeb\"usche,
von welchen ein jedes bez\"uglich des Complexes
die Polare derjenigen Kugel ist, welche die \"ubrigen vier
Geb\"usche mit einander gemein haben. Von diesen f\"unf Geb\"uschen
kann das erste willk\"urlich angenommen, und das $i^{\text{te}}$
beliebig durch diejenigen Kugeln gelegt werden, von welchen
die $i-1$ vorher angenommenen Geb\"usche die Polaren sind.
Denn die f\"unf Geb\"usche durchdringen sich zu vieren in einer
ganz beliebigen Gruppe von f\"unf bez\"uglich des Complexes
conjugirten Kugeln. Wenn eine der Constanten $k_i$, etwa $k_0$,
%-----File: 103.png-----------------------------------
Null ist, so hat der quadratische Complex eine Doppelkugel;
die Coordinaten derselben gen\"ugen den vier linearen
Gleichungen:
\[
P_1=0,\quad P_2 = 0,\quad P_3 = 0,\quad P_4 = 0.
\]
Sind zwei von den Constanten $k_i$ Null, so enth\"alt der Complex
alle Kugeln eines B\"uschels doppelt.

197. Der quadratische Kugelcomplex enth\"alt entweder
gar keine oder unendlich viele reelle Kugelb\"uschel resp.\
-B\"undel. Denn jedes Kugelgeb\"usch (resp.\ jeder B\"undel),
welches durch einen reellen B\"undel (B\"uschel) des Complexes
geht, hat mit demselben noch einen reellen B\"undel (B\"uschel)
gemein. Zwei B\"undel des Complexes, die mit einem gegebenen
dritten in zwei Geb\"uschen liegen, schneiden diesen
dritten in zwei Kugelb\"uscheln, die eine Kugel mit einander
gemein haben; die Polare dieser Kugel aber hat mit dem
Complexe alle drei B\"undel gemein (192.) und enth\"alt folglich
beide Geb\"usche, was nur m\"oglich ist, wenn die Kugel
eine Doppelkugel des Complexes ist und ihre Polare unbestimmt
wird.

198. Wir unterscheiden demnach drei Hauptarten des
quadratischen Kugelcomplexes, n\"amlich:
\begin{list}{}{\topsep0mm\itemsep0em\parsep0em\leftmargin2em}
\item[1)] den imagin\"aren Kugelcomplex, dessen Gleichung $P_0^2 +
P_1^2 + P_2^2 + P_3^2 + P_4^2 = 0$ durch keine reellen Werthe
der Kugelcoordinaten befriedigt wird;
\item[2)] den elliptischen, $P_0^2 + P_1^2 + P_2^2 + P_3^2 - P_4^2 = 0$, welcher
mit jedem ihn ber\"uhrenden Geb\"usche nur eine reelle Kugel
(deren Coordinaten n\"amlich reell sind) gemein hat;
\item[3)] den hyperbolischen oder einfach geraden, $P_0^2 + P_1^2
+ P_2^2 - P_3^2 - P_4^2 = 0$, welcher unendlich viele reelle
Kugelb\"uschel, aber keinen reellen Kugelb\"undel enth\"alt.
\end{list}
Der specielle quadratische Complex, welcher eine Doppelkugel
besitzt, ent\-h\"alt entweder keine weitere reelle Kugel,
oder unendlich viele Kugelb\"uschel aber keinen B\"undel, oder
drittens unendlich viele B\"undel; er ist also entweder imagin\"ar,
oder einfach gerade, oder drittens zweifach gerade. Der noch
speciellere Complex mit einem doppelten Kugelb\"uschel enth\"alt
entweder keine reellen Kugeln ausser in diesem B\"uschel,
oder unendlich viele reelle Kugelb\"undel.

199. Alle Kugeln von gegebenem Radius $r$ bilden einen
%-----File: 104.png-----------------------------------
elliptischen Complex zweiten Grades; die Gleichung desselben
(178.) kann auf die Form:
\[
\alpha_1^2 + \alpha_2^2 + \alpha_3^2 +
\left( \frac{\alpha_4}{2r} \right)^2 -
\left( \frac{\alpha_4}{2r} + \alpha_0 r \right)^2
= 0
\]
gebracht werden. Alle Kugeln ($\xi$, $\eta$, $\zeta$, $p$), welche eine gegebene
Kugel ($\xi_1$, $\eta_1$, $\zeta_1$, $p_1$) unter dem gegebenen Winkel $\varphi$ schneiden,
bilden einen quadratischen Complex, dessen Gleichung:
\begin{gather*}
  \cos^2 \varphi \centerdot
  (\xi^2 + \eta^2 + \zeta^2 - p) \centerdot
  (\xi_1^2 + \eta_1^2 + \zeta_1^2 - p_1)
%--corrected misprint (p for p_1) ----^
\\
= \left(
\xi\xi_1 + \eta\eta_1 + \zeta\zeta_1 - \frac{p+p_1}{2}
\right)^2 ,
\end{gather*}
wenn der Coordinatenanfang in das Centrum der gegebenen
Kugel gelegt wird, auf die Form gebracht werden kann:
\[
4 p_1 \cos^2\varphi \;(\xi^2+\eta^2+\zeta^2)
+ (p - p_1 \cos 2\varphi)^2 + (p_1 \sin 2\varphi)^2
= 0
\]
Da nun $p_1$ das negative Quadrat vom Radius der gegebenen
Kugel ist, so ist dieser Kugelcomplex ein hyperbolischer.
Zugleich ergiebt sich f\"ur $\varphi = 0$, dass alle Kugeln, welche
eine gegebene Kugel ber\"uhren, einen einfach geraden quadratischen
Complex bilden, und dass dieser die gegebene Kugel
doppelt enth\"alt. --- Auch die Kugeln, in Bezug auf welche
zwei gegebene Ebenen conjugirt sind, bilden einen hyperbolischen
Complex zweiten Grades.

200. Eine quadratische Kugelcongruenz besteht im Allgemeinen
aus allen Kugeln eines Geb\"usches, deren Mittelpunkte
auf einer Fl\"ache zweiter Ordnung liegen. Denn sie
wird dargestellt durch eine lineare und eine quadratische Gleichung
zwischen den Kugelcoordinaten $(\xi,\eta,\zeta,p)$; die erstere
Gleichung repr\"asentirt das Geb\"usch, und wenn man $p$ aus
beiden Gleichungen eliminirt, so erh\"alt man die Gleichung
der Fl\"ache zweiter Ordnung. Nur dann ist die Eliminirung
unm\"oglich, wenn das Geb\"usch ein symmetrisches ist; doch
kann dieser Fall durch reciproke Radien auf den allgemeinen
zur\"uckgef\"uhrt werden. Die Punktkugeln der quadratischen
Congruenz liegen auf der Raumcurve vierter Ordnung, welche
die Fl\"ache zweiter Ordnung mit der Orthogonalkugel des
Geb\"usches gemein hat; die Ebenen der Congruenz umh\"ullen
im Allgemeinen einen Kegel zweiten Grades. Die Potenzebenen,
welche die Kugeln der Congruenz mit zwei dem Geb\"usche
nicht angeh\"orenden Kugeln bestimmen, umh\"ullen
zwei Fl\"achen zweiter Classe, welche auf einander collinear
%-----File: 105.png-----------------------------------
und auf die Fl\"ache zweiter Ordnung reciprok bezogen sind
(101., 103.). Durch neun beliebige Kugeln eines Geb\"usches
kann allemal eine, und im Allgemeinen nur eine quadratische
Congruenz gelegt werden. --- Was die Cyclide betrifft, welche
auch bei der quadratischen Congruenz (als Umh\"ullungsfl\"ache
der Kugeln derselben) auf\/tritt, so verweisen wir auf die oben
genannten Werke von Casey und Darboux. --- Die Kugeln,
welche eine Fl\"ache zweiter Ordnung doppelt ber\"uhren, bilden
drei quadratische Congruenzen; ihre Mittelpunkte liegen in
den drei Symmetrie-Ebenen der Fl\"ache.

201. Eine quadratische Kugelschaar besteht im Allgemeinen
aus allen Kugeln eines B\"undels, deren Mittelpunkte
auf einem in der Centralebene des B\"undels gegebenen Kegelschnitte
liegen. Sie wird n\"amlich dargestellt durch zwei
lineare und eine quadratische Gleichung zwischen den Coordinaten
$(\xi, \eta, \zeta, p)$, und wenn man $p$ aus der quadratischen
und aus der einen linearen Gleichung mit H\"ulfe der anderen
eliminirt, so erh\"alt man die Gleichungen des Kegelschnittes.
Die Eliminirung wird nur dann unm\"oglich, wenn der Orthogonalkreis
des B\"undels in eine Gerade ausartet; doch kann
dieser Specialfall auf den allgemeinen zur\"uckgef\"uhrt werden
durch reciproke Radien. Die Punkte, welche der Kegelschnitt
mit dem Orthogonalkreise des B\"undels gemein hat,
sind Punktkugeln der Schaar; die Anzahl dieser Punktkugeln
ist h\"ochstens vier. Die Kugelschaar enth\"alt keine, eine oder zwei
reelle Ebenen, jenachdem der Kegelschnitt eine Ellipse, Parabel
oder Hyperbel ist. Die Potenzebenen, welche die Kugeln
der Schaar mit zwei beliebigen Kugeln bestimmen, umh\"ullen
zwei collineare Kegelfl\"achen zweiten Grades, welche auf den
Kegelschnitt reciprok bezogen sind (vgl.\ 200.). Durch f\"unf
beliebige Kugeln eines B\"undels kann im Allgemeinen eine
einzige quadratische Kugelschaar gelegt werden. Auch die
quadratische Kugelschaar wird von einer Cyclide umh\"ullt,
aber von einer ziemlich speciellen, welche eine Schaar von
kreisf\"ormigen Kr\"ummungslinien besitzt (130.); die Ebenen
dieser Kr\"ummungslinien gehen durch die Axe des Kugelb\"undels,
in welchem die Kugelschaar liegt.

\newpage

\small
\pagenumbering{gobble}
\begin{verbatim}

End of Project Gutenberg's Synthetische Geometrie der Kugeln
und linearen Kugelsysteme, by Theodor Reye

*** END OF THIS PROJECT GUTENBERG EBOOK SYNTHETISCHE GEOMETRIE ***

*** This file should be named 17153-t.tex or 17153-t.zip ***
*** or                    17153-pdf.pdf or 17153-pdf.pdf ***
This and all associated files of various formats will be found in:
        http://www.gutenberg.org/1/7/1/5/17153/

Produced by K.F. Greiner, Joshua Hutchinson and the Online
Distributed Proofreading Team at http://www.pgdp.net from
images generously made available by Cornell University
Digital Collections.


Updated editions will replace the previous one--the old editions
will be renamed.

Creating the works from public domain print editions means that no
one owns a United States copyright in these works, so the Foundation
(and you!) can copy and distribute it in the United States without
permission and without paying copyright royalties.  Special rules,
set forth in the General Terms of Use part of this license, apply to
copying and distributing Project Gutenberg-tm electronic works to
protect the PROJECT GUTENBERG-tm concept and trademark.  Project
Gutenberg is a registered trademark, and may not be used if you
charge for the eBooks, unless you receive specific permission.  If
you do not charge anything for copies of this eBook, complying with
the rules is very easy.  You may use this eBook for nearly any
purpose such as creation of derivative works, reports, performances
and research.  They may be modified and printed and given away--you
may do practically ANYTHING with public domain eBooks.
Redistribution is subject to the trademark license, especially
commercial redistribution.



*** START: FULL LICENSE ***

THE FULL PROJECT GUTENBERG LICENSE PLEASE READ THIS BEFORE YOU
DISTRIBUTE OR USE THIS WORK

To protect the Project Gutenberg-tm mission of promoting the free
distribution of electronic works, by using or distributing this work
(or any other work associated in any way with the phrase "Project
Gutenberg"), you agree to comply with all the terms of the Full
Project Gutenberg-tm License (available with this file or online at
http://gutenberg.net/license).


Section 1.  General Terms of Use and Redistributing Project
Gutenberg-tm electronic works

1.A.  By reading or using any part of this Project Gutenberg-tm
electronic work, you indicate that you have read, understand, agree
to and accept all the terms of this license and intellectual
property (trademark/copyright) agreement.  If you do not agree to
abide by all the terms of this agreement, you must cease using and
return or destroy all copies of Project Gutenberg-tm electronic
works in your possession. If you paid a fee for obtaining a copy of
or access to a Project Gutenberg-tm electronic work and you do not
agree to be bound by the terms of this agreement, you may obtain a
refund from the person or entity to whom you paid the fee as set
forth in paragraph 1.E.8.

1.B.  "Project Gutenberg" is a registered trademark.  It may only be
used on or associated in any way with an electronic work by people
who agree to be bound by the terms of this agreement.  There are a
few things that you can do with most Project Gutenberg-tm electronic
works even without complying with the full terms of this agreement.
See paragraph 1.C below.  There are a lot of things you can do with
Project Gutenberg-tm electronic works if you follow the terms of
this agreement and help preserve free future access to Project
Gutenberg-tm electronic works.  See paragraph 1.E below.

1.C.  The Project Gutenberg Literary Archive Foundation ("the
Foundation" or PGLAF), owns a compilation copyright in the
collection of Project Gutenberg-tm electronic works.  Nearly all the
individual works in the collection are in the public domain in the
United States.  If an individual work is in the public domain in the
United States and you are located in the United States, we do not
claim a right to prevent you from copying, distributing, performing,
displaying or creating derivative works based on the work as long as
all references to Project Gutenberg are removed.  Of course, we hope
that you will support the Project Gutenberg-tm mission of promoting
free access to electronic works by freely sharing Project
Gutenberg-tm works in compliance with the terms of this agreement
for keeping the Project Gutenberg-tm name associated with the work.
You can easily comply with the terms of this agreement by keeping
this work in the same format with its attached full Project
Gutenberg-tm License when you share it without charge with others.

1.D.  The copyright laws of the place where you are located also
govern what you can do with this work.  Copyright laws in most
countries are in a constant state of change.  If you are outside the
United States, check the laws of your country in addition to the
terms of this agreement before downloading, copying, displaying,
performing, distributing or creating derivative works based on this
work or any other Project Gutenberg-tm work.  The Foundation makes
no representations concerning the copyright status of any work in
any country outside the United States.

1.E.  Unless you have removed all references to Project Gutenberg:

1.E.1.  The following sentence, with active links to, or other
immediate access to, the full Project Gutenberg-tm License must
appear prominently whenever any copy of a Project Gutenberg-tm work
(any work on which the phrase "Project Gutenberg" appears, or with
which the phrase "Project Gutenberg" is associated) is accessed,
displayed, performed, viewed, copied or distributed:

This eBook is for the use of anyone anywhere at no cost and with
almost no restrictions whatsoever.  You may copy it, give it away or
re-use it under the terms of the Project Gutenberg License included
with this eBook or online at www.gutenberg.net

1.E.2.  If an individual Project Gutenberg-tm electronic work is
derived from the public domain (does not contain a notice indicating
that it is posted with permission of the copyright holder), the work
can be copied and distributed to anyone in the United States without
paying any fees or charges.  If you are redistributing or providing
access to a work with the phrase "Project Gutenberg" associated with
or appearing on the work, you must comply either with the
requirements of paragraphs 1.E.1 through 1.E.7 or obtain permission
for the use of the work and the Project Gutenberg-tm trademark as
set forth in paragraphs 1.E.8 or 1.E.9.

1.E.3.  If an individual Project Gutenberg-tm electronic work is
posted with the permission of the copyright holder, your use and
distribution must comply with both paragraphs 1.E.1 through 1.E.7
and any additional terms imposed by the copyright holder.
Additional terms will be linked to the Project Gutenberg-tm License
for all works posted with the permission of the copyright holder
found at the beginning of this work.

1.E.4.  Do not unlink or detach or remove the full Project
Gutenberg-tm License terms from this work, or any files containing a
part of this work or any other work associated with Project
Gutenberg-tm.

1.E.5.  Do not copy, display, perform, distribute or redistribute
this electronic work, or any part of this electronic work, without
prominently displaying the sentence set forth in paragraph 1.E.1
with active links or immediate access to the full terms of the
Project Gutenberg-tm License.

1.E.6.  You may convert to and distribute this work in any binary,
compressed, marked up, nonproprietary or proprietary form, including
any word processing or hypertext form.  However, if you provide
access to or distribute copies of a Project Gutenberg-tm work in a
format other than "Plain Vanilla ASCII" or other format used in the
official version posted on the official Project Gutenberg-tm web
site (www.gutenberg.net), you must, at no additional cost, fee or
expense to the user, provide a copy, a means of exporting a copy, or
a means of obtaining a copy upon request, of the work in its
original "Plain Vanilla ASCII" or other form.  Any alternate format
must include the full Project Gutenberg-tm License as specified in
paragraph 1.E.1.

1.E.7.  Do not charge a fee for access to, viewing, displaying,
performing, copying or distributing any Project Gutenberg-tm works
unless you comply with paragraph 1.E.8 or 1.E.9.

1.E.8.  You may charge a reasonable fee for copies of or providing
access to or distributing Project Gutenberg-tm electronic works
provided that

- You pay a royalty fee of 20% of the gross profits you derive from
   the use of Project Gutenberg-tm works calculated using the method
   you already use to calculate your applicable taxes.  The fee is
   owed to the owner of the Project Gutenberg-tm trademark, but he
   has agreed to donate royalties under this paragraph to the
   Project Gutenberg Literary Archive Foundation.  Royalty payments
   must be paid within 60 days following each date on which you
   prepare (or are legally required to prepare) your periodic tax
   returns.  Royalty payments should be clearly marked as such and
   sent to the Project Gutenberg Literary Archive Foundation at the
   address specified in Section 4, "Information about donations to
   the Project Gutenberg Literary Archive Foundation."

- You provide a full refund of any money paid by a user who notifies
   you in writing (or by e-mail) within 30 days of receipt that s/he
   does not agree to the terms of the full Project Gutenberg-tm
   License.  You must require such a user to return or
   destroy all copies of the works possessed in a physical medium
   and discontinue all use of and all access to other copies of
   Project Gutenberg-tm works.

- You provide, in accordance with paragraph 1.F.3, a full refund of
   any money paid for a work or a replacement copy, if a defect in
   the electronic work is discovered and reported to you within 90
   days of receipt of the work.

- You comply with all other terms of this agreement for free
   distribution of Project Gutenberg-tm works.

1.E.9.  If you wish to charge a fee or distribute a Project
Gutenberg-tm electronic work or group of works on different terms
than are set forth in this agreement, you must obtain permission in
writing from both the Project Gutenberg Literary Archive Foundation
and Michael Hart, the owner of the Project Gutenberg-tm trademark.
Contact the Foundation as set forth in Section 3 below.

1.F.

1.F.1.  Project Gutenberg volunteers and employees expend
considerable effort to identify, do copyright research on,
transcribe and proofread public domain works in creating the Project
Gutenberg-tm collection.  Despite these efforts, Project
Gutenberg-tm electronic works, and the medium on which they may be
stored, may contain "Defects," such as, but not limited to,
incomplete, inaccurate or corrupt data, transcription errors, a
copyright or other intellectual property infringement, a defective
or damaged disk or other medium, a computer virus, or computer codes
that damage or cannot be read by your equipment.

1.F.2.  LIMITED WARRANTY, DISCLAIMER OF DAMAGES - Except for the
"Right of Replacement or Refund" described in paragraph 1.F.3, the
Project Gutenberg Literary Archive Foundation, the owner of the
Project Gutenberg-tm trademark, and any other party distributing a
Project Gutenberg-tm electronic work under this agreement, disclaim
all liability to you for damages, costs and expenses, including
legal fees.  YOU AGREE THAT YOU HAVE NO REMEDIES FOR NEGLIGENCE,
STRICT LIABILITY, BREACH OF WARRANTY OR BREACH OF CONTRACT EXCEPT
THOSE PROVIDED IN PARAGRAPH F3.  YOU AGREE THAT THE FOUNDATION, THE
TRADEMARK OWNER, AND ANY DISTRIBUTOR UNDER THIS AGREEMENT WILL NOT
BE LIABLE TO YOU FOR ACTUAL, DIRECT, INDIRECT, CONSEQUENTIAL,
PUNITIVE OR INCIDENTAL DAMAGES EVEN IF YOU GIVE NOTICE OF THE
POSSIBILITY OF SUCH DAMAGE.

1.F.3.  LIMITED RIGHT OF REPLACEMENT OR REFUND - If you discover a
defect in this electronic work within 90 days of receiving it, you
can receive a refund of the money (if any) you paid for it by
sending a written explanation to the person you received the work
from.  If you received the work on a physical medium, you must
return the medium with your written explanation.  The person or
entity that provided you with the defective work may elect to
provide a replacement copy in lieu of a refund.  If you received the
work electronically, the person or entity providing it to you may
choose to give you a second opportunity to receive the work
electronically in lieu of a refund.  If the second copy is also
defective, you may demand a refund in writing without further
opportunities to fix the problem.

1.F.4.  Except for the limited right of replacement or refund set
forth in paragraph 1.F.3, this work is provided to you 'AS-IS', WITH
NO OTHER WARRANTIES OF ANY KIND, EXPRESS OR IMPLIED, INCLUDING BUT
NOT LIMITED TO WARRANTIES OF MERCHANTIBILITY OR FITNESS FOR ANY
PURPOSE.

1.F.5.  Some states do not allow disclaimers of certain implied
warranties or the exclusion or limitation of certain types of
damages. If any disclaimer or limitation set forth in this agreement
violates the law of the state applicable to this agreement, the
agreement shall be interpreted to make the maximum disclaimer or
limitation permitted by the applicable state law.  The invalidity or
unenforceability of any provision of this agreement shall not void
the remaining provisions.

1.F.6.  INDEMNITY - You agree to indemnify and hold the Foundation,
the trademark owner, any agent or employee of the Foundation, anyone
providing copies of Project Gutenberg-tm electronic works in
accordance with this agreement, and any volunteers associated with
the production, promotion and distribution of Project Gutenberg-tm
electronic works, harmless from all liability, costs and expenses,
including legal fees, that arise directly or indirectly from any of
the following which you do or cause to occur: (a) distribution of
this or any Project Gutenberg-tm work, (b) alteration, modification,
or additions or deletions to any Project Gutenberg-tm work, and (c)
any Defect you cause.


Section  2.  Information about the Mission of Project Gutenberg-tm

Project Gutenberg-tm is synonymous with the free distribution of
electronic works in formats readable by the widest variety of
computers including obsolete, old, middle-aged and new computers.
It exists because of the efforts of hundreds of volunteers and
donations from people in all walks of life.

Volunteers and financial support to provide volunteers with the
assistance they need, is critical to reaching Project Gutenberg-tm's
goals and ensuring that the Project Gutenberg-tm collection will
remain freely available for generations to come.  In 2001, the
Project Gutenberg Literary Archive Foundation was created to provide
a secure and permanent future for Project Gutenberg-tm and future
generations. To learn more about the Project Gutenberg Literary
Archive Foundation and how your efforts and donations can help, see
Sections 3 and 4 and the Foundation web page at
http://www.pglaf.org.


Section 3.  Information about the Project Gutenberg Literary Archive
Foundation

The Project Gutenberg Literary Archive Foundation is a non profit
501(c)(3) educational corporation organized under the laws of the
state of Mississippi and granted tax exempt status by the Internal
Revenue Service.  The Foundation's EIN or federal tax identification
number is 64-6221541.  Its 501(c)(3) letter is posted at
http://pglaf.org/fundraising.  Contributions to the Project
Gutenberg Literary Archive Foundation are tax deductible to the full
extent permitted by U.S. federal laws and your state's laws.

The Foundation's principal office is located at 4557 Melan Dr. S.
Fairbanks, AK, 99712., but its volunteers and employees are
scattered throughout numerous locations.  Its business office is
located at 809 North 1500 West, Salt Lake City, UT 84116, (801)
596-1887, email business@pglaf.org.  Email contact links and up to
date contact information can be found at the Foundation's web site
and official page at http://pglaf.org

For additional contact information:
     Dr. Gregory B. Newby
     Chief Executive and Director
     gbnewby@pglaf.org

Section 4.  Information about Donations to the Project Gutenberg
Literary Archive Foundation

Project Gutenberg-tm depends upon and cannot survive without wide
spread public support and donations to carry out its mission of
increasing the number of public domain and licensed works that can
be freely distributed in machine readable form accessible by the
widest array of equipment including outdated equipment.  Many small
donations ($1 to $5,000) are particularly important to maintaining
tax exempt status with the IRS.

The Foundation is committed to complying with the laws regulating
charities and charitable donations in all 50 states of the United
States.  Compliance requirements are not uniform and it takes a
considerable effort, much paperwork and many fees to meet and keep
up with these requirements.  We do not solicit donations in
locations where we have not received written confirmation of
compliance.  To SEND DONATIONS or determine the status of compliance
for any particular state visit http://pglaf.org

While we cannot and do not solicit contributions from states where
we have not met the solicitation requirements, we know of no
prohibition against accepting unsolicited donations from donors in
such states who approach us with offers to donate.

International donations are gratefully accepted, but we cannot make
any statements concerning tax treatment of donations received from
outside the United States.  U.S. laws alone swamp our small staff.

Please check the Project Gutenberg Web pages for current donation
methods and addresses.  Donations are accepted in a number of other
ways including including checks, online payments and credit card
donations.  To donate, please visit: http://pglaf.org/donate


Section 5.  General Information About Project Gutenberg-tm
electronic works.

Professor Michael S. Hart is the originator of the Project
Gutenberg-tm concept of a library of electronic works that could be
freely shared with anyone.  For thirty years, he produced and
distributed Project Gutenberg-tm eBooks with only a loose network of
volunteer support.

Project Gutenberg-tm eBooks are often created from several printed
editions, all of which are confirmed as Public Domain in the U.S.
unless a copyright notice is included.  Thus, we do not necessarily
keep eBooks in compliance with any particular paper edition.

Most people start at our Web site which has the main PG search
facility:

     http://www.gutenberg.net

This Web site includes information about Project Gutenberg-tm,
including how to make donations to the Project Gutenberg Literary
Archive Foundation, how to help produce our new eBooks, and how to
subscribe to our email newsletter to hear about new eBooks.

*** END: FULL LICENSE ***

\end{verbatim}
\end{document}
---------------------------------------------------------
Below is appended the log from the most recent compile.
You may use it to compare against a log from a new
compile to help spot differences.
---------------------------------------------------------
This is pdfeTeX, Version 3.141592-1.21a-2.2 (MiKTeX 2.4) (preloaded format=latex 2005.4.4)  4 DEC 2005 13:13
entering extended mode
**17153-t.tex
(17153-t.tex
LaTeX2e <2003/12/01>
Babel <v3.8a> and hyphenation patterns for english, french, german, ngerman, du
mylang, nohyphenation, loaded.
(C:\texmf\tex\latex\base\book.cls
Document Class: book 2004/02/16 v1.4f Standard LaTeX document class
(C:\texmf\tex\latex\base\leqno.clo
File: leqno.clo 1998/08/17 v1.1c Standard LaTeX option (left equation numbers)
) (C:\texmf\tex\latex\base\bk11.clo
File: bk11.clo 2004/02/16 v1.4f Standard LaTeX file (size option)
)
\c@part=\count79
\c@chapter=\count80
\c@section=\count81
\c@subsection=\count82
\c@subsubsection=\count83
\c@paragraph=\count84
\c@subparagraph=\count85
\c@figure=\count86
\c@table=\count87
\abovecaptionskip=\skip41
\belowcaptionskip=\skip42
\bibindent=\dimen102
)
(C:\texmf\tex\latex\amsmath\amsmath.sty
Package: amsmath 2000/07/18 v2.13 AMS math features
\@mathmargin=\skip43

For additional information on amsmath, use the `?' option.
(C:\texmf\tex\latex\amsmath\amstext.sty
Package: amstext 2000/06/29 v2.01
 (C:\texmf\tex\latex\amsmath\amsgen.sty
File: amsgen.sty 1999/11/30 v2.0
\@emptytoks=\toks14
\ex@=\dimen103
)) (C:\texmf\tex\latex\amsmath\amsbsy.sty
Package: amsbsy 1999/11/29 v1.2d
\pmbraise@=\dimen104
)
(C:\texmf\tex\latex\amsmath\amsopn.sty
Package: amsopn 1999/12/14 v2.01 operator names
)
\inf@bad=\count88
LaTeX Info: Redefining \frac on input line 211.
\uproot@=\count89
\leftroot@=\count90
LaTeX Info: Redefining \overline on input line 307.
\classnum@=\count91
\DOTSCASE@=\count92
LaTeX Info: Redefining \ldots on input line 379.
LaTeX Info: Redefining \dots on input line 382.
LaTeX Info: Redefining \cdots on input line 467.
\Mathstrutbox@=\box26
\strutbox@=\box27
\big@size=\dimen105
LaTeX Font Info:    Redeclaring font encoding OML on input line 567.
LaTeX Font Info:    Redeclaring font encoding OMS on input line 568.
\macc@depth=\count93
\c@MaxMatrixCols=\count94
\dotsspace@=\muskip10
\c@parentequation=\count95
\dspbrk@lvl=\count96
\tag@help=\toks15
\row@=\count97
\column@=\count98
\maxfields@=\count99
\andhelp@=\toks16
\eqnshift@=\dimen106
\alignsep@=\dimen107
\tagshift@=\dimen108
\tagwidth@=\dimen109
\totwidth@=\dimen110
\lineht@=\dimen111
\@envbody=\toks17
\multlinegap=\skip44
\multlinetaggap=\skip45
\mathdisplay@stack=\toks18
LaTeX Info: Redefining \[ on input line 2666.
LaTeX Info: Redefining \] on input line 2667.
)
(C:\texmf\tex\latex\amsfonts\amssymb.sty
Package: amssymb 2002/01/22 v2.2d

(C:\texmf\tex\latex\amsfonts\amsfonts.sty
Package: amsfonts 2001/10/25 v2.2f
\symAMSa=\mathgroup4
\symAMSb=\mathgroup5
LaTeX Font Info:    Overwriting math alphabet `\mathfrak' in version `bold'
(Font)                  U/euf/m/n --> U/euf/b/n on input line 132.
))
(C:\texmf\tex\generic\babel\babel.sty
Package: babel 2004/02/19 v3.8a The Babel package
 (C:\texmf\tex\generic\babel\germanb.ldf
Language: germanb 2004/02/19 v2.6k German support from the babel system

(C:\texmf\tex\generic\babel\babel.def
File: babel.def 2004/02/19 v3.8a Babel common definitions
\babel@savecnt=\count100
\U@D=\dimen112
)
\l@austrian = a dialect from \language\l@german 
Package babel Info: Making " an active character on input line 91.
)) (C:\texmf\tex\latex\soul\soul.sty
Package: soul 2003/11/17 v2.4 letterspacing/underlining (mf)
\SOUL@word=\toks19
\SOUL@lasttoken=\toks20
\SOUL@cmds=\toks21
\SOUL@buffer=\toks22
\SOUL@token=\toks23
\SOUL@spaceskip=\skip46
\SOUL@ttwidth=\dimen113
\SOUL@uldp=\dimen114
\SOUL@ulht=\dimen115
)
(17153-t.aux)
LaTeX Font Info:    Checking defaults for OML/cmm/m/it on input line 93.
LaTeX Font Info:    ... okay on input line 93.
LaTeX Font Info:    Checking defaults for T1/cmr/m/n on input line 93.
LaTeX Font Info:    ... okay on input line 93.
LaTeX Font Info:    Checking defaults for OT1/cmr/m/n on input line 93.
LaTeX Font Info:    ... okay on input line 93.
LaTeX Font Info:    Checking defaults for OMS/cmsy/m/n on input line 93.
LaTeX Font Info:    ... okay on input line 93.
LaTeX Font Info:    Checking defaults for OMX/cmex/m/n on input line 93.
LaTeX Font Info:    ... okay on input line 93.
LaTeX Font Info:    Checking defaults for U/cmr/m/n on input line 93.
LaTeX Font Info:    ... okay on input line 93.
 [1

{psfonts.map}] [1

]
LaTeX Font Info:    Try loading font information for U+msa on input line 202.
 (C:\texmf\tex\latex\amsfonts\umsa.fd
File: umsa.fd 2002/01/19 v2.2g AMS font definitions
)
LaTeX Font Info:    Try loading font information for U+msb on input line 202.

(C:\texmf\tex\latex\amsfonts\umsb.fd
File: umsb.fd 2002/01/19 v2.2g AMS font definitions
)
Overfull \hbox (2.66441pt too wide) in paragraph at lines 193--209
 \OT1/cmr/m/n/10.95 Die syn-the-ti-sche Geo-me-trie der Krei-se und Ku-geln ver
-dankt den Auf-schwung,
 []

[1

] [2] [3] [4]
Overfull \hbox (5.1156pt too wide) in paragraph at lines 385--385
[]\OT1/cmr/m/n/10.95 Seite 
 []

LaTeX Font Info:    Try loading font information for OMS+cmr on input line 386.

(C:\texmf\tex\latex\base\omscmr.fd
File: omscmr.fd 1999/05/25 v2.5h Standard LaTeX font definitions
)
LaTeX Font Info:    Font shape `OMS/cmr/m/n' in size <10.95> not available
(Font)              Font shape `OMS/cmsy/m/n' tried instead on input line 386.
 [5]
LaTeX Font Info:    Font shape `OMS/cmr/bx/n' in size <12> not available
(Font)              Font shape `OMS/cmsy/b/n' tried instead on input line 422.
 [6] [7] [8] [9] [10] [11] [12] [13]
[14] [15] [16] [17] [18] [19] [20] [21] [22] [23] [24] [25] [26] [27] [28]
[29] [30] [31] [32] [33] [34] [35] [36] [37] [38] [39] [40] [41] [42] [43]
[44] [45] [46] [47] [48] [49] [50] [51] [52] [53] [54] [55] [56] [57] [58]
[59] [60] [61] [62] [63] [64] [65] [66] [67] [68] [69] [70] [71] [72] [73]
[74] [75] [76] [77] [78] [79] [80] [81] [82] [83] [84] [85] [86] [87] [1]
[2] [3] [4] [5] [6] [7] [8] [9] (17153-t.aux)

 *File List*
    book.cls    2004/02/16 v1.4f Standard LaTeX document class
   leqno.clo    1998/08/17 v1.1c Standard LaTeX option (left equation numbers)
    bk11.clo    2004/02/16 v1.4f Standard LaTeX file (size option)
 amsmath.sty    2000/07/18 v2.13 AMS math features
 amstext.sty    2000/06/29 v2.01
  amsgen.sty    1999/11/30 v2.0
  amsbsy.sty    1999/11/29 v1.2d
  amsopn.sty    1999/12/14 v2.01 operator names
 amssymb.sty    2002/01/22 v2.2d
amsfonts.sty    2001/10/25 v2.2f
   babel.sty    2004/02/19 v3.8a The Babel package
 germanb.ldf    2004/02/19 v2.6k German support from the babel system
    soul.sty    2003/11/17 v2.4 letterspacing/underlining (mf)
    umsa.fd    2002/01/19 v2.2g AMS font definitions
    umsb.fd    2002/01/19 v2.2g AMS font definitions
  omscmr.fd    1999/05/25 v2.5h Standard LaTeX font definitions
 ***********

 ) 
Here is how much of TeX's memory you used:
 1782 strings out of 95512
 18606 string characters out of 1189449
 82856 words of memory out of 1080488
 4830 multiletter control sequences out of 60000
 14303 words of font info for 54 fonts, out of 500000 for 1000
 14 hyphenation exceptions out of 607
 27i,11n,24p,263b,296s stack positions out of 1500i,500n,5000p,200000b,32768s
PDF statistics:
 378 PDF objects out of 300000
 0 named destinations out of 300000
 1 words of extra memory for PDF output out of 65536
<C:\texmf\fonts\type1\bluesky\symbols\msbm7.pfb><C:\texmf\fonts\type1\bluesky
\cm\cmsy9.pfb><C:\texmf\fonts\type1\bluesky\cm\cmmi9.pfb> <C:\localtexmf\fonts\
pk\ljfour\ams\cmextra\dpi600\cmex9.pk><C:\texmf\fonts\type1\bluesky\cm\cmex10.p
fb><C:\texmf\fonts\type1\bluesky\symbols\msbm10.pfb><C:\texmf\fonts\type1\blues
ky\cm\cmsy6.pfb><C:\texmf\fonts\type1\bluesky\cm\cmmi8.pfb><C:\texmf\fonts\type
1\bluesky\symbols\msam10.pfb><C:\texmf\fonts\type1\bluesky\cm\cmsy8.pfb><C:\tex
mf\fonts\type1\bluesky\cm\cmmi10.pfb><C:\texmf\fonts\type1\bluesky\cm\cmbsy10.p
fb><C:\texmf\fonts\type1\bluesky\cm\cmsy10.pfb><C:\texmf\fonts\type1\bluesky\cm
\cmbx10.pfb><C:\texmf\fonts\type1\bluesky\cm\cmr9.pfb><C:\texmf\fonts\type1\blu
esky\cm\cmr6.pfb><C:\texmf\fonts\type1\bluesky\cm\cmr8.pfb><C:\texmf\fonts\type
1\bluesky\cm\cmbx12.pfb><C:\texmf\fonts\type1\bluesky\cm\cmcsc10.pfb><C:\texmf\
fonts\type1\bluesky\cm\cmr10.pfb><C:\texmf\fonts\type1\bluesky\cm\cmr12.pfb><C:
\texmf\fonts\type1\bluesky\cm\cmr17.pfb><C:\texmf\fonts\type1\bluesky\cm\cmtt10
.pfb>
Output written on 17153-t.pdf (98 pages, 458660 bytes).
